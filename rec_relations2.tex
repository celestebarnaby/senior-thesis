\chapter{Well-Typed Probabilistic Recurrence Relations}
We now turn to recurrences describing the cost of probabilistic algorithms. \cite{Karp} is premised on the notion that the cost of a stochastic algorithm may be described as a
recurrence relation of the form
\begin{align*}
T(x) = a(x) + T(h_1(x)) + \dots + T(h_n(x)).
\end{align*}
But what does this equation actually mean? To unpack this, we will attempt to assign types to this recurrence.

 Following Karp's definitions, we can immediately assign types to some of these terms:
\begin{align*}
T(x) &: \Omega_x \rightarrow \R \text{, where } \Omega_x \text{ is the sample space of all problem instances of size } x \\
T &: \prod_{x \in \R} (\Omega_x \rightarrow \R) \\ 
a &: \R \rightarrow \R
\end{align*}
Notice that the type of $T$ is dependent upon a real number $x$.
For other terms, it is less clear what the type should be. Karp alternates between saying that $h$ is a random 
variable and $h(x)$ is a random variable, so it is not immediately obvious 
what the meaning of this function is. However, since $h$ takes $x \in \R$ as an input, it would not make sense for $h$ to be a 
random variable of type $\R \rightarrow \R$. This would make $h$ a function that takes an input size and returns the 
size of the derived subproblem---thus sidestepping the stochastic part of the algorithm. Then it must be that $h(x)$ is a 
random variable on a specific input of size $x$---like $T(x)$, it takes an element of the sample space $\Omega_x$ and returns the size of the derived 
subproblem.
\begin{align*}
h_i(x) &: \Omega_x \rightarrow \R \\
h_i &: \prod_{x \in \R}(\Omega_x \rightarrow \R) 
\end{align*} 

However, there are some inconsistencies in these type assignments. To start, $T$ is a function that takes a real number and 
returns a random variable. But on the right-hand side of the relation, $T$ takes $h_i(x)$ as an argument; $h_i(x)$ is
not a real number, but a random variable. 

Thinking in terms of what we \emph{want} this recurrence to express, the arguments to the $T$'s on the righthand side of
the equation should be the sizes of the derived subproblems---that is, the result of applying $h_i(x)$ to the original input.
Then, given an input $l \in \Omega_x$, we should have
\begin{align*}
T(x)(l) &= a(x) + T(h_1(x)(l)) \dots + T(h_n(x)(l)) \\
&\text{where } h(x)(l) : \R 
\end{align*}
 Thus, the term $T(h_i(x)(l))$ typechecks. However, we now run into another problem: $T(x)(l)$ and $a(x)$ are real numbers,
but $T(h_i(x)(l))$ is a random variable of type $\Omega_{h_i(x)(l)} \rightarrow \R$. Then $T(h_i(x)(l))$ needs an argument.
Consider that $T(h_i(x)(l))$ is a random variable describing the running time of the algorithm with an input of size 
$h_i(x)(l)$. For instance, given a recurrence for randomized quicksort, $T(h_1(x)(l))$ and $T(h_2(x)(l))$ describe how long it 
takes to quicksort the left and right sublists of an input list $l$, which themselves have length $h_1(x)(l)$ and
$h_2(x)(l)$. The arguments to these random variables, then, should be the left and right sublists themselves. 

 In order for us to obtain these sublists, it will be necessary for us to define a function 
 \begin{align*}
 \hat{h_i}(x) &: \prod_{l \in \Omega_x} \Omega_{h_i(x)(l)} \\
 \hat{h_i} &: \prod_{x \in R} \ ( \prod_{l \in \Omega_x} \Omega_{h_i(x)(l)})
 \end{align*}
 such that $\hat{h_i}(x)$ takes an input of size $x$ and returns the derived subproblem of size $h_i(x)(l)$. See, then, that
 $T(h_i(x)(l))(\hat{h_i}(x)(l))$ has type $\R$. So, this leaves us with the equation
 \begin{align*}
 T(x)(l) &= a(x) + T(h_1(x)(l))(\hat{h_1}(x)(l)) \dots + T(h_n(x)(l))(\hat{h_n}(x)(l))
 \end{align*}
 which, finally, typechecks.
 
 Thus, we find that attempting to formally assigns types to Karp's probabilistic recurrences reveals a lot of hidden 
 complications and ambiguities, which Karp never addresses directly. Most notably, we find that the types of the 
 functions $T, \ h,$ and $\hat{h}$ are dependent upon a real number $x$, describing the size of problems in the sample space.
 
We define a language to write and type-check these recurrences. This language is based on $\lambda$LF---a simple system of 
dependent types (\cite{Pierce:2005aa}). However, instead of general dependent types, our language only needs to
describe sample space types $\Omega_n$ which are dependent upon a natural number $n$. Thus, we add
a constant $\lst$ to our syntax, which works as a function from expressions to types. That is, for each 
expression $e$ of type \texttt{nat}, $\lst(e)$ is a type. There are many stochastic divide-and-conquer algorithms involving sample spaces
on datatypes other than lists; this language could easily be expanded to include other constants such as
 $\texttt{tree}, \ \texttt{graph}$, etc. Note that lists can only be indexed over natural numbers (it would not make 
 sense for a list to be any other size), so we include the type $\texttt{nat}$ in our language as well.
 \[
\begin{array}{rcl}
\tau &::=& \texttt{real} \mid \texttt{nat} \mid \texttt{bool} \mid \tau \times \tau \mid \Pi x:\tau.\tau
\mid \texttt{list}(e) \\
e &::=& x  \mid \texttt{0} \mid \texttt{1} \mid \texttt{2} \mid \dotsc \mid \lambda x.e \mid e \ e \mid e + e \mid e - e \mid  e  *  e \mid e / e \mid \texttt{true} \mid \texttt{false} \mid \\
  && e  =  e \mid e < e \mid e > e \mid e \leq e \mid e \geq e \mid 
     \ifelse{e}{e}{e} \mid (e,e) \mid \\
     && \texttt{rec}(a, (h_1, \ldots, h_n), (\hat{h_1}, \ldots \hat{h_n})) 
\end{array}
\]

In this syntax, we replace the arrow type $\tau \rightarrow \sigma$ with the dependent product type
$\Pi (x:\tau).\sigma$, and introduce type families. 
These are collections of types that depend on an input: for instance,
the type $\Pi n:\texttt{nat}.\lst(n)$, where the type $\lst(n)$ is dependent upon an input $n$ of type \texttt{nat}.
As with our languages for deterministic recurrences, we need a way of defining recursive functions. 

We could attempt
to define a PCF-like language analogous to the one in chapter 3, with a $\texttt{fix}$ operator that assigns the least
fixed-point to continuous functions. However, the inclusion of dependent product types creates some unique problems
in our denotational semantics. Consider that in standard PCF, the denotation of the arrow type, $\llbracket \tau \rightarrow \sigma \rrbracket$, is the
set of continuous functions of type
$\llbracket \tau \rrbracket \rightarrow \llbracket \sigma \rrbracket$. But for dependent product types $\Pi(x:\tau).\sigma$,
this denotation does not make sense: 
since $\sigma$ may be dependent upon expressions of type $\tau$, there is no guarantee
that $\sigma$ is a well-formed type (for instance, given the type $\Pi(n:\texttt{nat}).\lst(n)$, see that
$ \lst(n)$ is not well-formed because we do not know what $n$ is).

Thus, it makes sense to alter the denotation of a dependent product type $\Pi(x:\tau).\sigma$ to instead be the set of all
continuous functions $f$ 
of type $\llbracket \tau \rrbracket \rightarrow \bigcup_{a \in \llbracket \tau \rrbracket} \llbracket \sigma \rrbracket\{x \mapsto a\}$
where $\forall a \in \llbracket \tau \rrbracket, \ f(a) \in \llbracket \sigma \rrbracket\{x \mapsto a\}$. If we again look at the 
type $\Pi(n:\texttt{nat}).\lst(n)$, we see that its denotation is the set of all functions that map a natural number $n$ 
to a list of length $n$.

But there is a problem here, stemming from the fact that our denotation of dependent product types must be a 
set of \emph{continuous} functions. This is the case because the denotation of our fix operator,
$\llbracket \texttt{fix}(\lambda f. \lambda x.e) \rrbracket$, is $fix(\llbracket \lambda f. \lambda x.e \rrbracket)$, where $fix$ 
is a function that assigns the least fixed point to continuous functions. Thus, we must be able to guarantee that the denotation of any $\lambda f. \lambda x.e$ expression is a continuous function. But continuity of a function is premised on the fact that the domain and range of
a function are CPOs. It is not necessarily the case that
 $\bigcup_{a \in \llbracket \tau \rrbracket} \llbracket \sigma \rrbracket\{ x \mapsto a\}$ is a CPO, even though 
 $\llbracket \sigma \rrbracket\{x\mapsto a\}$ is a CPO for all $a \in \llbracket \tau \rrbracket$.
 For instance, $\llbracket \lst(n) \rrbracket\{n \mapsto a\}$ is a CPO for all $a \in \N$, but 
 $\bigcup_{a \in \N} \llbracket\lst(n)\rrbracket\{n \mapsto a\}$---the union of all samples spaces $\Omega_n$---is not. 

As a result, continuity does not have any meaning for functions of type $\llbracket \Pi (x: \tau).\sigma \rrbracket$. 
So we cannot use the $\texttt{fix}$ operator on arbitrary $\lambda f. \lambda x.e$ expressions, and the whole
framework of our denotational semantics falls apart. 
 
 \begin{figure}
\[
\begin{array}{c}
\Gamma \vdash \texttt{real} :: * \\ \\
\Gamma \vdash \texttt{bool} :: *\\ \\
\Gamma \vdash \texttt{nat} :: * \\ \\ 
\dfrac{\Gamma \vdash \tau_1 :: * \ \ \ \Gamma, x:\tau_1 \vdash \tau_2 :: *}{\Gamma \vdash \Pi x:\tau_1.\tau_2 :: *} \\ \\
\dfrac{\Gamma \vdash e : \texttt{nat}}{\Gamma \vdash \lst(e) :: *} 
\end{array}
\]
\caption{Well-formed types for dependent type language.}
\end{figure}

As an alternative approach, we define a language which includes a $\texttt{rec}$ operator similar to the one in chapter 2,
 which allows to write recurrences of the form 
 $T(x)(l) = a(x) + T(h_1(x)(l))\hat{h_1}(x)(l) + \ldots + T(h_n(x)(l))\hat{h_n}(x)(l)$, where $T, \ a, \ h_i$, and $\hat{h_i}$ 
 are functions with types as described above. As in chapter 2, this approach restricts us to expressing only a subset
 of recursive functions; however, it does allow us to express the probabilistic recurrences described by Karp,
 without having to worry about the problem of requiring continuity on functions with dependent product types. 

We offer standard type judgements in Figure 2, as well as judgements for when a type is well-formed, denoted by
 $\tau :: *$, in Figure 1. We 
 also offer an equational semantics in Figure 3 and denotational semantics for types and expressions in Figures 4 and 5. 
 We may verify soundness of our denotational semantics with an argument analogous to the one given in chapter 2. 
 
  \begin{thm}
  If $\Gamma \vdash e_0 : \tau$ and $\Gamma \vdash e_1 : \tau$ and $e_0 = e_1$, then for all $\Gamma$-environments
  $\eta$,   
 \begin{align*}
 \llbracket \Gamma \vdash e_0 \rrbracket \eta 
 = \llbracket \Gamma \vdash e_1 \rrbracket \eta 
 \end{align*} 
 \end{thm}

\begin{figure}
\[
\begin{array}{c}
 \\ \\
\dfrac{}{\Gamma \vdash \texttt{0}: \texttt{real}}, \ \ \dfrac{}{\Gamma \vdash \texttt{1}: \texttt{real}}, \ldots \\ \\
\dfrac{}{\Gamma \vdash \texttt{true} : \texttt{bool}}, \ \ \dfrac{}{\Gamma \vdash \texttt{false} : \texttt{bool}} \\  \\
\dfrac{x : \tau \in \Gamma \ \ \ \Gamma \vdash \tau :: *}{\Gamma \vdash x : \tau } \\ \\ 
\dfrac{\Gamma \vdash \sigma :: * \ \ \ \Gamma, x : \sigma \vdash e : \tau}
	{\Gamma \vdash \lambda (x : \sigma).e : \Pi x:\sigma.\tau } \\ \\
\dfrac{\Gamma \vdash e_1: \Pi x:\sigma.\tau \ \ \ \Gamma \vdash e_2 : \sigma}
	{\Gamma \vdash e_1 \ e_2 : [x \mapsto e_2]\tau} \\ \\
\dfrac{\Gamma \vdash e_1 : \texttt{real} \ \ \ \Gamma \vdash e_2 : \texttt{real}}
	{\Gamma \vdash e_1 \circ e_2 : \texttt{real}}
, \circ \in \{+,-,*,/\} \\ \\ 
\dfrac{\Gamma \vdash e_1 : \texttt{real} \ \ \ \Gamma \vdash e_2 : \texttt{real}}
	{\Gamma \vdash e_1 \circ e_2 : \texttt{bool}}
	, \circ \in \{=, <, >, \geq, \leq\} \\ \\ 
\dfrac{\Gamma \vdash e_1 : \texttt{bool} \ \ \ \Gamma \vdash e_2 : \tau \ \ \ \Gamma \vdash e_3 : \tau}
	{\Gamma \vdash \ifelse{e_1}{e_2}{e_3} : \tau} \\ \\ 
\dfrac{\Gamma \vdash e_1 : \tau_1 \ \ \ \Gamma \vdash e_2 : \tau_2}{\Gamma \vdash (e_1,e_2) : \tau_1 \times \tau_2}\\ \\
\Gamma \vdash a: \Pi (n:\texttt{nat}).\texttt{nat} \\
\Gamma \vdash h_i: \Pi(n:\texttt{nat}).(\Pi (l:\lst(n)).\texttt{nat}) \\
\dfrac{\Gamma \vdash \hat{h_i}: \Pi(n: \texttt{nat}).(\Pi(l : \lst(n)).\lst(h n l))}
	{\Gamma \vdash \texttt{rec}(a, (h_1, \ldots, h_n), (\hat{h_1} \ldots \hat{h_n})) : \Pi(n:\texttt{nat}).(\Pi (l:\lst(n)).
\texttt{nat})}
\end{array}
\]
\caption{Typing for dependent type language.}
\end{figure}

\begin{figure}
\[
\begin{array}{lr}
e = e \\ \\
\texttt{0} + \texttt{0} = \texttt{0}, \ \texttt{0} + \texttt{1} = 1, \ldots, \ \texttt{3} + \texttt{5} = \texttt{8}, \ldots  \\
\text{equivalent rules for } -, \ * \text{, and } /\text{ operations.}
\\ \\
(n =n) = \texttt{true} \\ (n=m) = \texttt{false}\text{, provided } n, \ m \text{ distinct numerals.}\\ \\ 
\ifelse{\texttt{ true }}{e_1}{e_2} = e_1 \\
\ifelse{\texttt{ false }}{e_1}{e_2} = e_2 \\ \\ 
\lambda x.e = \lambda y.[x \mapsto y]e, \text{provided } y \text{ not free in } e. \\ \\ \\
\lambda x.e_1 \ e_2 = [x \mapsto e_2]e_1 \\
\texttt{rec}(\lambda x.a, ( \lambda x.\lambda l.h_1, \ldots \lambda x.\lambda l.h_n)
( \lambda x.\lambda l.\hat{h_1}, \ldots \lambda x.\lambda l.\hat{h_n}))= \\
\lambda x.(e_1 + \sum_{i=1}^n(\texttt{rec}(\lambda x.a, \sum_{i=1}^n \lambda x.\lambda l.h_i, 
\sum_{i=1}^n \lambda x.\lambda l.\hat{h_i}) h_i  \ \hat{h_i}))
\end{array}
\]
\caption{Equational Semantics for the language.}
\label{fig:typing}
\end{figure}

\begin{figure}
 \begin{align*}
\llbracket \texttt{real} \rrbracket &= \R \\
\llbracket \texttt{nat} \rrbracket &= \N\\
 \llbracket \texttt{bool} \rrbracket &= {\{true, false\}}\\
 \llbracket \Pi x:\tau.\sigma \rrbracket\eta &= \{f: \llbracket \tau \rrbracket\eta \rightarrow 
 \bigcup_{a \in \llbracket \tau \rrbracket\eta} \llbracket \sigma \rrbracket\eta\{x \mapsto a\}
 \mid  \\
 &\forall a \in \llbracket \tau \rrbracket\eta, \ f(a) \in \llbracket \sigma\rrbracket\eta\{x\mapsto a\}\} \\
 \llbracket \lst(e) \rrbracket\eta &= \Omega_{\llbracket e \rrbracket\eta} \\
 \forall n \in \N, \ \Omega_n &\text{ is the sample space of all lists of length } n. 
 \end{align*}
 \caption{Denotational semantics for types.}
 \end{figure}
 \begin{figure}
 \begin{align*}
 \llbracket \texttt{0} \rrbracket\eta &= 0, \  \llbracket \texttt{1} \rrbracket\eta = 1, \ \ldots \\
  \llbracket x : \tau \rrbracket\eta &= \eta(x) \\
  \llbracket \lambda (x : \tau) . e \rrbracket\eta &= f : \llbracket \tau \rrbracket\eta 
  \rightarrow \bigcup_{a \in \llbracket \tau \rrbracket\eta} \llbracket \sigma \rrbracket\eta\{x\mapsto a\} \\
\text{ s.t. } \forall a \in \llbracket \tau \rrbracket\eta, f(a) &= \llbracket e \rrbracket\eta\{ x \mapsto a \} 
  \in \llbracket \sigma \rrbracket\eta\{x\mapsto a\}\\
 \llbracket e_1 \ e_2 \rrbracket \eta &= \llbracket e_1 \rrbracket\eta ( \llbracket e_2 \rrbracket\eta ) \\
 \llbracket e_1 + e_2 \rrbracket\eta &= \llbracket e_1 \rrbracket\eta + \llbracket e_2 \rrbracket\eta \\
\text{Similar rules for $-, \ *, \ /$} \\
  \llbracket \texttt{true} \rrbracket\eta &= true, \ \llbracket \texttt{false} \rrbracket\eta = false \\
 \llbracket e_1 = e_2 \rrbracket\eta &= 
 \begin{cases} 
      true \text{ if } (\llbracket e_1 \rrbracket\eta = \llbracket e_2 \rrbracket\eta) \\
      false \text{  if } (\llbracket e_1 \rrbracket\eta \neq \llbracket e_2\rrbracket\eta)\\
   \end{cases} \\
\text{Similar rules for $<, \ \leq, >, \geq$} \\
  \llbracket \ifelse{e_1}{e_2}{e_3} \rrbracket \eta &= 
 \begin{cases} 
      \llbracket e_2 \rrbracket\eta \text{ if } \llbracket e_1 \rrbracket\eta = true \\
      \llbracket e_3 \rrbracket\eta \text{ if } \llbracket e_1 \rrbracket\eta = false \\ 
   \end{cases}
  \\
  \llbracket (e_1,e_2) \rrbracket\eta &= (\llbracket e_1 \rrbracket\eta, \llbracket e_2 \rrbracket\eta) \\
    \llbracket  \texttt{rec} (a,(h_1,\ldots,h_n),(\hat{h_1},\ldots,\hat{h_n})) \rrbracket\eta &= rec: \N \rightarrow 
    (\bigcup_{n\in \N} \Omega_n \rightarrow \N)   \text{ s.t. }  \\
  \forall n \in \N, \ 
    \forall l \in \Omega_n, 
    rec(n)(l) &= \sum_{j=0}^{\infty}(\sum_{(n',l') \in S^j(n,l)} \llbracket a \rrbracket\eta(n')) \\
    &S^j \text{ is defined inductively as follows:} \\
    S^0(n,l) &= \{(n, l)\} \\
    S^{j+1}(n,l) &= \bigcup_{0<i \leq n}\{(\llbracket h_i \rrbracket\eta(n')(l') , \llbracket \hat{h_i} \rrbracket\eta(n')(l')) \mid \\
     &\text{ \ \ \ \ \ \ \ \ \ \ \ }(n', l') \in S^j \}
 \end{align*}
 \caption{Denotational semantics for expressions.}
 \end{figure}

We want to gain some intuition that our denotation of a $\texttt{rec}$ expression is equivalent to the solution of
the corresponding semantic recurrence. To this end, we will show that, for a program $T$ describing a recurrence for 
quicksort, the denotation of $T$ is equal to the actual cost of quicksorting. 
 Following Karp's definition, this random variable has the form 
\begin{align*}
T(x)(l) = a(x) + T(h_1(x)(l))(\hat{h_1}(x)(l)) + T(h_2(x)(l))(\hat{h_2}(x)(l)), 
\end{align*}
where $a(x) = x-1$, the cost of one step of quicksort (i.e. comparing all other elements
in the list to the head of the list), $h_1(x)$ and $h_2(x)$ give the sizes of the left and right sublists of an input, and
$\hat{h_1}$ and $\hat{h_2}$ give the sublists themselves. 

We may express this function in our expression language
as follows:
\begin{align*}
\texttt{T} = \texttt{rec}(a, (h_1, h_2), (\hat{h_1}, \hat{h_2}))
\end{align*}
Then, we want to prove the following claim:
\begin{thm}
For all $n \in \N, \ l \in \Omega_n$, suppose $\llbracket h_1 \rrbracket\eta (n)(l)$ and $\llbracket h_2 \rrbracket\eta (n)(l)$
are equal to the number of elements in $l$ less than and greater than or equal to the head of the $l$, respectively (that is,
they accurately reflect the sizes of the subproblems of quicksort). Likewise, suppose
$\llbracket \hat{h_1} \rrbracket\eta (n)(l)$ and $\llbracket \hat{h_2} \rrbracket\eta (n)(l)$ are equal to lists containing the
elements of $l$ less than and greater than or equal to the head of $l$, respectively. 
 Then $\llbracket T \rrbracket\eta (n)(l)$ is equal to the cost
of performing quicksort on $l$. 
\end{thm} 

\begin{proof}
By induction on $n$.

\emph{Base case}: $n = 1$.  For all $l \in \Omega_n, \ l$ is a list of length $1$. Then
\begin{align*}
\llbracket \texttt{T} \rrbracket\eta (n)(l) &=  \sum_{j=0}^{\infty}(\sum_{(n',l') \in S^j(n,l)} \llbracket a \rrbracket\eta(n')) \\
&= \llbracket a \rrbracket\eta(n) + \sum_{j=1}^{\infty}(\sum_{(n',l') \in S^j(n,l)} \llbracket a \rrbracket\eta(n')) \\
&= \llbracket a \rrbracket\eta(n) + 0 = n - 1 = 0,
\end{align*}
since the subproblems of quicksorting a list of length 1 both have size $0$. See that this is the actual cost of performing
quicksort on a list of length $1$, so the claim holds.

\emph{Inductive case}: Suppose the claim holds for all $k < n$, and let $l \in \Omega_n$. Then
\begin{align*}
 \llbracket \texttt{T} \rrbracket\eta (n)(l) &=  \sum_{j=0}^{\infty}(\sum_{(n',l') \in S^j(n,l)} \llbracket a \rrbracket\eta(n')) \\
 &= \llbracket a \rrbracket\eta(n) + \sum_{j=1}^{\infty}(\sum_{(n',l') \in S^j(n,l)} \llbracket a \rrbracket\eta(n')) \\
&= (n-1) + \sum_{j=0}^{\infty}(\sum_{(n',l') \in S^j(\llbracket h_1 \rrbracket\eta(n),
\llbracket \hat{h_1} \rrbracket \eta(l))} \llbracket a \rrbracket\eta(n')) \\
 & \text{ \ \ \ \ \ \ \ \ \ \ \ \ } + \sum_{j=0}^{\infty}(\sum_{(n',l') \in S^j(\llbracket h_2 \rrbracket \eta(n),
 \llbracket \hat{h_2} \rrbracket \eta(l))} \llbracket a \rrbracket\eta(n')) \\
 &= (n-1) + \llbracket \texttt{T} \rrbracket\eta(\llbracket h_1 \rrbracket\eta(n)(l))(\llbracket \hat{h_1} \rrbracket\eta(n)(l)) \\
 &\text{ \ \ \ \ \ \ \ \ \ \ \ \ }+  \llbracket \texttt{T} \rrbracket\eta(\llbracket h_2 \rrbracket\eta(n)(l))(\llbracket \hat{h_2} \rrbracket\eta(n)(l))  
\end{align*}
By inductive hypothesis, 
\begin{align*}
&\llbracket \texttt{T} \rrbracket\eta(\llbracket h_1 \rrbracket\eta(n)(l))(\llbracket \hat{h_1} \rrbracket\eta(n)(l)) \text{ and } \\
&\llbracket \texttt{T} \rrbracket\eta(\llbracket h_2 \rrbracket\eta(n)(l))(\llbracket \hat{h_2} \rrbracket\eta(n)(l))
\end{align*}
 are equal
to the cost of performing quicksort on $\llbracket \hat{h_1} \rrbracket\eta(n)(l)$ and
$\llbracket \hat{h_2} \rrbracket\eta(n)(l)$---i.e., the left and right sublists of $l$. Then $\llbracket \texttt{T} \rrbracket\eta (n)(l)$ is equal 
to the sum of these costs, plus $n-1$; this is the cost of performing quicksort on $l$.

Therefore, the claim holds for all $n$.
\end{proof}
Thus, we have a clearer indication that the syntax and denotational semantics defined in this chapter correctly 
represent Karp's probabilistic recurrence relations.

