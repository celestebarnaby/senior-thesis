\chapter{Deterministic Recurrences as Fixed Points}
\cite[\S2]{Karp} offers several example applications of his cost-bounding theorems. However, the recurrences
he offers here do not follow the definition he set forth in the prior section. For example, a recurrence for a randomized list 
ranking algorithm is written as
\begin{align*}
T(1) &= 1 \\
T(n) &= 1 + T(3/4n).
\end{align*}
Note that this recurrence only makes sense for inputs $n$ such that
$(3/4)^in = 1$ for some $i$. Thus, we interpret the actual meaning of the general recurrence to be
$T(n) = 1 + T(\lfloor 3/4n \rfloor)$. 

The presence of an initial condition here is confusing: Karp has made no mention of initial conditions up to this point, and 
there is no clear way to write the above equations as one equation of the form $T(x) = a(x) + T(m(x))$. Simply put, 
this recurrence does not seem to be within the domain of recurrences Karp describes in section 1. Thus, we have no way
of writing this recurrence using the language defined in the previous chapter. 

In this chapter, we define a language which allows us to write and type-check the recurrence relations offered in section 2 of Karp's paper. The syntax of this language is mostly analogous to Programming Computable Functions (PCF) ---a typed functional language with a fixed-point combinator--- with  one exception: instead of a 
$\texttt{nat}$ type, we have a $\texttt{real}$ type, as this is the domain of the functions Karp discusses in this section. 
We will circumvent the issues we had with this approach in the previous chapter by writing these recurrences
using conditional statements $\ifelse{e_1}{e_2}{e_3}$, with $e_2$ corresponding to the initial condition and $e_3$
to the general condition. We will see that this approach will allow us to have strict evaluation of base operations, while
still obtaining correct solutions to recurrences. 
\[
\begin{array}{rcl}
\tau &::=& \texttt{real} \mid \texttt{bool} \mid \tau \times \tau \mid \tau \rightarrow \tau \\
e &::=& x  \mid \texttt{0} \mid \texttt{1} \mid \texttt{2} \mid \dotsc \mid \lambda x.e \mid e \ e \mid e + e \mid e - e \mid  e  *  e 
\mid e / e \mid \texttt{true} \mid \texttt{false} \mid \\
  && e  =  e \mid e < e \mid e > e \mid e \leq e \mid e \geq e \mid 
     \ifelse{e}{e}{e} \mid (e, e) \mid \\
     && \texttt{floor} \mid \texttt{fix} (\lambda f.\lambda x.e) 
\end{array}
\]

\begin{figure}
\[
\begin{array}{lr}
\dfrac{}{\Gamma \vdash \texttt{0}: \texttt{real}}, \quad \dfrac{}{\Gamma \vdash \texttt{1}: \texttt{real}}, \ldots \\
\dfrac{}{\Gamma \vdash \texttt{true} : \texttt{bool}}, \ \ \dfrac{}{\Gamma \vdash \texttt{false} : \texttt{bool}} \\ \\
\dfrac{x : \tau \in \Gamma}{\Gamma \vdash x : \tau } \\ \\
\dfrac{\Gamma, x : \sigma \vdash e : \tau}{\Gamma \vdash \lambda (x : \sigma).e : \sigma \rightarrow \tau } \\ \\ 
\dfrac{\Gamma \vdash e_1: \sigma \rightarrow \tau \ \ \ \Gamma \vdash e_2 : \sigma}{\Gamma \vdash e_1 \ e_2 : \tau} \\ \\ 
\dfrac{\Gamma \vdash e_1 : \texttt{real} \ \ \ \Gamma \vdash e_2 : \texttt{real}}{\Gamma \vdash e_1 \circ e_2 : \texttt{real}}
, \circ \in \{+,-,*,/\} \\ \\ 
\dfrac{\Gamma \vdash e_1 : \texttt{real} \ \ \ \Gamma \vdash e_2 : \texttt{real}}{\Gamma \vdash e_1 \circ e_2 : \texttt{bool}}
, \circ \in \{=, <, >, \geq, \leq\} \\ \\ 
\dfrac{\Gamma \vdash e_1 : \texttt{bool} \ \ \ \Gamma \vdash e_2 : \tau \ \ \ \Gamma \vdash e_3 : \tau}
{\Gamma \vdash \ifelse{e_1}{e_2}{e_3} : \tau} \\ \\ 
\dfrac{\Gamma \vdash e_1 : \tau_1 \ \ \ \Gamma \vdash e_2: \tau_2}{\Gamma \vdash (e_1, e_2) : \tau_1 \times \tau_2} \\ \\
\dfrac{\Gamma \vdash e : \texttt{real}}{\Gamma \vdash \texttt{floor}(e) : \texttt{real}} \\ \\
\dfrac{\Gamma \vdash \lambda f. \lambda x.e : (\tau \rightarrow \sigma) \rightarrow (\tau \rightarrow \sigma)}
{\Gamma \vdash \texttt{fix}(\lambda f. \lambda x.e) : \tau \rightarrow \sigma} 
\end{array}
\]
\caption{Typing for the language.}
\label{fig:cpo-typing}
\end{figure}

\begin{figure}
\[
\begin{array}{lr}
e = e \\ \\
\texttt{0} + \texttt{0} = \texttt{0}, \ \texttt{0} + \texttt{1} = 1, \ldots, \ \texttt{3} + \texttt{5} = \texttt{8}, \ldots  \\
\text{equivalent rules for } -, \ * \text{, and } /\text{ operations.}
\\ \\
(n =n) = \texttt{true} \\ (n=m) = \texttt{false}\text{, provided } n, \ m \text{ distinct numerals.}\\ \\ 
\ifelse{\texttt{ true }}{e_1}{e_2} = e_1 \\
\ifelse{\texttt{ false }}{e_1}{e_2} = e_2 \\ \\ 
\lambda x.e = \lambda y.[x \mapsto y]e, \text{provided } y \text{ not free in } e. \\ \\ \\
\lambda x.e_1 \ e_2 = [x \mapsto e_2]e_1 \\
\texttt{fix}(\lambda f. \lambda x.e) = [f \mapsto \texttt{fix}(\lambda f.\lambda x.e)](\lambda x.e)
\end{array}
\]
\caption{Equational Semantics for the language.}
\label{fig:cpo-eq-sem}
\end{figure}

The equational semantics of our language is given in Figure~\ref{fig:cpo-eq-sem}, and the denotational semantics is given in figures~\ref{fig:cpo-den-type} 
and~\ref{fig:cpo-den-term}.
This semantics is the standard one for PCF, where each type interprets to a complete
partially ordered set (CPO). A CPO is a set $S$ with two defining properties. Before we state these properties,
we first offer some definitions related to ordered sets. An element $a \in S$ is called the bottom element (denoted by $\perp$) if for all $s \in S, \ a \leq s$. A subset $C \subset S$ is called a chain if $C = \{c_1, c_2, c_3, \dots\}$, where 
$c_1 \leq c_2 \leq c_3 \leq \dots$. An upper bound on $C$ is an element $a \in S$ such that for all $i, \ c_i \leq a$. 
The least upper bound $\bigvee C$ is an upper bound of $C$ such that, for any upper bound $a$ of $C, \ \bigvee C \leq a$.

With these definitions in mind, we say $S$ is a CPO if
\begin{align*}
&\text{(1) $S$ has a bottom element }\perp.\\
&\text{(2) For any chain }C \subset S, \ \bigvee C \in S. 
\end{align*}
Additionally, if $P$ and $Q$ are CPOs, we say a function $f: P \rightarrow Q$ is continuous if, for every chain $C \subset P$, 
$f(\bigvee C) = \bigvee f(C)$.

\begin{figure}
 \begin{align*}
\llbracket \texttt{real} \rrbracket &= \R_{\perp} \\
 \llbracket \texttt{bool} \rrbracket &= {\{true, false\}}_{\perp} \\
\text{For any set } S, \ S_{\perp} &= S \cup \{\perp\}, \text{ where }\forall x, y\in S, \ x \leq y \iff x = \perp. \\
 \text{This is called the}&\text{ flat ordering of } S.\\
 \llbracket \tau \times \sigma \rrbracket &= \llbracket \tau \rrbracket \times \llbracket \sigma \rrbracket  \\
 \text{For all } (a, b), \ (c, d) \in \llbracket \tau \times \sigma \rrbracket &\text{, let }  (a,b) \leq (c,d) \text{ if and only if } a\leq c \text{ and } b \leq d. \\ 
 \llbracket \tau \rightarrow \sigma \rrbracket &= \{f: \llbracket \tau \rrbracket \rightarrow \llbracket \sigma \rrbracket \ : 
 \ f \text{ is continuous}\} \\
 \text{ For all } f, \ g \in \llbracket \tau \rightarrow \sigma \rrbracket &\text{, let } f \leq g \text{ if and only if } \forall x \in 
 \llbracket \tau \rrbracket, \ f(x) \leq g(x) 
 \end{align*}
 \caption{Denotational semantics for types.}
 \label{fig:cpo-den-type}
 \end{figure}
 \begin{figure}
 \begin{align*}
 \llbracket \texttt{0} \rrbracket\eta &= 0, \  \llbracket \texttt{1} \rrbracket\eta = 1, \ \ldots \\
  \llbracket x : \tau \rrbracket\eta &= \eta(x) \\
  \llbracket \lambda (x : \tau) . (e : \sigma) \rrbracket\eta &= f : \llbracket \tau \rrbracket \rightarrow \llbracket \sigma \rrbracket
\text{ s.t. } \forall d \in \llbracket \tau \rrbracket, f(d) = \llbracket e \rrbracket\eta\{ x \mapsto d \} \\
 \llbracket e_1 \ e_2 \rrbracket \eta &= \llbracket e_1 \rrbracket\eta ( \llbracket e_2 \rrbracket\eta ) \\
 \llbracket e_1 + e_2 \rrbracket\eta &= \llbracket e_1 \rrbracket\eta + \llbracket e_2 \rrbracket\eta \\
 \llbracket e_1 - e_2 \rrbracket\eta &= \llbracket e_1 \rrbracket\eta - \llbracket e_2 \rrbracket\eta \\
 \llbracket e_1 * e_2 \rrbracket\eta &= \llbracket e_1 \rrbracket\eta * \llbracket e_2 \rrbracket\eta \\
  \llbracket e_1 / e_2 \rrbracket\eta &=
  \begin{cases}
  \perp \text{ if }  \llbracket e_2 \rrbracket\eta = 0 \\
   \llbracket e_1 \rrbracket\eta / \llbracket e_2 \rrbracket\eta \text{ otherwise}
   \end{cases} \\
  \llbracket \texttt{true} \rrbracket\eta &= true, \ \llbracket \texttt{false} \rrbracket\eta = false \\
 \llbracket e_1 = e_2 \rrbracket\eta &= 
 \begin{cases} 
      true \text{ if } (\llbracket e_1 \rrbracket\eta = \llbracket e_2 \rrbracket\eta \neq \perp) \\
      false \text{  if } (\llbracket e_1 \rrbracket\eta \neq \llbracket e_2\rrbracket\eta, \llbracket e_1 \rrbracket\eta \neq \perp, \llbracket e_2 \rrbracket\eta \neq \perp)\\
      \perp \text{ otherwise}
   \end{cases} \\
\text{Similar rules for $<, \ \leq, >, \geq$} \\
  \llbracket \ifelse{e_1}{e_2}{e_3} \rrbracket \eta &= 
 \begin{cases} 
      \llbracket e_2 \rrbracket\eta \text{ if } \llbracket e_1 \rrbracket\eta = true \\
      \llbracket e_3 \rrbracket\eta \text{ if } \llbracket e_1 \rrbracket\eta = false \\
      \perp \text{      otherwise} \\
   \end{cases}
  \\
  \llbracket (e_1,e_2) \rrbracket\eta &= (\llbracket e_1 \rrbracket\eta,\llbracket e_2 \rrbracket\eta) \\
  \llbracket \texttt{floor}(e) \rrbracket\eta &= \lfloor \llbracket e \rrbracket\eta \rfloor \\
   \llbracket  \texttt{fix} (\lambda f.\lambda x.e) \rrbracket\eta &= fix\llbracket \lambda f.\lambda x.e \rrbracket\eta \\
 \text{where } fix &\text{ assigns the least fixed point to continuous functions} 
 \end{align*}
 \caption{Denotational semantics for expressions.}
 \label{fig:cpo-den-term}
 \end{figure}
 
 We will interpret $\texttt{fix}$ expressions using the CPO fixed-point theorem. This theorem states that, for a CPO $P$ and a continuous, order-preserving map $\Phi : P \rightarrow P$, the least fixed-point of $\Phi$ exists and is equal to $\bigvee \Phi^n(\perp)$.

In order to use this theorem on the interpretation of $\texttt{fix}(\lambda f. \lambda x.e): \tau \rightarrow \sigma$ expressions, 
we must first prove the following.
\begin{thm}
For all types $\tau, \ \llbracket \tau \rrbracket$ is a CPO --- that is, $\llbracket \tau \rrbracket$ is an ordered 
set with a bottom element, such that for all $ C \subset \llbracket \tau \rrbracket$, if  $C$ is a chain, then $C$ has a least 
upper bound in $\llbracket \tau \rrbracket$. 
\end{thm}
\begin{proof}
by induction on the structure of $\tau$. \\
\emph{Base Cases: }
\begin{itemize}
\item $\tau = \texttt{real}$
\item $\tau = \texttt{bool}$ \\
In these cases, $\llbracket \tau \rrbracket$ is a set with flat ordering, and is thus a CPO. 
\end{itemize}
\emph{Inductive Cases: }
\begin{itemize}
\item $\tau = \tau_1 \times \tau_2$  \\
Note $\llbracket \tau \rrbracket = \llbracket \tau_1 \rrbracket \times \llbracket \tau_2 \rrbracket$. By inductive hypothesis,
$\llbracket \tau_1 \rrbracket$ and $\llbracket \tau_2 \rrbracket$ are CPOs, and thus have bottom elements $\perp_1$ and $
\perp_2$, respectively. Thus $(\perp_1, \perp_2)$ is a bottom element of $\tau$. 

Let $C \subset \llbracket \tau \rrbracket$ be a
chain, and define
\begin{center}
$C_1 = \{x \in \llbracket \tau_1 \rrbracket \mid \exists y \in \llbracket \tau_2 \rrbracket \text{ s.t. } (x,y) \in C\},$ \\
$C_2 = \{ y \in \llbracket \tau_2 \rrbracket\mid \exists x \in \llbracket \tau_1 \rrbracket \text{ s.t. } (x,y) \in C\}$. \\ 
\end{center}
Then $C_1 \subset \llbracket \tau_1 \rrbracket$ and $C_2 \subset \llbracket \tau_2 \rrbracket$ are chains, so $\bigvee C_1$
and $\bigvee C_2$ exist. We claim that $(\bigvee C_1, \bigvee C_2)$ is the least upper bound of $C$. 

Let $(c_1, c_2) \in C$. Then $c_1 \in C_1$ and $c_2 \in C_2$, so $c_1 \leq \bigvee C_1$ and $c_2 \leq \bigvee C_2$. Thus, 
$(c_1, c_2) \leq (\bigvee C_1, \bigvee C_2)$, so $(\bigvee C_1, \bigvee C_2)$ is an upper bound of $C$.

Now let $(x,y) \in \llbracket \tau \rrbracket$ be an upper bound of C. Then $\forall c_1 \in C_1, \ c_1 \leq x$, so $x$ is an upper 
bound of $C_1$. Similarly, $y$ is an upper bound of $C_2$. Thus, it must be that $\bigvee C_1 \leq x$ and 
$\bigvee C_2 \leq y$, so $(\bigvee C_1, \bigvee C_2) \leq (x,y)$. Therefore, $(\bigvee C_1, \bigvee C_2)$ is the least upper
bound of $C$.

\item $\tau = \tau_1 \rightarrow \tau_2$ \\ 
Note $\llbracket \tau \rrbracket = \{f \mid \llbracket \tau_1 \rrbracket \rightarrow \llbracket \tau_2 \rrbracket : f \text{ is continuous}\}
$. By inductive hypothesis, $\llbracket \tau_1 \rrbracket$ and $\llbracket \tau_2 \rrbracket$ are CPOs, and thus have bottom
elements $\perp_1$ and $\perp_2$, respectively. Let $\perp: \llbracket \tau_1 \rrbracket \rightarrow \llbracket \tau_2 
\rrbracket$ be the function such that, for all $ x \in \llbracket \tau_1 \rrbracket, \ \perp(x) = \perp_2$. See, then, that 
$\perp$ is a continuous function that is less than or equal to every function in $\llbracket \tau \rrbracket$ (using pointwise
ordering), so $\perp$ is the bottom element of $\llbracket \tau \rrbracket$.  

Let $C \subset \llbracket \tau \rrbracket$ be a 
chain, and consider that for all $ x \in \tau_1, \{f(x) \mid f \in C\}$ is a chain in $\llbracket \tau_2 \rrbracket$. Call this set $F_x$,
and note that $\bigvee F_x$ exists. We define a function $g: \llbracket \tau_1 \rrbracket \rightarrow \llbracket \tau_2 \rrbracket$ 
such that $\forall x \in \llbracket \tau_1 \rrbracket, \ g(x) = \bigvee F_x$. We claim that $g = \bigvee C$. 

Let $f \in C$, and note that for all $ x \in \llbracket \tau_1 \rrbracket, \ f(x) \leq \bigvee F_x =g(x)$. Then $g$ is an upper bound of $C$.

Now let $\rho \in \llbracket \tau \rrbracket$ be an upper bound of $C$. Then for all $ x \in \llbracket \tau_1 \rrbracket, f \in C, \ f(x) \leq \rho(x)$, so $\rho(x)$ is an upper bound of $\{f(x) : f \in C\} = F_x$. Thus $\rho(x) \geq F_x = g(x), \ \forall x \in 
\llbracket \tau_1 \rrbracket$, so $g \leq \rho$. Therefore, $g$ is the least upper bound of $C$.
\end{itemize} 
\end{proof}
%

Further, we must show that the argument $\lambda f.\lambda x.e$ of a $\texttt{fix}$ expression interprets to a continuous function. It will suffice to prove type
soundness of our denotational semantics: $\lambda f\lambda x.e$ expressions have type $(\tau \rightarrow \sigma) \rightarrow (\tau \rightarrow \sigma)$, and 
$\llbracket (\tau \rightarrow \sigma) \rightarrow (\tau \rightarrow \sigma) \rrbracket$ is the set of continuous functions from $\llbracket \tau \rightarrow \sigma \rrbracket$ to
$\llbracket \tau \rightarrow \sigma \rrbracket$. 

We will see, that in order to prove type soundness for $\lambda x.e$ expressions, we will need to show that 
$\llbracket \lambda x.e \rrbracket$ is a continuous function. Thus, the statement of type soundness has some additional
complexities.

\begin{thm} 
For all expressions $e$ and environments $\Gamma$,
\begin{enumerate}
\item For all chains $a_0\leq a_1\leq\dotsb$,
\begin{align*}
\tmden{\typing\Gamma e\tau}{\extend\eta x {\bigvee_i a_i}} =
\bigvee_i\tmden{\typing\Gamma e\tau}{\extend\eta x {a_i}}.
\end{align*}
\item For all $\Gamma\vdash e : \tau$ and all $\Gamma$-environments~$\eta$,
$\tmden{\typing\Gamma e\tau}\eta\in\tyden{\tau}$.
\end{enumerate}
\end{thm}

\begin{proof}
By induction on the structure of $e$. For all cases, let $\{ a_i \}^{\infty}_{i=1}$ be a chain, and let $ a = \bigvee_i a_i$. 
Note that 2 is trivial in all but the abstraction and $\texttt{fix}$ cases.\\
 \emph{Base Cases: } 
 \begin{itemize}
 % real number constants
 \item $e \in \{ \texttt{0}, \ \texttt{1}, \ \ldots \}$
 % boolean constants
 \item $e \in \{ \texttt{true}, \ \texttt{false} \}$\\ 
 %
  In both of these cases, $\llbracket \Gamma \vdash e : \tau \rrbracket$ is a constant function from a $\Gamma$-environment
 $\eta$ to an element of $\llbracket \tau \rrbracket$, so clearly $\tmden{\typing\Gamma e\tau}{\extend\eta x {\bigvee_i a_i}} =
\bigvee_i\tmden{\typing\Gamma e\tau}{\extend\eta x {a_i}}$.

% variable
 \item $e = x : \tau$ \\ 
 We want to show that $\tmden{\typing\Gamma x\tau}{\extend\eta y {\bigvee_i a_i}} =
\bigvee_i\tmden{\typing\Gamma x\tau}{\extend\eta y {a_i}}$.

In the case that $ y \neq x$, note that
\begin{align*}
\tmden{\typing\Gamma x\tau}{\extend\eta y {\bigvee_i a_i}} &= \eta\{y\mapsto \bigvee_i a_i\}(x) \\
&= \eta (x) \\
&= \bigvee_i\tmden{\typing\Gamma x\tau}{\extend\eta y {a_i}}
\end{align*}

 In the case that $y = x$, note that
  \begin{align*}
  \llbracket \Gamma \vdash x : \tau \rrbracket\eta\{y\mapsto\bigvee_i a_i\} &= \eta\{y\mapsto \bigvee_i a_i\}(x) \\
  &= \bigvee_i a_i \\
  &=\bigvee_i \eta\{y \mapsto a_i\}(x)  \\
  &= \bigvee_i\tmden{\typing\Gamma x\tau}{\extend\eta y {a_i}}
  \end{align*}
 \end{itemize}
 \emph{Inductive Cases: }
 \begin{itemize}
 % arithmetic operations
 \item $e = e_1 \circ e_2, \ \circ \in \{+, -, *, / \}$ \\
 We assume that $\circ$ is continuous in each argument. Then 
 \begin{align*}
 \llbracket \Gamma \vdash e : \tau \rrbracket\eta\{x \mapsto \bigvee_i a_i\} &= \llbracket \Gamma\vdash e_1 : \tau 
 \rrbracket\eta\ \{x \mapsto \bigvee_i a_i\} \circ \llbracket \Gamma \vdash e_2 : \tau \rrbracket\eta\{x \mapsto \bigvee_i a_i\}\\
  \text{(by inductive hypothesis)} &=  \bigvee_i\llbracket \Gamma \vdash e_1: \tau \rrbracket\eta\{x\mapsto a_i\} \circ
  \bigvee_i\llbracket \Gamma \vdash e_2: \tau \rrbracket\eta\{x\mapsto a_i\}  \\
  &= \bigvee_i(\llbracket \Gamma \vdash e_1: \tau \rrbracket\eta\{x\mapsto a_i\} \circ 
  \llbracket \Gamma \vdash e_2: \tau \rrbracket\eta\{x\mapsto a_i\}) \\
  &= \bigvee_i \llbracket \Gamma \vdash e: \tau \rrbracket\eta\{x\mapsto a_i\}
 \end{align*}

  % equality operation
  \item $e = e_1 \circ e_2, \ \circ \in \{=,<,>,\leq,\geq\}$\\ 
  We have three cases to consider: either $\llbracket \Gamma \vdash e_1:\tau\rrbracket\{x\mapsto \bigvee_i a_i\} = 
  \perp$,  $\llbracket \Gamma \vdash e_2:\tau\rrbracket\{x\mapsto \bigvee_i a_i\} = \perp$, or both 
  $\llbracket \Gamma \vdash e_1:\tau\rrbracket\{x\mapsto \bigvee_i a_i\} \neq \perp$ and 
  $\llbracket \Gamma \vdash e_2:\tau\rrbracket\{x\mapsto \bigvee_i a_i\} \neq \perp$.
  
  In the case that $\llbracket \Gamma \vdash e_1:\tau\rrbracket\{x\mapsto \bigvee_i a_i\} = \perp$, by inductive
  hypothesis $\bigvee_i \llbracket \Gamma \vdash e_1 \rrbracket\{x\mapsto a_i\} = \perp$, so 
  $ \llbracket \Gamma \vdash e_1 \rrbracket\{x\mapsto a_i\} = \perp$ for all $i$. Then
  \begin{align*}
  \llbracket \Gamma \vdash e_1 \circ e_2\rrbracket\eta\{x \mapsto \bigvee_i a_i\}
  &=  \llbracket \Gamma \vdash e_1 \rrbracket\eta\{x \mapsto \bigvee_i a_i\} \circ
  \llbracket \Gamma \vdash e_2 \rrbracket\eta\{x \mapsto \bigvee_i a_i\}\\
  &= \perp \circ \llbracket e_2\rrbracket\{x\mapsto\bigvee_i a_i\} = \perp 
  \end{align*}
  Also,
  \begin{align*}
  \bigvee_i \llbracket \Gamma \vdash e_1 \circ e_2\rrbracket\eta\{x \mapsto a_i\}
  &= \bigvee_i \{ \llbracket \Gamma \vdash e_1 \rrbracket\eta\{x \mapsto a_i\} \circ
  \llbracket \Gamma \vdash e_2 \rrbracket\eta\{x \mapsto a_i\}\} \\
  &=\bigvee_i \{ \perp \circ
  \llbracket \Gamma \vdash e_2 \rrbracket\eta\{x \mapsto a_i\}\} \\
  &= \bigvee_i\{ \perp \} = \perp
  \end{align*}

  The same reasoning holds when $\llbracket \Gamma \vdash e_2:\tau\rrbracket\{x\mapsto \bigvee_i a_i\} = \perp$.
  
 In the case that both $\llbracket \Gamma \vdash e_1:\tau\rrbracket\{x\mapsto \bigvee_i a_i\} \neq \perp$ and 
 $\llbracket \Gamma \vdash e_2:\tau\rrbracket\{x\mapsto \bigvee_i a_i\} \neq \perp$, by inductive hypothesis 
 $\bigvee_i \llbracket \Gamma \vdash e_1:\tau\rrbracket\{x\mapsto a_i\} \neq \perp$ and
 $\bigvee_i \llbracket \Gamma \vdash e_2:\tau\rrbracket\{x\mapsto a_i\} \neq \perp$.
 Note $\llbracket \texttt{real} \rrbracket = \R_{\perp}$, so
 \begin{align*}
 \exists i_1 \forall i \geq i_1, \llbracket \Gamma \vdash e_1\rrbracket\eta\{x\mapsto a_i\} = v_1 \in \R \\
 \exists i_2 \forall i \geq i_2, \llbracket \Gamma \vdash e_2\rrbracket\eta\{x\mapsto a_i\} = v_2 \in \R 
 \end{align*}
 Let $i' = \max\{i_1, i_2\}$, and note
 \begin{align*}
  \llbracket \Gamma \vdash e_1 \circ e_2\rrbracket\eta\{x \mapsto \bigvee_i a_i\}
  &=  \llbracket \Gamma \vdash e_1 \rrbracket\eta\{x \mapsto \bigvee_i a_i\} \circ
  \llbracket \Gamma \vdash e_2 \rrbracket\eta\{x \mapsto \bigvee_i a_i\}\\
\text{(by inductive hypothesis) } &= \bigvee_i \llbracket \Gamma \vdash e_1 \rrbracket\eta\{x \mapsto  a_i\} \circ
  \bigvee_i \llbracket \Gamma \vdash e_2 \rrbracket\eta\{x \mapsto a_i\}\\
  &= \bigvee_{i \geq i'} \llbracket \Gamma \vdash e_1 \rrbracket\eta\{x \mapsto  a_i\} \circ
  \bigvee_{i \geq i'} \llbracket \Gamma \vdash e_2 \rrbracket\eta\{x \mapsto a_i\}\\
   &= v_1 \circ v_2
  \end{align*}
  Also,
  \begin{align*}
  \bigvee_i \llbracket \Gamma \vdash e_1 \circ e_2\rrbracket\eta\{x \mapsto a_i\}
  &= \bigvee_i \{ \llbracket \Gamma \vdash e_1 \rrbracket\eta\{x \mapsto a_i\} \circ
  \llbracket \Gamma \vdash e_2 \rrbracket\eta\{x \mapsto a_i\}\} \\
  &= \bigvee_{i \geq i'} \{ \llbracket \Gamma \vdash e_1 \rrbracket\eta\{x \mapsto a_i\} \circ
  \llbracket \Gamma \vdash e_2 \rrbracket\eta\{x \mapsto a_i\}\} \\
  &= \bigvee_{i \geq i'}\{ v_1 \circ v_2\} = v_1 \circ v_2
  \end{align*}  
  % conditional
 \item $e = \ \ifelse{e_1}{e_2}{e_3} : \tau$ \\ 
In the case that $\llbracket \Gamma \vdash e_1 : \texttt{bool} \rrbracket\eta\{x\mapsto  \bigvee_i a_i\} = \perp$, note that $\llbracket \Gamma \vdash e : 
\tau \rrbracket\eta\{x\mapsto  \bigvee_i a_i\} =  \perp$. By inductive hypothesis, 
$\llbracket \Gamma \vdash e_1 : \texttt{bool} \rrbracket\eta\{x\mapsto  \bigvee_i a_i\} = 
\bigvee_i\llbracket \Gamma \vdash e_1 : \tau \rrbracket\eta\{x\mapsto a_i\} = \perp$. 
Then it must be that
$\forall i, \llbracket \Gamma \vdash e_1 : \tau \rrbracket\eta\{x\mapsto a_i\} = \perp$. 
Thus, $\bigvee_i\llbracket \Gamma \vdash e : \tau \rrbracket\eta\{x\mapsto  a_i\} = \perp$, so 
$\llbracket \Gamma \vdash e : \tau \rrbracket\eta\{x\mapsto  \bigvee_i a_i\} = 
\bigvee_i\llbracket \Gamma \vdash e : \tau \rrbracket\eta\{x\mapsto  \bigvee_i a_i\}$. \\ \\
%
In the case that $\llbracket \Gamma \vdash e_1 : \texttt{bool} \rrbracket\eta\{x\mapsto  \bigvee_i a_i\} = true$, 
note that $\llbracket \Gamma \vdash e : 
\tau \rrbracket\eta\{x\mapsto  \bigvee_i a_i\} =  \llbracket \Gamma \vdash e_2 \rrbracket\eta\{x\mapsto  \bigvee_i a_i\}$. 
By inductive hypothesis, 
$\llbracket \Gamma \vdash e_1 : \texttt{bool} \rrbracket\eta\{x\mapsto  \bigvee_i a_i\} = 
\bigvee_i\llbracket \Gamma \vdash e_1 : \tau \rrbracket\eta\{x\mapsto a_i\} = true$. 
Since $\{a_i\}^{\infty}_{i=1}$ is a chain 
and $\llbracket\texttt{bool}\rrbracket$ is a flat CPO, there must exist $n \in \mathbb{N}$ such that 
$\forall i > n, \llbracket \Gamma \vdash e_1 : \texttt{bool}
\rrbracket\eta\{x\mapsto a_i\} = true$ and 
$\forall i \leq n, \llbracket \Gamma \vdash e_1 : \texttt{bool}\rrbracket\eta\{x\mapsto  a_i\} = \perp$. Thus, it must 
be that $\bigvee_i \llbracket \Gamma \vdash e : \tau \rrbracket\eta\{x\mapsto a_i\} = 
\llbracket \Gamma \vdash e_2 \rrbracket\eta\{x\mapsto  \bigvee_i a_i\} = 
\llbracket \Gamma \vdash e : \tau \rrbracket\eta\{x\mapsto  \bigvee_i a_i\}$. \\ 
%
In the case that $\llbracket \Gamma \vdash e_1 : \texttt{bool} \rrbracket\eta\{x\mapsto  \bigvee_i a_i\} = false$, by reasoning parallel to the previous case $\bigvee_i \llbracket \Gamma \vdash e : \tau \rrbracket\eta\{x\mapsto a_i\}  = 
\llbracket \Gamma \vdash e : \tau \rrbracket\eta\{x\mapsto  \bigvee_i a_i\}$.
 %pair
 \item $e = (e_1,e_2)$ 
 \begin{align*}
 \llbracket \Gamma \vdash e : \tau_1 \times \tau_2\rrbracket\eta\{x \mapsto a\}
 &= (\llbracket \Gamma \vdash e_1 : \tau_1 \rrbracket\eta\{x \mapsto a\}, \llbracket \Gamma \vdash e_2 : \tau_2 \rrbracket\eta\{x \mapsto a\}) \\
 \text{(by inductive hypothesis)} &= (\bigvee_i \llbracket \Gamma \vdash e_1 : \tau_1 \rrbracket\eta\{x \mapsto a_i\},
 \bigvee_i \llbracket \Gamma \vdash e_2 : \tau_2 \rrbracket\eta\{x \mapsto a_i\}) \\
 &= \bigvee_i ( \llbracket \Gamma \vdash e_1 : \tau_1 \rrbracket\eta\{x \mapsto a_i\},
  \llbracket \Gamma \vdash e_2 : \tau_2 \rrbracket\eta\{x \mapsto a_i\}) \\
  &= \bigvee_i \llbracket \Gamma \vdash (e_1, e_2): \tau_1 \times \tau_2 \rrbracket\eta\{x\mapsto a_i\}
  \end{align*}
  %floor
  \item $e = \texttt{floor}(e_1)$
  \begin{align*}
   \llbracket \Gamma \vdash e \rrbracket\eta\{x \mapsto a\} &= 
   \lfloor \llbracket e_1 \rrbracket\eta\{x \mapsto a\} \rfloor \\
  \text{(by inductive hypothesis)} &= \bigvee_i \lfloor \llbracket e_1 \rrbracket\eta\{x \mapsto a_i\} \rfloor
   \end{align*}
% application
 \item $ e = e_1 \ e_2$ \\ 
 We use Lemma 5 in this case, which is stated after the proof.
 
Let $\alpha = \llbracket \Gamma \vdash e_1 : \sigma \rightarrow \tau \rrbracket\eta$,  $\beta = \llbracket \Gamma \vdash e_2 : \sigma \rrbracket\eta$. By inductive hypothesis, 
\begin{align*}
\alpha\{x\mapsto \bigvee _i a_i\} &= \bigvee_i \alpha\{x \mapsto a_i\} \text{ and }\\
\beta\{x\mapsto \bigvee _j a_j\} &= \bigvee_j \beta\{x \mapsto a_j\} 
\end{align*}
For all $i, \ \alpha\{x \mapsto a_i\} \in \llbracket \sigma \rightarrow \tau \rrbracket$, which is 
 the set of continuous functions from $\llbracket \sigma \rrbracket$ to $\llbracket \tau \rrbracket$. Further, since
 $\{a_i\}_i$ is a chain, $\{\alpha\{x \mapsto a_i\}\}_i$ is a chain as well. Then, since $\llbracket \sigma \rightarrow 
 \tau\rrbracket$ is a CPO, the supremum of $\{\alpha\{x \mapsto a_i\}\}_i$ is an element of $\llbracket \sigma \rightarrow 
 \tau\rrbracket$. That is, $\bigvee \alpha\{x \mapsto a_i\}$ is a continuous function from 
 $\llbracket \sigma \rrbracket$ to $\llbracket \tau \rrbracket$. Then
\begin{align*}
\llbracket \Gamma \vdash e : \tau \rrbracket\eta\{x \mapsto \bigvee_j a_j\} &=
	\alpha\{x\mapsto \bigvee _i a_i\}(\beta\{x\mapsto \bigvee _j a_j\}) \\
&=  (\bigvee_i \alpha\{x \mapsto a_i\})(\bigvee_j \beta\{x \mapsto a_j\}) \\
\text{(by continuity of $\bigvee_i \alpha\{x \mapsto a_i\}$) }&= \bigvee_j(\bigvee_i \alpha\{x \mapsto a_i \})(\beta\{ x \mapsto a_j\}) \\
\text{(by Lemma 5)}&= \bigvee_i \alpha\{x \mapsto a_i \}(\beta\{ x \mapsto a_i\}) \\
&= \bigvee_i \llbracket \Gamma \vdash e : \tau\rrbracket\eta\{x \mapsto a_i\}
\end{align*}
 % abstraction
 \item $e = \lambda (y : \tau) . (e' : \tau')$
 \begin{enumerate}
 \item Then $\llbracket \Gamma \vdash e : \tau \rightarrow \tau' \rrbracket\eta\{x\mapsto  \bigvee_i a_i\}$ 
 is a function $f: \llbracket \tau \rrbracket \rightarrow \llbracket \tau'  \rrbracket$ such that 
 $\forall d \in \llbracket \tau \rrbracket, \ f(d) = \llbracket \Gamma \ y : \tau \vdash e' : \tau 
 \rrbracket\eta\{x\mapsto  \bigvee_i a_i\}\{y \mapsto d\}$. We want to show that
 \begin{align*}
 \tmden{\typing\Gamma e\tau \rightarrow \tau'}{\extend\eta x {\bigvee_i a_i}} &=
\bigvee_i\tmden{\typing\Gamma e\tau \rightarrow \tau'}{\extend\eta x {a_i}}
\end{align*}
 
Let $d \in \llbracket \tau \rrbracket$, and consider that
\begin{align*}
\llbracket \Gamma \vdash e : \tau \rightarrow \tau' \rrbracket\eta\{x\mapsto  \bigvee_i a_i\}(d) 
&= \llbracket \Gamma.y : \tau \vdash e' : \tau'\rrbracket\eta\{x\mapsto  \bigvee_i a_i\}
\{y \mapsto d\} \\
&= \llbracket \Gamma.y : \tau \vdash e' : \tau'\rrbracket(\eta\{y \mapsto d\})\{x\mapsto  \bigvee_i a_i\} \\
\text{(by inductive hypothesis)}&=\bigvee_i\llbracket \Gamma.y:\tau\vdash e':\tau'\rrbracket(\eta\{y\mapsto d\})
\{x\mapsto a_i\} \\
&=\bigvee_i\llbracket \Gamma.y:\tau\vdash e':\tau'\rrbracket(\eta\{x\mapsto a_i\})\{y\mapsto d\} \\
&= \bigvee_i \llbracket \Gamma \vdash \lambda y : \tau.e' : \tau' \rrbracket \eta\{x\mapsto a_i\}(d) \\
&= \bigvee_i \llbracket \Gamma \vdash e : \tau \rightarrow \tau' \rrbracket \eta\{x\mapsto a_i\}(d) 
\end{align*}
%
 \item We want to show that $\llbracket \Gamma \vdash e : \tau \rightarrow \tau' \rrbracket\eta \in \llbracket \tau \rightarrow 
 \tau'\rrbracket$---that is, it is a continuous function from $\llbracket \tau \rrbracket$ to $\llbracket \tau' \rrbracket$. Let 
 $\{a_i\}^{\infty}_{i=1}$ be a chain of elements in $\llbracket \tau \rrbracket$, and see that
 \begin{align*}
 \llbracket \Gamma \vdash e : \tau \rightarrow \tau' \rrbracket\eta(\bigvee_i a_i) &= \llbracket \Gamma.x : \tau 
 \vdash e' : \tau'\rrbracket\eta \{x \mapsto \bigvee_i a_i\} \\
 \text{(by inductive hypothesis)} &= \bigvee_i \llbracket \Gamma.x : \tau \vdash e' : \tau'\rrbracket\eta\{x\mapsto a_i\}\\
 &= \bigvee_i \llbracket \Gamma \vdash e : \tau \rightarrow \tau' \rrbracket\eta(a_i) 
 \end{align*}
 \end{enumerate} 
 % fix operation
 \item $e = \texttt{fix} (\lambda f.\lambda y.e)$ \\
  We use several Lemmas 6 and 7 for this case, which are stated after the proof.
 \begin{enumerate}
\item Let $F_i = \llbracket \Gamma \vdash \lambda f. \lambda y.e\rrbracket\eta\{x\mapsto a_i\}$, and consider that 
 \begin{align*}
 \llbracket \texttt{fix} (\lambda f.\lambda y.e)\rrbracket\eta\{x\mapsto \bigvee_i a_i\} &= 
 fix(\llbracket \lambda f. \lambda y.e \rrbracket\eta\{x\mapsto \bigvee_i a_i\})\\
 \text{(by inductive hypothesis)} &= fix(\bigvee_i(F_i)) \\
 &= \bigvee_j(\bigvee_i((F_i)^j )\perp) \\
 &= \bigvee_j(\bigvee_i((F_i)^j \perp)) \\ 
\text{(by Lemma \ref{lem1})} &= \bigvee_i(\bigvee_j((F_i)^j \perp)) \\ 
&= \bigvee_i(fix(F_i))\\
&= \bigvee_i \llbracket \Gamma \vdash \texttt{fix}(\lambda f \lambda y.e)\rrbracket\eta\{x\mapsto a_i\}
 \end{align*}
 %
 \item Let $F = \llbracket \Gamma \vdash \lambda f. \lambda x.e\rrbracket\eta$. We want to show that 
 \begin{align*}
 \llbracket\Gamma\vdash\texttt{fix}(\lambda f \lambda x.e) :  \sigma 
 \rightarrow\tau\rrbracket\eta \in \llbracket \sigma \rightarrow \tau \rrbracket.
 \end{align*}
 That is, it is a continuous function from 
 $\llbracket \sigma \rrbracket$ to $\llbracket \tau \rrbracket$. Let $\{a_i\}^{\infty}_{i=1}$ be a chain of elements in $\llbracket \sigma \rrbracket$,  and see that 
 \begin{align*}
 (\llbracket\Gamma\vdash\texttt{fix}(\lambda f \lambda x.e) : \tau \rightarrow\tau\rrbracket\eta)( \bigvee_i a_i) &= 
 fix(F)(\bigvee_i a_i)\\ 
 &=(\bigvee_j F^j \perp)(\bigvee_i a_i)\\
 &= \bigvee_j (F^j \perp(\bigvee_i a_i))\\ 
 \text{(by continuity of $F^j \perp$) }&= \bigvee_j(\bigvee_i(F^j \perp(a_i)))\\ 
 \text{(by Lemma \ref{lem2})} &=\bigvee_i(\bigvee_j(F^j \perp(a_i))) \\
 &= \bigvee_i(fix(F)(a_i))\\
 &= \bigvee_i(\llbracket\Gamma\vdash\texttt{fix}(\lambda f \lambda x.e) : \tau \rightarrow\tau\rrbracket\eta(a_i))
 \end{align*}
 \end{enumerate}
 \end{itemize}
 Thus, by definition, $\llbracket \Gamma \vdash \texttt{fix}(\lambda f \lambda x.e) \rrbracket\eta$ is a continuous function
 from $\llbracket \tau \rrbracket$ to $\llbracket \tau \rrbracket$.
 \end{proof}
 % Lemmas
 \begin{lemma}
 Let $\{f_i\}_i$ be a chain of functions of type $\llbracket \sigma \rrbracket \to \llbracket \tau \rrbracket$, and let
 $\{x_i\}_i$ be a chain in $\llbracket \sigma \rrbracket$. Then
 \begin{align*}
 \bigvee_j\{(\bigvee_i f_i)(x_j)\} = \bigvee_i \{f_i (x_i)\}
 \end{align*}
 \end{lemma}
 \begin{proof}
 $\Rightarrow$ Note that for all $j$,
 \begin{align*}
 (\bigvee_i f_i)(x_j) &= (\bigvee_{i \geq j} f_i)(x_j) \\
 &= \bigvee_{i \geq j} \{f_i(x_j)\} \\
 &\leq \bigvee_{i \geq j} \{f_i(x_i)\} \\
 &= \bigvee_{i} \{f_i(x_i)\}
 \end{align*}
 Thus, $\bigvee_j\{(\bigvee_i f_i)(x_j)\} \leq \bigvee_i \{f_i (x_i)\}$.
 
$\Leftarrow$ Note, also, that for all $i'$,
 \begin{align*}
(f_{i'}) (x_{i'}) &\leq (\bigvee_i f_i) (x_{i'}) \\
&\leq \bigvee_j \{(\bigvee_i f_i)(x_j)\}
\end{align*}
Therefore, $\bigvee_j\{(\bigvee_i f_i)(x_j)\} \geq \bigvee_i \{f_i (x_i)\}$, so 
$\bigvee_j\{(\bigvee_i f_i)(x_j)\} = \bigvee_i \{f_i (x_i)\}$
 \end{proof}
 
 \begin{lemma} $\bigvee_j(\bigvee_i((F_i)^j \perp))= \bigvee_i(\bigvee_j((F_i)^j \perp))$
 \label{lem1}
 \end{lemma}
 \begin{proof}
 $\Rightarrow$ We want to show that\\ \\ \centerline{$\bigvee_j(\bigvee_i((F_i)^j \perp))
  \geq  \bigvee_i(\bigvee_j((F_i)^j \perp))$} \\ \\ First, notice that $\forall i, j \bigvee_j(\bigvee_i
 ((F_i)^j \perp)) \geq (F_i)^j \perp$. \\
 Suppose, now, that 
 \begin{align*}
 &\bigvee_j(\bigvee_i((F_i)^j \perp)) < \bigvee_i(\bigvee_j((F_i)^j \perp))\\ 
 \Rightarrow \exists i \text{ s.t. \ \  \ } &\bigvee_j(\bigvee_i((F_i)^j \perp)) < \bigvee_j (F_i)^j \\
 \Rightarrow \exists i, j \text{ s.t. } &\bigvee_j(\bigvee_i((F_i)^j \perp)) < (F_i)^j \perp  \ \
 \Rightarrow\Leftarrow
 \end{align*}
 Thus, it must be that $\bigvee_j(\bigvee_i((F_i)^j \perp)) \geq  \bigvee_i(\bigvee_j((F_i)^j \perp))$ \\ \\
 $\Leftarrow$ By parallel reasoning, $\bigvee_j(\bigvee_i((F_i)^j \perp)) \leq  \bigvee_i(\bigvee_j((F_i)^j  \perp))$ \\ \\
 Therefore, $\bigvee_j(\bigvee_i((F_i)^j \perp)) = \bigvee_i(\bigvee_j((F_i)^j \perp))$ 
 \end{proof}
 
 \begin{lemma}
 \label{lem2}
 $\bigvee_j(\bigvee_i(F^j \perp(x_i)))= \bigvee_i(\bigvee_j(F^j \perp(x_i)))$
 \end{lemma}
 \begin{proof} 
 By reasoning similar to Lemma \ref{lem1}. 
 \end{proof}
 %

 We will prove equational soundness of our denotational semantics, in order to ensure that
 our interpretation of recurrences reflects their equational meaning.
  \begin{thm}
  If $\Gamma \vdash e_0 : \tau$, $\Gamma \vdash e_1 : \tau$, and $e_0 = e_1$, then for all $\Gamma$-environments
  $\eta$,   
 \begin{align*}
 \llbracket \Gamma \vdash e_0 \rrbracket \eta 
 = \llbracket \Gamma \vdash e_1 \rrbracket \eta 
 \end{align*} 
 \end{thm}
 
\begin{proof}
 We will verify only the equation for $\texttt{fix}$ expressions, as all others are trivial.

We may show by a simple inductive argument that 
$\llbracket [x \mapsto e']e\rrbracket\eta = \llbracket e \rrbracket\eta\{x \mapsto \llbracket e' \rrbracket\eta\}$.
Let $\texttt{F} = \lambda f \lambda x.e$, and see that
\begin{align*}
\llbracket [f \mapsto \texttt{fix}(\texttt{F})]\lambda x.e\rrbracket\eta 
&= \llbracket \lambda x.e\rrbracket\eta\{f \mapsto \llbracket \texttt{fix}(\texttt{F})\rrbracket\} \\
&=\llbracket \lambda x.e\rrbracket\eta\{f \mapsto fix\}(\llbracket \texttt{F} \rrbracket\eta)\} \\
&= \llbracket \lambda f.\lambda x.e \rrbracket\eta(fix(\llbracket \texttt{F}\rrbracket\eta)\} \\
&= \llbracket F \rrbracket\eta(fix(\llbracket \texttt{F}\rrbracket\eta)) \\ 
&= fix(\llbracket \texttt{F}\rrbracket\eta) =  \llbracket \texttt{fix}(\lambda f. \lambda x.e)\rrbracket \eta
\end{align*}
\end{proof}
 
We will look at a recurrence Karp offers in this section as an example, to check whether recurrences in our 
language have the same solution. First, consider the following recurrence:
\begin{align*}
T(1) &= 0 \\
T(n) &= 1 + T(pn), \ 0 < p < 1
\end{align*} 
Again, we assume that the actual meaning of the general recurrence is $T(n) = 1 + T(\lfloor pn \rfloor)$.
We may express this recurrence in our language as follows:
\begin{align*}
\texttt{T} &= \texttt{fix}(\lambda f. \lambda x.\ifelse{x=1}{0}{1 + f(\texttt{floor}(p*x))})
 \end{align*}
 We want to show that the interpretation of this expression is equal to the solution of the recurrence described above. 
 First, let 
 \begin{align*}
 \texttt{F} =\lambda f. \lambda x.\ifelse{x=1}{0}{1 + f(\texttt{floor}(p*x))},
 \end{align*}
 and note the following: 
 \begin{align*}
 \llbracket \texttt{T} \rrbracket &= fix(\llbracket \texttt{F} \rrbracket) \\
 \llbracket \texttt{F} \rrbracket &= F : \llbracket \texttt{real} \rightarrow \texttt{real}\rrbracket \rightarrow \llbracket \texttt{real} \rightarrow \texttt{real}\rrbracket \\ 
 &\text{ s.t. } \forall \alpha \in \llbracket \texttt{real}\rrbracket \rightarrow \llbracket \texttt{real} \rrbracket, \\   
 &F(\alpha) = \llbracket\lambda x.\ifelse{x=1}{0}{1 + f(\texttt{floor}(p*x))}\rrbracket\{f \mapsto \alpha\} \\
 \llbracket \texttt{real}\rrbracket &= \R_{\perp}  
 \end{align*}
 By the CPO fixed-point theorem,  
\begin{align*}
 fix(F) &= \bigvee_n(F^n (\perp)) \\
&= \lim_{n \to \infty}\{F^n (\perp)\}
 \end{align*}
 \begin{lemma}
 $\forall x \in \R_{\perp}, \ F^{n+1}(\perp)(x) =  
 \begin{cases}
 0 \text{ if }\lfloor x \rfloor = 1  \\
 1 \text{ if } \lfloor px  \rfloor= 1 \\
 \dots \\
 n  \text{ if }  \lfloor p^{n}x \rfloor = 1  \\
 \perp \text{ otherwise}
 \end{cases}$
 \end{lemma}
\begin{proof}

\emph{Base Case: } $n = 0$
\begin{align*}
F^{n+1}(\perp)(x) &= F(\perp) \\
&= 
\begin{cases}
0 \text{ if }\lfloor x \rfloor=1 \\
1 + \perp(\lfloor px \rfloor) \text{ otherwise}
\end{cases} \\
&=
\begin{cases}
0 \text{ if } \lfloor x \rfloor=1 \\
\perp \text{ otherwise}
\end{cases} 
\end{align*}
Note $p^{n}x = p^0x = x$, so trivially the claim holds. \\
\emph{Inductive Case: } Suppose the claim holds for all $k < n$.
\begin{align*}
F^{n+1}(\perp)(x) &=  
\begin{cases}
0 \text{ if } \lfloor x \rfloor =1 \\
1 + F^{n}(\perp)(px) \text{ otherwise}
\end{cases} 
\end{align*}
By inductive hypothesis,
\begin{align*}
F^{n}(\perp)(px) &= F^{(n-1) + 1}(\perp)(px) \\
&=  
 \begin{cases}
 0 \text{ if } \lfloor px \rfloor = 1 \\
 1 \text{ if } \lfloor  p^2x \rfloor = 1  \\
 \dots \\
 n - 1 \text{ if }  \lfloor p^{n}x \rfloor =1  \\
 \perp \text{ otherwise}
 \end{cases}
 \end{align*}
\end{proof}
See, then, that
\begin{align*}
F^{n+1}(\perp)(x) &=  
\begin{cases}
0 \text{ if } \lfloor x \rfloor =1 \\
1 \text{ if } \lfloor px \rfloor = 1 \\
2 \text{ if } \lfloor p^2x \rfloor = 1  \\
\dots \\
n \text{ if } \lfloor p^{n}x \rfloor = 1  \\
\perp \text{ otherwise } 
\end{cases}
\end{align*}
Therefore, the claim holds for all $n$.

Then $\llbracket \texttt{T} \rrbracket = \lim_{n \to \infty}\{F^n (\perp)\}$ is defined for all $x \in \R$ as
\begin{align*}
\lim_{n \to \infty}\{F^n (\perp)\}(x) = k
\end{align*}
Where $\lfloor p^kx \rfloor =1$. Then $k = \lfloor \log_{p}1/x \rfloor$, so
 $\llbracket \texttt{T} \rrbracket(x) = \lfloor \log_{p}1/x \rfloor= T(x)$. Thus, $\llbracket \texttt{T} \rrbracket = T$.
 



