\chapter{Expression Language}
In section 2, Karp offers several example applications of his cost-bounding theorems. However, the recurrences
he offers here do not follow the definition he set forth in the prior section. For example, a recurrence for a randomized list 
ranking algorithms written as
\begin{align*}
T(1) &= 1 \\
T(n) &= 1 + T(3/4n)
\end{align*}
The presence of an initial condition here is puzzling: Karp has made no mention of initial conditions up to this point, and 
there is no clear way to write the above equations as one equation of the form $T(x) = a(x) + T(m(x))$. Simply put, 
this recurrence does not seem to be within the domain of recurrences Karp describes in section 1. Thus, we have no way
of writing this recurrence using the expression language defined in the previous chapter. 

In this chapter, we define an expression language which will allow us to write and type-check the recurrence relations offered 
in section 2 of Karp's paper. The syntax of this language is mostly analogous to PCF, with one key exception: instead of a 
$\texttt{nat}$ type, we have a $\texttt{real}$ type, as this is the domain of the functions Karp discusses in this section. 
We will circumvent the issues we had with this approach in the previous chapter by writing these recurrences
using conditional statements $\ifelse{e_1}{e_2}{e_3}$, with $e_2$ corresponding to the initial condition and $e_3$
to the general condition. We will see that this approach will allow us to have strict evaluation of addition, while
still obtaining correct solutions to recurrences.\\ 
\[
\begin{array}{rcl}
\tau &::=& \texttt{real} \mid \texttt{bool} \mid \tau \times \tau \mid \tau \rightarrow \tau \\
e &::=& x  \mid \texttt{0} \mid \texttt{1} \mid \texttt{2} \mid \dotsc \mid \lambda x.e \mid e \ e \mid e + e \mid e - e \mid  e  *  e 
\mid e / e \mid \texttt{true} \mid \texttt{false} \mid \\
  && e  =  e \mid e < e \mid e > e \mid e \leq e \mid e \geq e \mid 
     if \ e \ then \ e \ else \ e \mid \texttt{fix} (\lambda f.\lambda x.e) \\
\end{array}
\]

\section{Type System}
\begin{figure}[H]
\[
\begin{array}{lr}
\dfrac{}{\Gamma \vdash \texttt{0}: \texttt{real}}, \ \ \dfrac{}{\Gamma \vdash \texttt{1}: \texttt{real}}, \ldots \\
\dfrac{}{\Gamma \vdash \texttt{true} : \texttt{bool}}, \ \ \dfrac{}{\Gamma \vdash \texttt{false} : \texttt{bool}} \\ \\ \\
\dfrac{x : \tau \in \Gamma}{\Gamma \vdash x : \tau } \\ \\ \\
\dfrac{\Gamma, x : \sigma \vdash e : \tau}{\Gamma \vdash \lambda (x : \sigma).e : \sigma \rightarrow \tau } \\ \\ \\ 
\dfrac{\Gamma \vdash e_1: \sigma \rightarrow \tau \ \ \ \Gamma \vdash e_2 : \sigma}{\Gamma \vdash e_1 \ e_2 : \tau} \\ \\ \\
\dfrac{\Gamma \vdash e_1 : \texttt{real} \ \ \ \Gamma \vdash e_2 : \texttt{real}}{\Gamma \vdash e_1 \circ e_2 : \texttt{real}}
, \circ \in \{+,-,*,/\} \\ \\ \\
\dfrac{\Gamma \vdash e_1 : \texttt{real} \ \ \ \Gamma \vdash e_2 : \texttt{real}}{\Gamma \vdash e_1 \circ e_2 : \texttt{bool}}
, \circ \in \{=, <, >, \geq, \leq\} \\ \\ \\
\dfrac{\Gamma \vdash e_1 : \texttt{bool} \ \ \ \Gamma \vdash e_2 : \tau \ \ \ \Gamma \vdash e_3 : \tau}
{\Gamma \vdash \ifelse{e_1}{e_2}{e_3} : \tau} \\ \\ \\
\dfrac{\Gamma \vdash \lambda f. \lambda x.e : (\tau \rightarrow \tau) \rightarrow (\tau \rightarrow \tau)}
{\Gamma \vdash \texttt{fix}(\lambda f. \lambda x.e) : \tau \rightarrow \tau} \\ \\ \\ 
\end{array}
\]
\end{figure}

\section{Equational Semantics}
\begin{figure}[H]
\[
\begin{array}{lr}
e = e \\ \\
\texttt{0} + \texttt{0} = \texttt{0}, \ \texttt{0} + \texttt{1} = 1, \ldots, \ \texttt{3} + \texttt{5} = \texttt{8}, \ldots  \\
\text{equivalent rules for } -, \ * \text{, and } /\text{ operations.}
\\ \\
(n =n) = \texttt{true} \\ (n=m) = \texttt{false}\text{, provided } n, \ m \text{ distinct numerals.}\\ \\ 
\ifelse{\texttt{ true }}{e_1}{e_2} = e_1 \\
\ifelse{\texttt{ false }}{e_1}{e_2} = e_2 \\ \\ 
\lambda x.e = \lambda y.e\{x \mapsto y \}, \text{provided } y \text{ not free in } e. \\ \\ \\
\lambda x.e_1 \ e_2 = e_1\{x \mapsto e_2\} \\
\texttt{fix}(\lambda f. \lambda x.e) = (\lambda x.e)\{f \mapsto \texttt{fix}(\lambda f.\lambda x.e)\}
\end{array}
\]
\end{figure}

\section{Denotational Semantics}
The denotational semantics of our expression language is  standard, with types interpreting to flat-ordered sets. 
That is, each type interprets to a set with a bottom element, $\perp$, where each element of the set is comparable
only to $\perp$. 
\begin{figure}[H]
 \begin{align*}
\llbracket \texttt{real} \rrbracket &= \R_{\perp} \\
 \llbracket \texttt{bool} \rrbracket &= {\{true, false\}}_{\perp} \\
 \llbracket \tau \times \sigma \rrbracket &= \llbracket \tau \rrbracket \times \llbracket \sigma \rrbracket  \\
 \text{For } \forall (a, b), \ (c, d) \in \llbracket \tau \times \sigma \rrbracket &\text{, let }  (a,b) \leq (c,d) \text{ if and only if } a<c \text{ and } b<d. \\ 
 \llbracket \tau \rightarrow \sigma \rrbracket &= \{f: \llbracket \tau \rrbracket \rightarrow \llbracket \sigma \rrbracket \ : 
 \ f \text{ is continuous}\} \\
 \text{ For } \forall f, \ g \in \llbracket \tau \rightarrow \sigma \rrbracket &\text{, let } f \leq g \text{ if and only if } \forall x \in 
 \llbracket \tau \rrbracket, \ f(x) \leq g(x) \\
 %\end{align*}
 %\end{figure}
 %\begin{figure}[H]
 %\begin{align*}
 \llbracket \texttt{0} \rrbracket\eta &= 0, \  \llbracket \texttt{1} \rrbracket\eta = 1, \ \ldots \\
  \llbracket x : \tau \rrbracket\eta &= \eta(x) \\
  \llbracket \lambda (x : \tau) . (e : \sigma) \rrbracket\eta &= f : \llbracket \tau \rrbracket \rightarrow \llbracket \sigma \rrbracket
\text{ s.t. } \forall d \in \llbracket \tau \rrbracket, f(d) = \llbracket e \rrbracket\eta\{ x \mapsto d \} \\
 \llbracket e_1 \ e_2 \rrbracket \eta &= \llbracket e_1 \rrbracket\eta ( \llbracket e_2 \rrbracket\eta ) \\
 \llbracket e_1 + e_2 \rrbracket\eta &= \llbracket e_1 \rrbracket\eta + \llbracket e_2 \rrbracket\eta \\
 \llbracket e_1 - e_2 \rrbracket\eta &= \llbracket e_1 \rrbracket\eta - \llbracket e_2 \rrbracket\eta \\
 \llbracket e_1 * e_2 \rrbracket\eta &= \llbracket e_1 \rrbracket\eta * \llbracket e_2 \rrbracket\eta \\
  \llbracket e_1 / e_2 \rrbracket\eta &=
  \begin{cases}
  \perp \text{ if }  \llbracket e_2 \rrbracket\eta = 0 \\
   \llbracket e_1 \rrbracket\eta / \llbracket e_2 \rrbracket\eta \text{ otherwise}
   \end{cases} \\
  \llbracket \texttt{true} \rrbracket\eta &= true, \ \llbracket \texttt{false} \rrbracket\eta = false \\
 \llbracket e_1 = e_2 \rrbracket\eta &= 
 \begin{cases} 
      true \text{ if } (\llbracket e_1 \rrbracket\eta = \llbracket e_2 \rrbracket\eta \neq \perp) \\
      false \text{  if } (\llbracket e_1 \rrbracket\eta \neq \llbracket e_2\rrbracket\eta, \llbracket e_1 \rrbracket\eta \neq \perp, \llbracket e_2 \rrbracket\eta \neq \perp)\\
      \perp \text{ otherwise}
   \end{cases} \\
\text{Similar rules for $<, \ \leq, >, \geq$} \\
  \llbracket \ifelse{e_1}{e_2}{e_3} \rrbracket \eta &= 
 \begin{cases} 
      \llbracket e_2 \rrbracket\eta \text{ if } \llbracket e_1 \rrbracket\eta = true \\
      \llbracket e_3 \rrbracket\eta \text{ if } \llbracket e_1 \rrbracket\eta = false \\
      \perp \text{      otherwise} \\
   \end{cases}
  \\
   \llbracket  \texttt{fix} (\lambda f.\lambda x.e) \rrbracket\eta &= fix\llbracket \lambda f.\lambda x.e \rrbracket\eta \
 \text{, where } fix \text{ assigns the least fixed point} \\ 
 &\text{ \ \ \ \ \ \ \ \ \ \ \ \ \ \ \ \ \ \ \ \ \ \ \ to continuous functions} \\
 \end{align*}
 \end{figure}
 
 We will interpret $\texttt{fix}$ expressions using the CPO fixpoint theorem. This theorem states that, for a CPO $P$ and a continuous, order-preserving map $\Phi : P \rightarrow P$, the least fixpoint of $\Phi$ exists and is equal to $\bigvee_i \Phi^n(\perp)$.

In order to use this theorem on the interpretation of $\texttt{fix}(\lambda f. \lambda x.e): \tau \rightarrow \tau$ expressions, 
we must first show that for all types $\tau$, $\llbracket \tau \rrbracket$ is a CPO.
\begin{thm}
For all types $\tau, \ \llbracket \tau \rrbracket$ is a CPO --- that is, $\llbracket \tau \rrbracket$ is an ordered 
set with a bottom element, such that for $\forall C \subset \llbracket \tau \rrbracket$, if  $C$ is a chain, then $C$ has a least 
upper bound in $\llbracket \tau \rrbracket$. \\
\end{thm}
\begin{proof}
by induction on the structure of $\tau$. \\
\emph{Base Cases: }
\begin{itemize}
\item $\tau = \texttt{real}$
\item $\tau = \texttt{bool}$ \\
In this case, $\llbracket \tau \rrbracket$ is set with flat ordering, and is thus a CPO. 
\end{itemize}
\emph{Inductive Cases: }
\begin{itemize}
\item $\tau = \tau_1 \times \tau_2$  \\
Note $\llbracket \tau \rrbracket = \llbracket \tau_1 \rrbracket \times \llbracket \tau_2 \rrbracket$. By inductive hypothesis,
$\llbracket \tau_1 \rrbracket$ and $\llbracket \tau_2 \rrbracket$ are CPO's, and thus have bottom elements $\perp_1$ and $
\perp_2$, respectively. Thus $(\perp_1, \perp_2)$ is a bottom element of $\tau$. 

Let $C \subset \llbracket \tau \rrbracket$ be a
chain, and define
\begin{center}
$C_1 = \{x \in \llbracket \tau_1 \rrbracket : (x,y) \in C\},$ \\
$C_2 = \{ y \in \llbracket \tau_2 \rrbracket: (x,y) \in C\}$. \\ 
\end{center}
Then $C_1 \subset \llbracket \tau_1 \rrbracket$ and $C_2 \subset \llbracket \tau_2 \rrbracket$ are chains, so $\bigvee C_1$
and $\bigvee C_2$ exist. We claim that $(\bigvee C_1, \bigvee C_2)$ is the least upper bound of $C$. 

Let $(c_1, c_2) \in C$. Then $c_1 \in C_1$ and $c_2 \in C_2$, so $c_1 \leq \bigvee C_1$ and $c_2 \leq \bigvee C_2$. Thus, 
$(c_1, c_2) \leq (\bigvee C_1, \bigvee C_2)$, so $(\bigvee C_1, \bigvee C_2)$ is an upper bound of $C$.

Now let $(x,y) \in \llbracket \tau \rrbracket$ be an upper bound of C. Then $\forall c_1 \in C_1, \ c_1 \leq x$, so $x$ is an upper 
bound of $C_1$. Similarly, $y$ is an upper bound of $C_2$. Thus, it must be that $\bigvee C_1 \leq x$ and 
$\bigvee C_2 \leq y$, so $(\bigvee C_1, \bigvee C_2) \leq (x,y)$. Therefore, $(\bigvee C_1, \bigvee C_2)$ is the least upper
bound of $C$.
 
See, then, that by definition $\llbracket \tau \rrbracket$ is a CPO.

\item $\tau = \tau_1 \rightarrow \tau_2$ \\ \\
Note $\llbracket \tau \rrbracket = \{f : \llbracket \tau_1 \rrbracket \rightarrow \llbracket \tau_2 \rrbracket : f \text{ is continuous}\}
$. By inductive hypothesis, $\llbracket \tau_1 \rrbracket$ and $\llbracket \tau_2 \rrbracket$ are CPO's, and thus have bottom
elements $\perp_1$ and $\perp_2$, respectively. Let $\perp: \llbracket \tau_1 \rrbracket \rightarrow \llbracket \tau_2 
\rrbracket$ be a function such that, for $\forall x \in \llbracket \tau_1 \rrbracket, \ \perp(x) = \perp_2$. See, then, that 
$\perp$ is a continuous function that is less than or equal to every function in $\llbracket \tau \rrbracket$ (using pointwise
ordering), so $\perp$ is the bottom element of $\llbracket \tau \rrbracket$.  

Let $C \subset \llbracket \tau \rrbracket$ be a 
chain, and consider that for $\forall x \in \tau_1, \{f(x) : f \in C\}$ is a chain in $\llbracket \tau_2 \rrbracket$. Call this set $F_x$,
and note that $\bigvee F_x$ exists. We define a function $g: \llbracket \tau_1 \rrbracket \rightarrow \llbracket \tau_2 \rrbracket$ 
such that $\forall x \in \llbracket \tau_1 \rrbracket, \ g(x) = \bigvee F_x$. We claim that $g = \bigvee C$. 

Let $f \in C$, and note that for $\forall x \in \llbracket \tau_1 \rrbracket, \ f(x) \leq \bigvee F_x =g(x)$. Then $g$ is an upper bound of $C$.

Now let $\rho \in \llbracket \tau \rrbracket$ be an upper bound of $C$. Then for $\forall x \in \llbracket \tau_1 \rrbracket, \forall f \in C, \ f(x) \leq \rho(x)$, so $\rho(x)$ is an upper bound of $\{f(x) : f \in C\} = F_x$. Thus $\rho(x) \geq F_x = g(x), \ \forall x \in 
\llbracket \tau_1 \rrbracket$, so $g \leq \rho$. Therefore, g is the least upper bound of $C$.

See, then, that by definition $\llbracket \tau \rrbracket$ is a CPO. \\
\end{itemize} 
\end{proof}
%

Further, we must show that all $\texttt{fix}$ expressions interpret to continuous functions. It will suffice to prove type
soundness of our denotational semantics: $\text{fix}$ expressions have type $\tau \rightarrow \tau$, and 
$\llbracket \tau \rightarrow \tau \rrbracket$ is the set of continuous functions from $\llbracket \tau \rrbracket$ to
$\llbracket \tau \rrbracket$. 

\begin{thm} 
For all expressions $e$ and environments $\Gamma$,
\begin{enumerate}
\item For all chains $a_0\leq a_1\leq\dotsb$,
$\tmden{\typing\Gamma e\tau}{\extend\eta x {\bigvee_i a_i}} =
\bigvee_i\tmden{\typing\Gamma e\tau}{\extend\eta x {a_i}}$.
\item For all $\Gamma\vdash e : \tau$ and all $\Gamma$-environments~$\eta$,
$\tmden{\typing\Gamma e\tau}\eta\in\tyden{\tau}$.
\end{enumerate}
\end{thm}
We will see, that in order to prove type soundness for $\lambda x.e$ expressions, we will need to show that 
$\llbracket \lambda x.e \rrbracket$ is a continuous function. Thus, it will be necessary to have $1$.
\begin{proof}
by induction on the structure of $e$. For all cases, let $\{ \eta_i \}^{\infty}_{i=1}$ be a chain, and suppose $ a = \bigvee_i a_i$. 
Note that 2 is trivial in all but the abstraction and $\texttt{fix}$ cases.\\
 \emph{Base Cases: } 
 \begin{itemize}
 % real number constants
 \item $e \in \{ \texttt{0}, \ \texttt{1}, \ \ldots \}$
 % boolean constants
 \item $e \in \{ \texttt{true}, \ \texttt{false} \}$\\ \\
 %
  In both of these cases, $\llbracket \Gamma \vdash e : \tau \rrbracket$ is a constant function from a $\Gamma$-environment
 $\eta$ to an element of $\llbracket \tau \rrbracket$, so clearly $\tmden{\typing\Gamma e\tau}{\extend\eta x {\bigvee_i a_i}} =
\bigvee_i\tmden{\typing\Gamma e\tau}{\extend\eta x {a_i}}$.

% variable
 \item $e = x : \tau$ \\ \\
 We want to show that $\tmden{\typing\Gamma x\tau}{\extend\eta y {\bigvee_i a_i}} =
\bigvee_i\tmden{\typing\Gamma x\tau}{\extend\eta y {a_i}}$.

In the case that $ y \neq x$, note that
\begin{align*}
\tmden{\typing\Gamma x\tau}{\extend\eta y {\bigvee_i a_i}} &= \eta\{y\mapsto \bigvee_i a_i\}(x) \\
&= \eta (x) \\
&= \bigvee_i\tmden{\typing\Gamma x\tau}{\extend\eta y {a_i}}
\end{align*}

 In the case that $y = x$, note that
  \begin{align*}
  \llbracket \Gamma \vdash x : \tau \rrbracket\eta\{y\mapsto\bigvee_i a_i\} &= \eta\{y\mapsto \bigvee_i a_i\}(x) \\
  &= \bigvee_i a_i \\
  &=\bigvee_i \eta\{y \mapsto a_i\}(x)  \\
  &= \bigvee_i\tmden{\typing\Gamma x\tau}{\extend\eta y {a_i}} \\
  \end{align*}
 \end{itemize}
 \emph{Inductive Cases: }
 \begin{itemize}
 % arithmetic operations
 \item $e = e_1 \circ e_2, \ \circ \in \{+, -, *, / \}$ \\ \\
 Then 
 \begin{align*}
 \llbracket \Gamma \vdash e : \tau \rrbracket\eta\{x \mapsto \bigvee_i a_i\} &= \llbracket \Gamma\vdash e_1 : \tau 
 \rrbracket\eta\ \{x \mapsto \bigvee_i a_i\} \circ \llbracket \Gamma \vdash e_2 : \tau \rrbracket\eta\{x \mapsto \bigvee_i a_i\}\\
  \text{(by inductive hypothesis)} &=  \bigvee_i\llbracket \Gamma \vdash e_1: \tau \rrbracket\eta\{x\mapsto a_i\} \circ
  \bigvee_i\llbracket \Gamma \vdash e_2: \tau \rrbracket\eta\{x\mapsto a_i\}  \\
  &= \bigvee_i(\llbracket \Gamma \vdash e_1: \tau \rrbracket\eta\{x\mapsto a_i\} \circ 
  \llbracket \Gamma \vdash e_2: \tau \rrbracket\eta\{x\mapsto a_i\}) \\
  &= \bigvee_i \llbracket \Gamma \vdash e: \tau \rrbracket\eta\{x\mapsto a_i\}
 \end{align*}

  % equality operation
  \item $e = e_1 \circ e_2, \ \circ \in \{=,<,>,\leq,\geq\}$\\ \\
  In the case that 
 \begin{align*}
  \llbracket \Gamma \vdash e_1 : \tau \rrbracket\eta\{x\mapsto  \bigvee_i a_i\} &= \perp \\ 
  \text{or }  \llbracket \Gamma \vdash e_2 : \tau \rrbracket\eta\{x\mapsto  \bigvee_i a_i\} &= \perp,
  \end{align*} 
  note that $\llbracket \Gamma \vdash e : \texttt{bool}\rrbracket\eta\{x\mapsto  \bigvee_i a_i\} = \perp$. 
  By inductive hypothesis, 
  \begin{align*}
  \llbracket \Gamma \vdash e_1 : \tau \rrbracket\{x\mapsto  \bigvee_i a_i\} &= \bigvee_i\llbracket \Gamma \vdash e_1 : \tau \rrbracket\{x\mapsto a_i\} \\
 \text{and } \llbracket \Gamma \vdash e_2 : \tau \rrbracket\{x\mapsto  \bigvee_i a_i\} &= \bigvee_i\llbracket \Gamma \vdash e_2 : \tau \rrbracket\{x\mapsto a_i\}, 
  \end{align*}
  so assuming WLOG that $\llbracket \Gamma \vdash e_1 : \tau \rrbracket\eta\{x\mapsto  \bigvee_i a_i\} = \perp$, 
  it must be that for \\
  $\forall i, \llbracket \Gamma \vdash e_1 : \tau \rrbracket\eta\{x\mapsto a_i\} = \perp$. 
  Thus, $\bigvee_i\llbracket \Gamma \vdash e : \tau \rrbracket\eta\{x\mapsto a_i\} = \perp$, so \\
  $\bigvee_i\llbracket \Gamma \vdash e : \tau \rrbracket\eta\{x\mapsto a_i\} = 
  \llbracket \Gamma \vdash e_1 : \tau \rrbracket\eta\{x\mapsto  \bigvee_i a_i\}$.\\ \\
  %
  In the case that $\llbracket \Gamma \vdash e_1 : \tau \rrbracket\eta\{x\mapsto  \bigvee_i a_i\} 
  \circ \llbracket \Gamma \vdash e_2 : \tau \rrbracket\eta\{x\mapsto  \bigvee_i a_i\}, \ 
  \llbracket \Gamma \vdash e_1:\tau \rrbracket\eta\{x\mapsto  \bigvee_i a_i\} \neq \perp, 
  \llbracket \Gamma \vdash e_2 : \tau \rrbracket\eta\{x\mapsto  \bigvee_i a_i\} \neq \perp$, 
  note that $\llbracket \Gamma \vdash e : \texttt{bool}\rrbracket\eta\{x\mapsto  \bigvee_i a_i\} = true$. 
 By inductive hypothesis, 
  \begin{align*}
  \llbracket \Gamma \vdash e_1 : \tau \rrbracket\{x\mapsto  \bigvee_i a_i\} &= \bigvee_i\llbracket \Gamma \vdash e_1 : \tau \rrbracket\{x\mapsto a_i\} \\
 \text{and } \llbracket \Gamma \vdash e_2 : \tau \rrbracket\{x\mapsto  \bigvee_i a_i\} &= \bigvee_i\llbracket \Gamma \vdash e_2 : \tau \rrbracket\{x\mapsto a_i\}, 
  \end{align*}
  so $\bigvee_i\llbracket \Gamma \vdash e_1 : \tau \rrbracket\{x\mapsto a_i\} 
  \circ \bigvee_i\llbracket \Gamma \vdash e_2 : \tau \rrbracket\{x\mapsto a_i\}$, with neither supremum equal to bottom. Thus there must $\exists n \in \mathbb{N} \text{ s.t. } \forall i \geq n, \llbracket \Gamma \vdash e_1 : \tau \rrbracket\eta\{x\mapsto   a_i\} \circ
  \llbracket \Gamma \vdash e_2 : \tau \rrbracket\eta\{x\mapsto   a_i\}$. Therefore, $\bigvee_i\llbracket \Gamma \vdash e : \tau \rrbracket\eta\{x\mapsto  a_i\}\} = true$, so $\llbracket \Gamma \vdash e : \texttt{bool}\rrbracket\eta\{x\mapsto \bigvee_i a_i\} 
  = \bigvee_i\llbracket \Gamma \vdash e : \tau \rrbracket\eta\{x\mapsto  a_i\}$. \\ \\
  %
 In the case that $\llbracket \Gamma \vdash e_1 : \tau \rrbracket\eta\{x\mapsto  \bigvee_i a_i\} \ \slashed{\circ} \ \llbracket \Gamma \vdash e_2 : \tau\rrbracket\eta\{x\mapsto  \bigvee_i a_i\}, \ \\
 \llbracket \Gamma \vdash e_1 : \tau \rrbracket\eta\{x\mapsto  \bigvee_i a_i\} \neq \perp, 
 \llbracket \Gamma \vdash e_2 : \tau \rrbracket\eta\{x\mapsto  \bigvee_i a_i\} \neq \perp$, by reasoning similar to the previous case see that $\llbracket \Gamma \vdash e : \tau \rrbracket\eta\{x\mapsto  \bigvee_i a_i\} = 
 \bigvee_i\llbracket \Gamma \vdash e : \tau \rrbracket\eta\{x\mapsto  a_i\}$.
  % conditional
 \item $e = \ \ifelse{e_1}{e_2}{e_3} : \tau$ \\ \\
In the case that $\llbracket \Gamma \vdash e_1 : \texttt{bool} \rrbracket\eta\{x\mapsto  \bigvee_i a_i\} = \perp$, note that $\llbracket \Gamma \vdash e : 
\tau \rrbracket\eta\{x\mapsto  \bigvee_i a_i\} =  \perp$. By inductive hypothesis, 
$\llbracket \Gamma \vdash e_1 : \texttt{bool} \rrbracket\eta\{x\mapsto  \bigvee_i a_i\} = 
\bigvee_i\llbracket \Gamma \vdash e_1 : \tau \rrbracket\eta\{x\mapsto a_i\} = \perp$. 
Then it must be that for 
$\forall i, \llbracket \Gamma \vdash e_1 : \tau \rrbracket\eta\{x\mapsto a_i\} = \perp$. 
Thus, $\bigvee_i\llbracket \Gamma \vdash e : \tau \rrbracket\eta\{x\mapsto  a_i\} = \perp$, so 
$\llbracket \Gamma \vdash e : \tau \rrbracket\eta\{x\mapsto  \bigvee_i a_i\} = 
\bigvee_i\llbracket \Gamma \vdash e : \tau \rrbracket\eta\{x\mapsto  \bigvee_i a_i\}$. \\ \\
%
In the case that $\llbracket \Gamma \vdash e_1 : \texttt{bool} \rrbracket\eta\{x\mapsto  \bigvee_i a_i\} = true$, 
note that $\llbracket \Gamma \vdash e : 
\tau \rrbracket\eta\{x\mapsto  \bigvee_i a_i\} =  \llbracket \Gamma \vdash e_2 \rrbracket\eta\{x\mapsto  \bigvee_i a_i\}$. 
By inductive hypothesis, 
$\llbracket \Gamma \vdash e_1 : \texttt{bool} \rrbracket\eta\{x\mapsto  \bigvee_i a_i\} = 
\bigvee_i\llbracket \Gamma \vdash e_1 : \tau \rrbracket\eta\{x\mapsto a_i\} = true$. 
Since $\{a_i\}^{\infty}_{i=1}$ is a chain 
and $\llbracket\texttt{bool}\rrbracket$ is a flat CPO, there must exist $n \in \mathbb{N}$ such that 
$\forall i > n, \llbracket \Gamma \vdash e_1 : \texttt{bool}
\rrbracket\eta\{x\mapsto a_i\} = true$ and 
$\forall i \leq n, \llbracket \Gamma \vdash e_1 : \texttt{bool}\rrbracket\eta\{x\mapsto  a_i\} = \perp$. Thus, it must 
be that $\bigvee_i \llbracket \Gamma \vdash e : \tau \rrbracket\eta\{x\mapsto a_i\} = 
\llbracket \Gamma \vdash e_2 \rrbracket\eta\{x\mapsto  \bigvee_i a_i\} = 
\llbracket \Gamma \vdash e : \tau \rrbracket\eta\{x\mapsto  \bigvee_i a_i\}$. \\ \\
%
In the case that $\llbracket \Gamma \vdash e_1 : \texttt{bool} \rrbracket\eta\{x\mapsto  \bigvee_i a_i\} = false$, by reasoning parallel to the previous case $\bigvee_i \llbracket \Gamma \vdash e : \tau \rrbracket\eta\{x\mapsto a_i\}  = 
\llbracket \Gamma \vdash e : \tau \rrbracket\eta\{x\mapsto  \bigvee_i a_i\}$.
% application
 \item $ e = e_1 \ e_2$ \\ \\
Then
\begin{align*}
\llbracket \Gamma \vdash e : \tau \rrbracket\eta\{x \mapsto \bigvee_i a_i\} &=
\llbracket \Gamma \vdash e_1 : \sigma \rightarrow \tau \rrbracket\eta\{x \mapsto \bigvee_i a_i\}
(\llbracket \Gamma \vdash e_2 : \sigma \rrbracket\eta\{x \mapsto \bigvee_i a_i\}) \\
&= \llbracket \Gamma \vdash \lambda (y:\sigma).e' : \tau \rrbracket\eta\{x \mapsto \bigvee_i a_i\}
(\llbracket \Gamma \vdash e_2 : \sigma \rrbracket\eta\{x \mapsto \bigvee_i a_i\}) \\
\text{(by inductive hypothesis)} &= 
\llbracket \Gamma \vdash \lambda (y:\sigma).e' : \tau \rrbracket\eta\{x \mapsto \bigvee_i a_i\}
(\bigvee_i \llbracket \Gamma \vdash e_2 : \sigma \rrbracket\eta\{x \mapsto a_i\}) \\
&= \llbracket \Gamma. y : \sigma \vdash e' : \tau' \rrbracket\eta\{x \mapsto \bigvee_i a_i\}\{y \mapsto (\bigvee_i \llbracket 
\Gamma \vdash e_2 : \sigma \rrbracket\eta\{x \mapsto a_i\})\} \\
\text{(by inductive hypothesis)}&= \bigvee_i \llbracket \Gamma. y : \sigma \vdash e' : \tau' \rrbracket\eta\{x \mapsto 
a_i\}\{y \mapsto (\llbracket \Gamma \vdash e_2 : \sigma \rrbracket\eta\{x \mapsto a_i\})\} \\
&=\bigvee_i \llbracket \Gamma \vdash e_1: \sigma \rightarrow \tau\rrbracket\eta\{x\mapsto a_i\}
(\llbracket\Gamma\vdash e_2 : \sigma\rrbracket\eta\{x\mapsto a_i\}) \\
&= \bigvee_i \llbracket \Gamma \vdash e:\tau\rrbracket\eta\{ x \mapsto a_i\} \\
\end{align*}
 % abstraction
 \item $e = \lambda (y : \tau) . (e' : \tau')$
 \begin{enumerate}
 \item Then $\llbracket \Gamma \vdash e : \tau \rightarrow \tau' \rrbracket\eta\{x\mapsto  \bigvee_i a_i\}$ i
 s a function $f: \llbracket \tau \rrbracket \rightarrow \llbracket \tau'  \rrbracket$ such that for 
 $\forall d \in \llbracket \tau \rrbracket, \ f(d) = \llbracket \Gamma y : \tau \vdash e' : \tau 
 \rrbracket\eta\{x\mapsto  \bigvee_i a_i\}\{y \mapsto d\}$. We want to show that
 $\tmden{\typing\Gamma e\tau}{\extend\eta x {\bigvee_i a_i}} =
\bigvee_i\tmden{\typing\Gamma e\tau}{\extend\eta x {a_i}}$
 
Let $d \in \llbracket \tau \rrbracket$, and consider that
\begin{align*}
\llbracket \Gamma \vdash e : \tau \rightarrow \tau' \rrbracket\eta\{x\mapsto  \bigvee_i a_i\}(d) 
&= \llbracket \Gamma.y : \tau \vdash e' : \tau'\rrbracket\eta\{x\mapsto  \bigvee_i a_i\}
\{y \mapsto d\} \\
&= \llbracket \Gamma.y : \tau \vdash e' : \tau'\rrbracket(\eta\{y \mapsto d\})\{x\mapsto  \bigvee_i a_i\} \\
\text{(by inductive hypothesis)}&=\bigvee_i\llbracket \Gamma.y:\tau\vdash e':\tau'\rrbracket(\eta\{y\mapsto d\})
\{x\mapsto a_i\} \\
&=\bigvee_i\llbracket \Gamma.y:\tau\vdash e':\tau'\rrbracket(\eta\{x\mapsto a_i\})\{y\mapsto d\} \\
&= \bigvee_i \llbracket \Gamma \vdash \lambda y : \tau.e' : \tau' \rrbracket \eta\{x\mapsto a_i\}(d) \\
&= \bigvee_i \llbracket \Gamma \vdash e : \tau \rightarrow \tau' \rrbracket \eta\{x\mapsto a_i\}(d) \\ 
\end{align*}
%
 \item We want to show that $\llbracket \Gamma \vdash e : \tau \rightarrow \tau' \rrbracket\eta \in \llbracket \tau \rightarrow 
 \tau'\rrbracket$---that is, it is a continuous function from $\llbracket \tau \rrbracket$ to $\llbracket \tau' \rrbracket$. Let 
 $\{a_i\}^{\infty}_{i=1}$ be a chain of elements in $\llbracket \tau \rrbracket$, and see that
 \begin{align*}
 \llbracket \Gamma \vdash e : \tau \rightarrow \tau' \rrbracket\eta(\bigvee_i a_i) &= \llbracket \Gamma.x : \tau 
 \vdash e' : \tau'\rrbracket\eta \{x \mapsto \bigvee_i a_i\} \\
 \text{(by 1)} &= \bigvee_i \llbracket \Gamma.x : \tau \vdash e' : \tau'\rrbracket\eta\{x\mapsto a_i\}\\
 &= \bigvee_i \llbracket \Gamma \vdash e : \tau \rightarrow \tau' \rrbracket\eta(a_i) 
 \end{align*}
 \end{enumerate} 
 % fix operation
 \item $e = \texttt{fix} (\lambda f.\lambda x.e)$ \\
 \begin{enumerate}
\item Let $F_i = \llbracket \Gamma \vdash \lambda f. \lambda x.e\rrbracket\eta\{x\mapsto a_i\}$, and consider that 
 \begin{align*}
 \llbracket \texttt{fix} (\lambda f.\lambda x.e)\rrbracket\eta\{x\mapsto \bigvee_i a_i\} &= 
 fix(\llbracket \lambda f. \lambda x.e \rrbracket\eta\{x\mapsto \bigvee_i a_i\})\\
 \text{(by inductive hypothesis)} &= fix(\bigvee\{F_i\}^{\infty}_{i=1}) \\
 &= \bigvee\{\bigvee\{(F_i)^j \}^{\infty}_{i=1}\perp\}^{\infty}_{j=1} \\
 &= \bigvee\{\bigvee\{(F_i)^j \perp\}^{\infty}_{i=1}\}^{\infty}_{j=1} \\ 
\text{(by \ref{lem1})} &= \bigvee\{\bigvee\{(F_i)^j \perp\}^{\infty}_{j=1}\}^{\infty}_{i=1} \\ 
&= \bigvee\{fix(F_i)\}^{\infty}_{i=1}\\
&= \bigvee_i \llbracket \Gamma \vdash \texttt{fix}(\lambda f \lambda x.e)\rrbracket\eta\{x\mapsto a_i\}
 \end{align*}
 %
 \item Let $F = \llbracket \Gamma \vdash \lambda f. \lambda x.e\rrbracket\eta$. We want to show that 
 $\llbracket\Gamma\vdash\texttt{fix}(\lambda f \lambda x.e) : \tau 
 \rightarrow\tau\rrbracket\eta \in \llbracket \tau \rightarrow \tau \rrbracket$---that is, it is a continuous function from 
 $\llbracket \tau \rrbracket$ to $\llbracket \tau \rrbracket$. Let $\{a_i\}^{\infty}_{i=1}$ be a chain of elements in $\llbracket \tau \rrbracket$,
  and see that 
 \begin{align*}
 \llbracket\Gamma\vdash\texttt{fix}(\lambda f \lambda x.e) : \tau \rightarrow\tau\rrbracket\eta \bigvee_i a_i &= 
 fix(F)\bigvee_i a_i\\
 &=\bigvee\{F^j \perp\}^{\infty}_{j=1}\bigvee_i a_i\\
 &= \bigvee\{F^j \perp(\bigvee_i a_i)\}^{\infty}_{j=1}\\ 
 \text{(by continuity of $F^j \perp$) }&= \bigvee\{\bigvee\{F^j \perp(a_i)\}^{\infty}_{i=1}\}^{\infty}_{j=1}\\ 
 \text{(by \ref{lem2})} &=\bigvee\{\bigvee\{F^j \perp(a_i)\}^{\infty}_{j=1}\}^{\infty}_{i=1}\\
 &= \bigvee\{fix(F)(a_i)\}^{\infty}_{i=1}\\
 &= \bigvee\{\llbracket\Gamma\vdash\texttt{fix}(\lambda f \lambda x.e) : \tau \rightarrow\tau\rrbracket\eta(a_i)\}^{\infty}_{i=1}\\
 \end{align*}
 \end{enumerate}
 \end{itemize}
 Thus, by definition, $\llbracket \Gamma \vdash \texttt{fix}(\lambda f \lambda x.e) \rrbracket\eta$ is a continuous function
 from $\llbracket \tau \rrbracket$ to $\llbracket \tau \rrbracket$.
 \end{proof}
 % Lemmas
 \begin{lemma} $\bigvee\{\bigvee\{(F_i)^j \perp\}^{\infty}_{i=1}\}^{\infty}_{j=1}= \bigvee\{\bigvee\{(F_i)^j \perp\}^{\infty}
 \label{lem1}
 _{j=1}\}^{\infty}_{i=1}$\\
 \end{lemma}
 \begin{proof}
 $\Rightarrow$ We want to show that\\ \\ \centerline{$\bigvee\{\bigvee\{(F_i)^j \perp\}^{\infty}_{i=1}\}^{\infty}
 _{j=1} \geq  \bigvee\{\bigvee\{(F_i)^j \perp\}^{\infty}_{j=1}\}^{\infty}_{i=1}$} \\ \\ First, notice that $\forall i, j \bigvee\{\bigvee
 \{(F_i)^j \perp\}^{\infty}_{i=1}\}^{\infty}_{j=1} \geq (F_i)^j$. \\
 Suppose, now, that 
 \begin{align*}
 &\bigvee\{\bigvee\{(F_i)^j \perp\}^{\infty}_{i=1}\}^{\infty}_{j=1} < \bigvee\{\bigvee\{(F_i)^j \perp\}^{\infty}_{j=1}\}^{\infty}_{i=1}\\ 
 \Rightarrow \exists i \text{ s.t. \ \  \ } &\bigvee\{\bigvee\{(F_i)^j \perp\}^{\infty}_{i=1}\}^{\infty}_{j=1} < \bigvee\{(F_i)^j \perp\}
 ^{\infty}_{j=1}\\
 \Rightarrow \exists i, j \text{ s.t. } &\bigvee\{\bigvee\{(F_i)^j \perp\}^{\infty}_{i=1}\}^{\infty}_{j=1} < (F_i)^j \perp 
 \Rightarrow\Leftarrow\\
 \end{align*}
 Thus, it must be that $\bigvee\{\bigvee\{(F_i)^j \perp\}^{\infty}_{i=1}\}^{\infty}_{j=1} \geq  \bigvee\{\bigvee\{(F_i)^j \perp\}^{\infty}
 _{j=1}\}^{\infty}_{i=1}$ \\ \\
 $\Leftarrow$ By parallel reasoning, $\bigvee\{\bigvee\{(F_i)^j \perp\}^{\infty}_{i=1}\}^{\infty}_{j=1} \leq  \bigvee\{\bigvee\{(F_i)^j 
 \perp\}^{\infty}_{j=1}\}^{\infty}_{i=1}$ \\ \\
 Therefore, $\bigvee\{\bigvee\{(F_i)^j \perp\}^{\infty}_{i=1}\}^{\infty}_{j=1} = \bigvee\{\bigvee\{(F_i)^j \perp\}^{\infty}_{j=1}\}
 ^{\infty}_{i=1}$ 
 \end{proof}
 %
 \begin{lemma}
 \label{lem2}
 $\bigvee\{\bigvee\{F^j \perp(x_i)\}^{\infty}_{i=1}\}^{\infty}_{j=1} = \bigvee\{\bigvee\{F^j \perp(x_i)\}^{\infty}
 _{j=1}\}^{\infty}_{i=1}$\\ 
 \end{lemma}
 \begin{proof} 
 By reasoning similar to \ref{lem1}. 
 \end{proof}
 %
\section{Examples}
We will look at some of the example recurrences Karp offers in this section, to check whether recurrences in our expression
language exhibit the same behavior. First, consider the following recurrence:
\begin{align*}
T(1) &= 0 \\
T(n) &= 1 + T(pn), \ 0 < p < 1
\end{align*} 
For $\forall n > 1, \ \exists k \in \N$ such that $p^kn \leq1, p^{k-1}n > 1$. See, then, that $T(n) = 1 + \left \lfloor{\log_{p}1/n}\right \rfloor$.

We may express this recurrence in our expression language as the follows:
\begin{align*}
\texttt{T} = \texttt{fix}(\lambda f. \lambda x.\ifelse{x<1}{0}{1 + f(p*x)})
 \end{align*}
 We want to show that the denotation of this expression has the same behavior as the recurrence described above. 
 First, let $\texttt{F} =\lambda f. \lambda x.\ifelse{x<1}{0}{1 + f(p*x)}$, and note the following: 
 \begin{align*}
 \llbracket \texttt{T} \rrbracket &= fix(\llbracket \texttt{F} \rrbracket) \\
 \llbracket \texttt{F} \rrbracket &= F : \llbracket \texttt{real} \rightarrow \texttt{real}\rrbracket \rightarrow \llbracket \texttt{real} \rightarrow \texttt{real}\rrbracket \\ 
 &\text{ s.t. } \forall \alpha \in \llbracket \texttt{real}\rrbracket \rightarrow \llbracket \texttt{real} \rrbracket, \\   
 &F(\alpha) = \llbracket\lambda x.\ifelse{x<1}{0}{1 + f(p*x)}\rrbracket\{f \mapsto \alpha\} \\
 \llbracket \texttt{real}\rrbracket &= \R_{\perp}  \\
 \\\text{By the CPO fixpoint theorem, } fix(\llbracket \texttt{F} \rrbracket) &= fix(F) \\
&= \bigvee\{F^n (\perp)\}^{\infty}_{n=1} \\
&= \lim_{n \to \infty}\{F^n (\perp)\}
 \end{align*}
 \begin{lemma}
 $\forall x \in \R_{\perp}, \ F^{n+1}(\perp)(x) =  
 \begin{cases}
 0 \text{ if } x < 1 \\
 1 \text{ if } px < 1 \leq x \\
 \dots \\
 n - 1 \text{ if } \ p^{n}x < 1 \leq p^{n-1} \\
 \perp \text{ otherwise}
 \end{cases}$
 \end{lemma}
\begin{proof}

\emph{Base Cases: } n = 0
\begin{align*}
F^{n+1}(\perp)(x) &= F(\perp) \\
&= 
\begin{cases}
0 \text{ if } x<1 \\
1 + \perp(px) \text{ otherwise}
\end{cases} \\
&=
\begin{cases}
0 \text{ if } x<1 \\
\perp \text{ otherwise}
\end{cases} \\
\end{align*}
Note $p^{n}x = p^0x = x$, so trivially the claim holds. \\
\emph{Inductive Case: } Suppose the claim holds for $\forall k < n$.
\begin{align*}
F^{n+1}(\perp)(x) &=  
\begin{cases}
0 \text{ if } x<1 \\
1 + F^{n}(\perp)(px) \text{ otherwise}
\end{cases} \\
\end{align*}
By inductive hypothesis,
\begin{align*}
F^{n}(\perp)(px) &= F^{(n-1) + 1}(\perp)(px) \\
&=  
 \begin{cases}
 0 \text{ if } px < 1 \\
 1 \text{ if } p^2x < 1 \leq x \\
 \dots \\
 n - 1 \text{ if }  p^{n}x < 1 \leq p^{n-1}x \\
 \perp \text{ otherwise}
 \end{cases}
 \end{align*}
\end{proof}
See, then, that
\begin{align*}
F^{n+1}(\perp)(x) &=  
\begin{cases}
0 \text{ if } x<1 \\
1 \text{ if } px < 1 \\
2 \text{ if } p^2x < 1 \leq x \\
\dots \\
n-1 \text{ if } p^{n}x < 1 \leq p^{n-1}x \\
\perp \text{ otherwise } 
\end{cases}
\end{align*}
Therefore, the claim holds for $\forall n$.

Consider that since $p<1, \ \lim_{n \to \infty}\{(1/p)^n\} = \infty$. Then for $\forall x \in \R, \exists n \in \N$ such that 
\begin{align*}
(1/p)^{n-1} < \ &1/x \leq (1/p)^n \\
x/p^{n-1} < \ &1 \leq x/p^n \\
p^nx < \ &1 \leq p^{n-1}x \\
\end{align*}
Then $\llbracket \texttt{T} \rrbracket = lim_{n \to \infty}\{F^n (\perp)\}$ is defined for $\forall x \in \R$ as
\begin{align*}
\lim_{n \to \infty}\{F^n (\perp)\}(x) = n - 1
\end{align*}
Where $n \in \N$ is the minimum number such that $p^nx<1$. Then $\llbracket \texttt{T} \rrbracket(x) = 1 + \left \lfloor{\log_{p}1/n}\right \rfloor = T(x)$. Therefore, $\llbracket \texttt{T} \rrbracket = T$.
 



