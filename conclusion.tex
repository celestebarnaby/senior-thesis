\chapter{Conclusion}

We have defined three languages which allow us to write and type-check the several types of recurrences
described by Karp. The first allows us to describe deterministic recurrences of the form $T(x) = a(x) + T(m(x))$, where $a$ and $m$ are real-valued functions. The second allows us to describe recurrences of this form, but with an added initial 
condition $T(m) = c$. And the third allows us to describe probabilistic recurrences of the form 
$T(x) = a(x) + T(h(x)(l))(\hat{h}(x)(l))$, where $a$ is a real-valued function, $h(x)$ is a random variable of type
$\Omega_x \rightarrow \R$, and $\hat{h}(x)$ is a random variable of type
 $\prod_{l \in \Omega_x} \Omega_{h(x)(l)}$. Moreover, we have discussed a means of translating between
 the first and second syntaxes. 
 
 \section{Future Work}
 In future work, we hope to tie in the work done in this thesis to the current cost extraction framework developed
 by \cite{N.-Danner:2015aa}. Its current structure is as follows: a source language program $e$ is mapped by an 
 extraction function to a syntactic cost recurrence $E$. This syntactic recurrence is then mapped by a denotational
 semantics to a semantic recurrence $\llbracket E \rrbracket$ which describes an upper bound on the cost of $e$. 
 \begin{align*}
 e\xrightarrow{\text{extr. function}}E\xrightarrow{\text{den. semantics}}\llbracket E \rrbracket
 \end{align*}
 However, as discussed in the introduction, this denotational semantics does not always offer useful bounds on 
 probabilistic recurrences. We aim to expand the cost recurrence syntax to support the syntax for 
 probabilistic recurrences offered in this thesis, as well as to include the denotational semantics for that syntax.
 This will mean that, given a syntactic recurrence $E$, we can interpret it using either denotational semantics;
which semantics we use will be determined by what kind of analysis we wish to make. This will allow us to obtain 
better bounds on the costs of a wider range of algorithms.