\documentclass{westhesis}
\title{A Syntax and Denotational Semantics for Probabilistic Recurrence Relations}

\usepackage{amsthm}
	% Some theorem-like environments, all of which share the same numbering.
	\theoremstyle{plain}
	\newtheorem{thm}{Theorem}
	\newtheorem{lemma}[thm]{Lemma}
	\newtheorem{prop}[thm]{Proposition}

	\theoremstyle{definition}
	\newtheorem{defn}{Definition}

\usepackage{amsmath}
\usepackage{amssymb}
\usepackage{stmaryrd}
\usepackage{enumerate}
\usepackage{wasysym}
\usepackage{slashed}
\let\wasysymLightning\lightning

\newcommand{\R}{\mathbb{R}}
\newcommand{\extR}{\R_{[0,+\infty]}}
\newcommand{\N}{\mathbb{N}}

\newcommand{\Ctxt}       {\mathcal{E}}
\newcommand{\InCtxt} [1] {\Ctxt[#1]}
\newcommand{\ssosredex}        {\rightarrow}
\newcommand{\ctxtreduce}       {\mapsto}
\newcommand{\sstep}     [3] [] {#2 &\ssosredex&  #3 &\textsc{#1}}
\newcommand{\ctxtstep}  [3] [] {#2 &\ctxtreduce& #3 &\textsc{#1}}
\newcommand{\tmden}[2]{\llbracket{#1}\rrbracket{#2}}
\newcommand{\tyden}[1]{\llbracket{#1}\rrbracket}
\newcommand{\typing}[3]{{#1}\vdash{#2}:{#3}}
\newcommand{\extend}[3]{#1\{{#2}\mapsto{#3}\}}
\newcommand{\sfix}{\mathtt{fix}}
\newcommand{\smfix}{\underline{\mathit{fix}}}
\newcommand{\set}[1]{\{#1\}}
\let\bottom\bot
\newcommand{\env}[1]{\{#1\}}
\newcommand{\bind}[2]{{#1}\mapsto{#2}}
\newcommand{\ifelse}[3]{\texttt{if } #1 \texttt{ then } #2 \texttt{ else } #3}

% ND comments.
\usepackage{cprotect}
\usepackage{framed}
\usepackage[normalem]{ulem}
\usepackage{versions}
	\markversion{ndcomments}
	\markversion{master}
\setlength{\marginparwidth}{1.25in}
\reversemarginpar
\newcommand{\ndComment}[2][\relax]{\ifx#1\relax\else\uline{#1}\fi\marginpar{\renewcommand{\baselinestretch}{1.0}\tiny ND \ifx#1\relax\else(#1)\fi: {#2}}}
\newenvironment{ndBlockComment}{\begin{oframed}\noindent ND:\\}{\end{oframed}}
\newcommand{\ndTypo}[1]{\ndComment[#1]{$\longrightarrow$}}
\newcommand{\ndDel}[1]{\sout{#1}\ndComment{$\longrightarrow$}}

\begin{document}
\begin{abstract}
abstract goes here:)
\end{abstract}

%Chapters
\chapter{Introduction}
Analyzing the complexity and performance of an algorithm is an important problem in computer science. We want our programs to run as quickly and as efficiently as possible; thus, it is crucial that we understand how much time and space it takes to execute a program, with respect to its input size. Prior work has been done on the development of tools
for automating the analysis of program complexity, which would allow us to make these judgements more quickly and accurately. 

However, such work has so far only offered a method for obtaining an upper bound on the worst-case runtime of an algorithm---which in certain cases does not reflect the actual runtime of an algorithm. In particular, this method does not
provide useful bounds on algorithms with stochastic processes, where the runtime may be quite slow in the worst case but
is, on average, much more efficient. Developing methods for extracting a tighter upper-bound on the cost of probabilistic algorithms would allow us to analyze a wider array of programs.

\section{Automated Complexity Analysis}

The cost of an algorithm is a function from the size of the input of the algorithm to the number of steps necessary to evaluate 
that input. As an example, consider the following $\texttt{add}$ function, which adds an integer $x$ to every element 
in a list $ys$ of integers: \\
\begin{verbatim}
let rec add x ys =
	match ys with
	| [] -> []
	| (y::ys') -> (y+x):: add x ys'
\end{verbatim}
In a traditional cost analysis of this function, we assume that integer addition requires some constant number of steps $c$. The cost of this function is dependent upon the length of the list---that is, the more integers in the list, the more adding we need to
do, and the longer the function will take to run. We may write a recurrence $T$ describing its cost, based on the length
$n$ of the input list:
\begin{align*}
T(0) &= 0 \\
T(n) &= c + T(n-1)
\end{align*}
To determine the cost of $\texttt{add}$, we need only solve this recurrence. We find that $T(n) = cn$, meaning this function
is linear in the length of the list. 

However, this approach has several problems. To start, this is an ad hoc way to analyze cost. We cannot generalize
this result for use in other functions; thus, all traditional cost analyses involve writing recurrences by hand. Further, there is 
no formal connection between the algorithm and the recurrence, meaning we must also manually verify that the recurrence
we write down accurately describes the function we are analyzing. The combination of these issues creates a lot of potential
for human error.

Another problem is that we assume that the work done to each element of the list requires a constant number of steps. 
This assumption is fine in the case of $\texttt{add}$, where all we are doing is adding integers; however, if we consider
the more general $\texttt{map}$ function, we run into some complications. 
\begin{verbatim}
let rec map f ys =
	match ys with
	| [] -> []
	| (y::ys') -> f(y):: map f ys'
\end{verbatim}
Suppose $ys$ is a list of lists of integers, and $f$ is a function that performs selection sort on a single list. Then $f(y)$ does
not take a constant amount of work; rather, its cost is quadratic in the length of $y$. In order to write a correct recurrence
for $\texttt{map}$, we need some information about the cost of $f(y)$, where the size of y is variable. This issue
becomes even thornier if the function $f$ is a composition of functions $g \circ h$. In this case, we need information about the cost of $g(h(y))$, meaning we in turn need information about the \emph{size} of $h(y)$--- the list output by applying a list $y$ to $h$. 

\cite{N.-Danner:2015aa} aim to resolve these issues by developing a formal method for extracting
recurrences that bound the complexity of higher-order programs. This work offers an extraction function that maps programs written in a language with structural list recursion--- referred to as a source language--- to a complexity. A complexity is a pair consisting of a cost--- the steps required to evaluate the program---and a potential--- the size of the 
output of this program. Further, it provides a bounding theorem which guarantees that this extracted complexity is a 
bound on the actual cost of evaluating the source language program. 

The recurrences produced by the extraction function are all syntactic: they are cost-annotated source language
programs, rather than what we would traditionally think of as a recurrence for an algorithm. Danner et al. define a 
denotational semantics for this complexity language, as a formal means of constructing a mathematical object to
describe the meaning of programs in this language. The denotations of these cost-annotated programs resemble
more familiar recurrences.

\section{Probabilistic Recurrence Relations}

Some algorithms have probabilistic elements which affect the way we analyze them. A prime example of this is randomized 
quicksort---the standard quicksort algorithm wherein the pivot is chosen to be a random element in the list. In the worst case,
this algorithm is $O(n^2)$, making it no more efficient than other sorting algorithms such as bubble sort or insertion sort. 
However, in the average case it is $O(n\log n)$, making it one of the most efficient sorting algorithms. Hence, analyses of probabilistic algorithms must consider analyses of runtime other than worst case. One approach is to bound the upper
tails of the probability distribution of the algorithm. This allows us to ensure, for instance, that the probability of 
quicksort taking much longer than $O(n \log n)$ is very small. 

While there are proofs that the average-case runtime of randomized quicksort is $O(n\log n)$, they commonly use
imprecise, ad hoc arguments: that is, they cannot be applied to other algorithms with similar randomized elements. 
\cite{Karp} offers a method for analyzing stochastic divide-and-conquer processes by describing them as recurrence relations of the form
\begin{align*}
T(x) = a(x) + T(h_1(x)) + \dots + T(h_n(x))
\end{align*}
where $x$ is a non-negative real number describing the size of the input,\\ $h_1(x) \dots h_n(x)$ are random variables 
describing the sizes of the subproblems (e.g. in the case of quicksort $h_1(x)$ and $h_2(x)$ describe the sizes of the two 
sublists), $a$ is a function describing the work necessary to generate these subproblems, and $T(x)$ is a random variable 
describing the total running time of the algorithm. Karp then offers several methods for obtaining tight bounds on the upper tails of the probability distribution of $T(x)$. 

Note, however, that the recurrence relation $T(x) = a(x) + T(h(x))$ has some clear flaws that 
emerge when we attempt to assign types to its expressions. Since $T(x)$ is supposed to be random
variable over inputs of size $x$, $T(x)$ is a function of type 
$\Omega_x \rightarrow \R$, where $\Omega_x$ is the sample space of inputs of size $x$. Then $T$ is a 
function of type $\R \rightarrow (\Omega_x \rightarrow \R)$. However, we see that $T$ recursively takes 
$h(x)$ as an argument, even though $h(x)$ is a described as random variable with type $\Omega_x \rightarrow \R$, 
rather than $\R$. This type-checking error is likely a result of the informality of the recurrence description, 
and we are meant to understand that $T$ takes not the 
random variable $h(x)$ as an argument, but the result of plugging an input $l \in \Omega_x$ into $h(x)$. Using this 
interpretation, we then have $T(x)(l) = a(x) + T(h(x)(l))$. But this is also problematic, since $T(x)(l)$, the result 
of plugging an input $l$ into $T(x)$, is a real number, while $T(h(x)(l))$ is a random variable. 

Taking this a step further, we could interpret this to mean $T(x)(l) = a(x) + T(h(x)(l))(l^\prime)$, where $l^\prime \in 
\Omega_{h(x)(l)}$ is the derived subproblem. But acquiring such an $l^\prime$ requires us to define an 
additional function $\hat{h}(x)$ that takes an input $l$ and returns the derived subproblem $l'$. 
Even worse, this function is of type $\Omega_x \rightarrow \Omega_{h(x)(l)}$ --- that is, its type is dependent 
upon the random variable $h(x)$. Thus, we find that Karp's seemingly straightforward recurrence relation quickly becomes 
increasingly complicated when we attempt to type-check it. Other researchers (\cite{Tassarotti:2017aa}) have corroborated 
these observations. 


\section{Contributions of This Thesis}

Karp offers the equation $T'(x) = a(x) + T'(m(x))$ as a deterministic counterpart of a recurrence relation $T(x) = a(x) + 
T(h(x))$, where, for all $x$, $m(x)$ is equal to the upper bound of $h(x)$. This thesis defines syntaxes that allow
us to write and type-check both probabilistic recurrences of the form $T'$, and deterministic recurrences
of the form $T$. These syntaxes comprise a simple expression language including natural number constants, identifiers, basic arithmetic operations, booleans, and application. Further, these syntaxes must provide
a way to express recursive functions, since this work centers around expressing recurrence relations. 
A natural approach to this is to model our languages off of PCF, including a fixed-point combinator $\texttt{fix}$.  

 We also define a 
denotational semantics for these expression languages, wherein all types interpret to a complete partial order---that is, an ordered
set which has a bottom element and where every chain has a least upper bound. For instance, the type {\tt bool}, assigned
to boolean constants, interprets to $\{\text{true, false}\}_{\perp}$---the flat partial order of the set  $\{true, 
\ false\}$ where
both elements are comparable to only a bottom element $\perp$. In addition each arrow type $\sigma 
\rightarrow \tau$ interprets to the set of all continuous functions of type $\llbracket \sigma \rrbracket \rightarrow 
\llbracket \tau \rrbracket$. 

 While imposing these restrictions limits the programs we can interpret to some extent, it ensures 
that every type interprets to a complete partial order (CPO), and every arrow type expression interprets to a continuous 
function. These two criteria allow us to use the CPO fixed-point theorem to define a function $fix$ which assigns the least 
fixed point to continuous functions with type $\tau \rightarrow \tau$. This in turn allows us to interpret recurrence relations 
by defining a $\texttt{fix}$ operation and letting all $\texttt{fix}(\lambda f.\lambda x.e)$ expressions interpret to 
$fix(\llbracket\lambda f.\lambda x.e\rrbracket$. We show that our denotation of a recurrence is equal to the solution Karp 
offers, thus verifying that our recurrence relations are equivalent to Karp's.
 
 However, as we will see in chapters 2 and 5,  complications arise which  prohibit us from taking
 such an approach in all but one specific case (described in chapter 3). 
 For our syntaxes for general deterministic and probabilistic recurrences, we instead introduce a $\texttt{rec}$ 
 operator, allowing us to interpret a subset of recursive functions---including recurrences of the form
 described by Karp.
 
 The last three paragraphs of Chapter 1 need some work.  This should be a sort of narrative table of contents of the remaining chapters, and should follow the order of those chapters.  Thus you might start off by saying that in Chapter 2 you will define a language and denotational semantics that formalizes deterministic recurrences.  A natural approach would be to use a general fixpoint construction and CPO semantics, but this turns out to result in unexpected problems.  Instead, you define a more restricted language that has a more restricted recursion construct, which you interpret using infinite sums.  However, the recurrences that Karp proves theorems about do not have the same form as the ones that come up in his examples of applications.  Thus in Chapter 3 you formulate a language for those recurrences, and there you are able to use a general fixpoint construction and CPO semantics.  You then investigate the difficulty of reconciling these two seemingly different forms of recursive definitions in Chapter 4.  Of course, the main point of Karp's paper is probabilistic recurrences, and so you turn to a language and semantics for them in Chapter 5.  This turns out to be quite complex, as the typing of these recurrences itself is already problematic.  The solution is to use dependent types.  However, it is quite difficult to combine dependent types with CPOs, so you focus on a semantics that corresponds to the infinite sum interpretation for deterministic recurrences.  You conclude in Chapter 6 with a discussion of how your work fits into Danner et al.'s recurrence extraction framework and further work.
 


\chapter{Deterministic Recurrences as Fixed Points}
\cite[\S2]{Karp} offers several example applications of his cost-bounding theorems. However, the recurrences
he offers here do not follow the definition he set forth in the prior section. For example, a recurrence for a randomized list 
ranking algorithms is written as
\begin{align*}
T(1) &= 1 \\
T(n) &= 1 + T(3/4n)
\end{align*}
The presence of an initial condition here is confusing: Karp has made no mention of initial conditions up to this point, and 
there is no clear way to write the above equations as one equation of the form $T(x) = a(x) + T(m(x))$. Simply put, 
this recurrence does not seem to be within the domain of recurrences Karp describes in section 1. Thus, we have no way
of writing this recurrence using the language defined in the previous chapter. 

In this chapter, we define a language which allows us to write and type-check the recurrence relations offered in section 2 of Karp's paper. The syntax of this language is mostly analogous to Programming Computable Functions (PCF) ---a typed functional language with a fixed-point combinator--- with one exception: instead of a 
$\texttt{nat}$ type, we have a $\texttt{real}$ type, as this is the domain of the functions Karp discusses in this section. 
We will circumvent the issues we had with this approach in the previous chapter by writing these recurrences
using conditional statements $\ifelse{e_1}{e_2}{e_3}$, with $e_2$ corresponding to the initial condition and $e_3$
to the general condition. We will see that this approach will allow us to have strict evaluation of base operations, while
still obtaining correct solutions to recurrences.
\[
\begin{array}{rcl}
\tau &::=& \texttt{real} \mid \texttt{bool} \mid \tau \times \tau \mid \tau \rightarrow \tau \\
e &::=& x  \mid \texttt{0} \mid \texttt{1} \mid \texttt{2} \mid \dotsc \mid \lambda x.e \mid e \ e \mid e + e \mid e - e \mid  e  *  e 
\mid e / e \mid \texttt{true} \mid \texttt{false} \mid \\
  && e  =  e \mid e < e \mid e > e \mid e \leq e \mid e \geq e \mid 
     \ifelse{e}{e}{e} \mid (e_1, e_2) \mid \\
     && \texttt{fix} (\lambda f.\lambda x.e) 
\end{array}
\]

\begin{figure}
\[
\begin{array}{lr}
\dfrac{}{\Gamma \vdash \texttt{0}: \texttt{real}}, \quad \dfrac{}{\Gamma \vdash \texttt{1}: \texttt{real}}, \ldots \\
\dfrac{}{\Gamma \vdash \texttt{true} : \texttt{bool}}, \ \ \dfrac{}{\Gamma \vdash \texttt{false} : \texttt{bool}} \\ \\
\dfrac{x : \tau \in \Gamma}{\Gamma \vdash x : \tau } \\ \\
\dfrac{\Gamma, x : \sigma \vdash e : \tau}{\Gamma \vdash \lambda (x : \sigma).e : \sigma \rightarrow \tau } \\ \\ 
\dfrac{\Gamma \vdash e_1: \sigma \rightarrow \tau \ \ \ \Gamma \vdash e_2 : \sigma}{\Gamma \vdash e_1 \ e_2 : \tau} \\ \\ 
\dfrac{\Gamma \vdash e_1 : \texttt{real} \ \ \ \Gamma \vdash e_2 : \texttt{real}}{\Gamma \vdash e_1 \circ e_2 : \texttt{real}}
, \circ \in \{+,-,*,/\} \\ \\ 
\dfrac{\Gamma \vdash e_1 : \texttt{real} \ \ \ \Gamma \vdash e_2 : \texttt{real}}{\Gamma \vdash e_1 \circ e_2 : \texttt{bool}}
, \circ \in \{=, <, >, \geq, \leq\} \\ \\ 
\dfrac{\Gamma \vdash e_1 : \texttt{bool} \ \ \ \Gamma \vdash e_2 : \tau \ \ \ \Gamma \vdash e_3 : \tau}
{\Gamma \vdash \ifelse{e_1}{e_2}{e_3} : \tau} \\ \\ 
\dfrac{\Gamma \vdash e_1 : \tau_1 \ \ \ \Gamma \vdash e_2: \tau_2}{\Gamma \vdash (e_1, e_2) : \tau_1 \times \tau_2}
\dfrac{\Gamma \vdash \lambda f. \lambda x.e : (\tau \rightarrow \sigma) \rightarrow (\tau \rightarrow \sigma)}
{\Gamma \vdash \texttt{fix}(\lambda f. \lambda x.e) : \tau \rightarrow \sigma} 
\end{array}
\]
\caption{Typing for the language.}
\label{fig:typing}
\end{figure}

\begin{figure}
\[
\begin{array}{lr}
e = e \\ \\
\texttt{0} + \texttt{0} = \texttt{0}, \ \texttt{0} + \texttt{1} = 1, \ldots, \ \texttt{3} + \texttt{5} = \texttt{8}, \ldots  \\
\text{equivalent rules for } -, \ * \text{, and } /\text{ operations.}
\\ \\
(n =n) = \texttt{true} \\ (n=m) = \texttt{false}\text{, provided } n, \ m \text{ distinct numerals.}\\ \\ 
\ifelse{\texttt{ true }}{e_1}{e_2} = e_1 \\
\ifelse{\texttt{ false }}{e_1}{e_2} = e_2 \\ \\ 
\lambda x.e = \lambda y.[x \mapsto y]e, \text{provided } y \text{ not free in } e. \\ \\ \\
\lambda x.e_1 \ e_2 = [x \mapsto e_2]e_1 \\
\texttt{fix}(\lambda f. \lambda x.e) = [f \mapsto \texttt{fix}(\lambda f.\lambda x.e)](\lambda x.e)
\end{array}
\]
\caption{Equational Semantics for the language.}
\label{fig:typing}
\end{figure}

The equational semantics of our language is given in Figure 2, and the denotational semantics is given in figures 3 and 4.
This semantics is the standard one for PCF, where each type interprets to a complete
partially ordered set (CPO). A CPO is a set $S$ with two defining properties. But before we state these properties,
we first offer some definitions related to ordered sets. An element $\perp \in S$ is called the bottom element if for all
$s \in S, \ \perp \leq s$. A subset $C \subset S$ is called a chain if $C = \{c_1, c_2, c_3, \dots\}$, where 
$c_1 \leq c_2 \leq c_3 \leq \dots$. An upper bound on $C$ is an element $a \in S$ such that for all $i, \ c_i \leq a$. 
The least upper bound $\bigvee C$ is an upper bound of C such that, for any upper bound $a$ of $C, \ \bigvee C \leq a$.

With these definitions in mind, we say S is a CPO if
\begin{align*}
&\text{1) S has a bottom element }\perp.\\
&\text{2) For any chain }C \subset S, \ \bigvee C \in S. \\
\end{align*}

\begin{figure}
 \begin{align*}
\llbracket \texttt{real} \rrbracket &= \R_{\perp} \\
 \llbracket \texttt{bool} \rrbracket &= {\{true, false\}}_{\perp} \\
\text{For any set } S, \ S_{\perp} &= S \cup \{\perp\}, \text{ where }\forall x, y\in S, \ x \leq y \iff x = \perp. \\
 \text{This is called the}&\text{ flat ordering of } S.\\
 \llbracket \tau \times \sigma \rrbracket &= \llbracket \tau \rrbracket \times \llbracket \sigma \rrbracket  \\
 \text{For all } (a, b), \ (c, d) \in \llbracket \tau \times \sigma \rrbracket &\text{, let }  (a,b) \leq (c,d) \text{ if and only if } a<c \text{ and } b<d. \\ 
 \llbracket \tau \rightarrow \sigma \rrbracket &= \{f: \llbracket \tau \rrbracket \rightarrow \llbracket \sigma \rrbracket \ : 
 \ f \text{ is continuous}\} \\
 \text{ For all } f, \ g \in \llbracket \tau \rightarrow \sigma \rrbracket &\text{, let } f \leq g \text{ if and only if } \forall x \in 
 \llbracket \tau \rrbracket, \ f(x) \leq g(x) 
 \end{align*}
 \caption{Denotational semantics for types.}
 \end{figure}
 \begin{figure}
 \begin{align*}
 \llbracket \texttt{0} \rrbracket\eta &= 0, \  \llbracket \texttt{1} \rrbracket\eta = 1, \ \ldots \\
  \llbracket x : \tau \rrbracket\eta &= \eta(x) \\
  \llbracket \lambda (x : \tau) . (e : \sigma) \rrbracket\eta &= f : \llbracket \tau \rrbracket \rightarrow \llbracket \sigma \rrbracket
\text{ s.t. } \forall d \in \llbracket \tau \rrbracket, f(d) = \llbracket e \rrbracket\eta\{ x \mapsto d \} \\
 \llbracket e_1 \ e_2 \rrbracket \eta &= \llbracket e_1 \rrbracket\eta ( \llbracket e_2 \rrbracket\eta ) \\
 \llbracket e_1 + e_2 \rrbracket\eta &= \llbracket e_1 \rrbracket\eta + \llbracket e_2 \rrbracket\eta \\
 \llbracket e_1 - e_2 \rrbracket\eta &= \llbracket e_1 \rrbracket\eta - \llbracket e_2 \rrbracket\eta \\
 \llbracket e_1 * e_2 \rrbracket\eta &= \llbracket e_1 \rrbracket\eta * \llbracket e_2 \rrbracket\eta \\
  \llbracket e_1 / e_2 \rrbracket\eta &=
  \begin{cases}
  \perp \text{ if }  \llbracket e_2 \rrbracket\eta = 0 \\
   \llbracket e_1 \rrbracket\eta / \llbracket e_2 \rrbracket\eta \text{ otherwise}
   \end{cases} \\
  \llbracket \texttt{true} \rrbracket\eta &= true, \ \llbracket \texttt{false} \rrbracket\eta = false \\
 \llbracket e_1 = e_2 \rrbracket\eta &= 
 \begin{cases} 
      true \text{ if } (\llbracket e_1 \rrbracket\eta = \llbracket e_2 \rrbracket\eta \neq \perp) \\
      false \text{  if } (\llbracket e_1 \rrbracket\eta \neq \llbracket e_2\rrbracket\eta, \llbracket e_1 \rrbracket\eta \neq \perp, \llbracket e_2 \rrbracket\eta \neq \perp)\\
      \perp \text{ otherwise}
   \end{cases} \\
\text{Similar rules for $<, \ \leq, >, \geq$} \\
  \llbracket \ifelse{e_1}{e_2}{e_3} \rrbracket \eta &= 
 \begin{cases} 
      \llbracket e_2 \rrbracket\eta \text{ if } \llbracket e_1 \rrbracket\eta = true \\
      \llbracket e_3 \rrbracket\eta \text{ if } \llbracket e_1 \rrbracket\eta = false \\
      \perp \text{      otherwise} \\
   \end{cases}
  \\
  \llbracket (e_1,e_2) \rrbracket\eta &= (\llbracket e_1 \rrbracket\eta,\llbracket e_2 \rrbracket\eta) \\
   \llbracket  \texttt{fix} (\lambda f.\lambda x.e) \rrbracket\eta &= fix\llbracket \lambda f.\lambda x.e \rrbracket\eta \\
 \text{where } fix &\text{ assigns the least fixed point to continuous functions} 
 \end{align*}
 \caption{Denotational semantics for expressions.}
 \end{figure}
 
 We will interpret $\texttt{fix}$ expressions using the CPO fixed-point theorem. This theorem states that, for a CPO $P$ and a continuous, order-preserving map $\Phi : P \rightarrow P$, the least fixed-point of $\Phi$ exists and is equal to $\bigvee_i \Phi^n(\perp)$.

In order to use this theorem on the interpretation of $\texttt{fix}(\lambda f. \lambda x.e): \tau \rightarrow \tau$ expressions, 
we must first show that for all types $\tau$, $\llbracket \tau \rrbracket$ is a CPO.
\begin{thm}
For all types $\tau, \ \llbracket \tau \rrbracket$ is a CPO --- that is, $\llbracket \tau \rrbracket$ is an ordered 
set with a bottom element, such that for all $ C \subset \llbracket \tau \rrbracket$, if  $C$ is a chain, then $C$ has a least 
upper bound in $\llbracket \tau \rrbracket$. 
\end{thm}
\begin{proof}
by induction on the structure of $\tau$. \\
\emph{Base Cases: }
\begin{itemize}
\item $\tau = \texttt{real}$
\item $\tau = \texttt{bool}$ \\
In this case, $\llbracket \tau \rrbracket$ is a set with flat ordering, and is thus a CPO. 
\end{itemize}
\emph{Inductive Cases: }
\begin{itemize}
\item $\tau = \tau_1 \times \tau_2$  \\
Note $\llbracket \tau \rrbracket = \llbracket \tau_1 \rrbracket \times \llbracket \tau_2 \rrbracket$. By inductive hypothesis,
$\llbracket \tau_1 \rrbracket$ and $\llbracket \tau_2 \rrbracket$ are CPO's, and thus have bottom elements $\perp_1$ and $
\perp_2$, respectively. Thus $(\perp_1, \perp_2)$ is a bottom element of $\tau$. 

Let $C \subset \llbracket \tau \rrbracket$ be a
chain, and define
\begin{center}
$C_1 = \{x \in \llbracket \tau_1 \rrbracket \mid \exists y \in \llbracket \tau_2 \rrbracket \text{ s.t. } (x,y) \in C\},$ \\
$C_2 = \{ y \in \llbracket \tau_2 \rrbracket\mid \exists x \in \llbracket \tau_1 \rrbracket \text{ s.t. } (x,y) \in C\}$. \\ 
\end{center}
Then $C_1 \subset \llbracket \tau_1 \rrbracket$ and $C_2 \subset \llbracket \tau_2 \rrbracket$ are chains, so $\bigvee C_1$
and $\bigvee C_2$ exist. We claim that $(\bigvee C_1, \bigvee C_2)$ is the least upper bound of $C$. 

Let $(c_1, c_2) \in C$. Then $c_1 \in C_1$ and $c_2 \in C_2$, so $c_1 \leq \bigvee C_1$ and $c_2 \leq \bigvee C_2$. Thus, 
$(c_1, c_2) \leq (\bigvee C_1, \bigvee C_2)$, so $(\bigvee C_1, \bigvee C_2)$ is an upper bound of $C$.

Now let $(x,y) \in \llbracket \tau \rrbracket$ be an upper bound of C. Then $\forall c_1 \in C_1, \ c_1 \leq x$, so $x$ is an upper 
bound of $C_1$. Similarly, $y$ is an upper bound of $C_2$. Thus, it must be that $\bigvee C_1 \leq x$ and 
$\bigvee C_2 \leq y$, so $(\bigvee C_1, \bigvee C_2) \leq (x,y)$. Therefore, $(\bigvee C_1, \bigvee C_2)$ is the least upper
bound of $C$.

\item $\tau = \tau_1 \rightarrow \tau_2$ \\ \\
Note $\llbracket \tau \rrbracket = \{f \mid \llbracket \tau_1 \rrbracket \rightarrow \llbracket \tau_2 \rrbracket : f \text{ is continuous}\}
$. By inductive hypothesis, $\llbracket \tau_1 \rrbracket$ and $\llbracket \tau_2 \rrbracket$ are CPO's, and thus have bottom
elements $\perp_1$ and $\perp_2$, respectively. Let $\perp: \llbracket \tau_1 \rrbracket \rightarrow \llbracket \tau_2 
\rrbracket$ be the function such that, for all $ x \in \llbracket \tau_1 \rrbracket, \ \perp(x) = \perp_2$. See, then, that 
$\perp$ is a continuous function that is less than or equal to every function in $\llbracket \tau \rrbracket$ (using pointwise
ordering), so $\perp$ is the bottom element of $\llbracket \tau \rrbracket$.  

Let $C \subset \llbracket \tau \rrbracket$ be a 
chain, and consider that for all $ x \in \tau_1, \{f(x) \mid f \in C\}$ is a chain in $\llbracket \tau_2 \rrbracket$. Call this set $F_x$,
and note that $\bigvee F_x$ exists. We define a function $g: \llbracket \tau_1 \rrbracket \rightarrow \llbracket \tau_2 \rrbracket$ 
such that $\forall x \in \llbracket \tau_1 \rrbracket, \ g(x) = \bigvee F_x$. We claim that $g = \bigvee C$. 

Let $f \in C$, and note that for all $ x \in \llbracket \tau_1 \rrbracket, \ f(x) \leq \bigvee F_x =g(x)$. Then $g$ is an upper bound of $C$.

Now let $\rho \in \llbracket \tau \rrbracket$ be an upper bound of $C$. Then for all $ x \in \llbracket \tau_1 \rrbracket, f \in C, \ f(x) \leq \rho(x)$, so $\rho(x)$ is an upper bound of $\{f(x) : f \in C\} = F_x$. Thus $\rho(x) \geq F_x = g(x), \ \forall x \in 
\llbracket \tau_1 \rrbracket$, so $g \leq \rho$. Therefore, g is the least upper bound of $C$.
\end{itemize} 
\end{proof}
%

Further, we must show that the argument $\lambda f\lambda x.e$ of a $\texttt{fix}$ expression interprets to a continuous function. Note that a function $f: P \rightarrow Q$ (where $P$ and $Q$ are CPOs) is defined to be continuous if, for every
chain $C$ in $P, \ f(\bigvee C) = \bigvee f(C)$ It will suffice to prove type
soundness of our denotational semantics: $\lambda f\lambda x.e$ expressions have type $(\tau \rightarrow \sigma) \rightarrow (\tau \rightarrow \sigma)$, and 
$\llbracket (\tau \rightarrow \sigma) \rightarrow (\tau \rightarrow \sigma) \rrbracket$ is the set of continuous functions from $\llbracket \tau \rightarrow \sigma \rrbracket$ to
$\llbracket \tau \rightarrow \sigma \rrbracket$. 

We will see, that in order to prove type soundness for $\lambda x.e$ expressions, we will need to show that 
$\llbracket \lambda x.e \rrbracket$ is a continuous function. Thus, the statement of type soundness has some additional
complexities.

\begin{thm} 
For all expressions $e$ and environments $\Gamma$,
\begin{enumerate}
\item For all chains $a_0\leq a_1\leq\dotsb$,
\begin{align*}
\tmden{\typing\Gamma e\tau}{\extend\eta x {\bigvee_i a_i}} =
\bigvee_i\tmden{\typing\Gamma e\tau}{\extend\eta x {a_i}}.
\end{align*}
\item For all $\Gamma\vdash e : \tau$ and all $\Gamma$-environments~$\eta$,
$\tmden{\typing\Gamma e\tau}\eta\in\tyden{\tau}$.
\end{enumerate}
\end{thm}

\begin{proof}
by induction on the structure of $e$. For all cases, let $\{ a_i \}^{\infty}_{i=1}$ be a chain, and suppose $ a = \bigvee_i a_i$. 
Note that 2 is trivial in all but the abstraction and $\texttt{fix}$ cases.\\
 \emph{Base Cases: } 
 \begin{itemize}
 % real number constants
 \item $e \in \{ \texttt{0}, \ \texttt{1}, \ \ldots \}$
 % boolean constants
 \item $e \in \{ \texttt{true}, \ \texttt{false} \}$\\ 
 %
  In both of these cases, $\llbracket \Gamma \vdash e : \tau \rrbracket$ is a constant function from a $\Gamma$-environment
 $\eta$ to an element of $\llbracket \tau \rrbracket$, so clearly $\tmden{\typing\Gamma e\tau}{\extend\eta x {\bigvee_i a_i}} =
\bigvee_i\tmden{\typing\Gamma e\tau}{\extend\eta x {a_i}}$.

% variable
 \item $e = x : \tau$ \\ 
 We want to show that $\tmden{\typing\Gamma x\tau}{\extend\eta y {\bigvee_i a_i}} =
\bigvee_i\tmden{\typing\Gamma x\tau}{\extend\eta y {a_i}}$.

In the case that $ y \neq x$, note that
\begin{align*}
\tmden{\typing\Gamma x\tau}{\extend\eta y {\bigvee_i a_i}} &= \eta\{y\mapsto \bigvee_i a_i\}(x) \\
&= \eta (x) \\
&= \bigvee_i\tmden{\typing\Gamma x\tau}{\extend\eta y {a_i}}
\end{align*}

 In the case that $y = x$, note that
  \begin{align*}
  \llbracket \Gamma \vdash x : \tau \rrbracket\eta\{y\mapsto\bigvee_i a_i\} &= \eta\{y\mapsto \bigvee_i a_i\}(x) \\
  &= \bigvee_i a_i \\
  &=\bigvee_i \eta\{y \mapsto a_i\}(x)  \\
  &= \bigvee_i\tmden{\typing\Gamma x\tau}{\extend\eta y {a_i}}
  \end{align*}
 \end{itemize}
 \emph{Inductive Cases: }
 \begin{itemize}
 % arithmetic operations
 \item $e = e_1 \circ e_2, \ \circ \in \{+, -, *, / \}$ \\
 We assume that $\circ$ is continuous in each argument. Then 
 \begin{align*}
 \llbracket \Gamma \vdash e : \tau \rrbracket\eta\{x \mapsto \bigvee_i a_i\} &= \llbracket \Gamma\vdash e_1 : \tau 
 \rrbracket\eta\ \{x \mapsto \bigvee_i a_i\} \circ \llbracket \Gamma \vdash e_2 : \tau \rrbracket\eta\{x \mapsto \bigvee_i a_i\}\\
  \text{(by inductive hypothesis)} &=  \bigvee_i\llbracket \Gamma \vdash e_1: \tau \rrbracket\eta\{x\mapsto a_i\} \circ
  \bigvee_i\llbracket \Gamma \vdash e_2: \tau \rrbracket\eta\{x\mapsto a_i\}  \\
  &= \bigvee_i(\llbracket \Gamma \vdash e_1: \tau \rrbracket\eta\{x\mapsto a_i\} \circ 
  \llbracket \Gamma \vdash e_2: \tau \rrbracket\eta\{x\mapsto a_i\}) \\
  &= \bigvee_i \llbracket \Gamma \vdash e: \tau \rrbracket\eta\{x\mapsto a_i\}
 \end{align*}

  % equality operation
  \item $e = e_1 \circ e_2, \ \circ \in \{=,<,>,\leq,\geq\}$\\ 
  We have three cases to consider: either $\llbracket \Gamma \vdash e_1:\tau\rrbracket\{x\mapsto \bigvee_i a_i\} = 
  \perp$,  $\llbracket \Gamma \vdash e_2:\tau\rrbracket\{x\mapsto \bigvee_i a_i\} = \perp$, or both 
  $\llbracket \Gamma \vdash e_1:\tau\rrbracket\{x\mapsto \bigvee_i a_i\} \neq \perp$ and 
  $\llbracket \Gamma \vdash e_2:\tau\rrbracket\{x\mapsto \bigvee_i a_i\} \neq \perp$.
  
  In the case that $\llbracket \Gamma \vdash e_1:\tau\rrbracket\{x\mapsto \bigvee_i a_i\} = \perp$, by inductive
  hypothesis $\bigvee_i \llbracket \Gamma \vdash e_1 \rrbracket\{x\mapsto a_i\} = \perp$, so 
  $ \llbracket \Gamma \vdash e_1 \rrbracket\{x\mapsto a_i\} = \perp$ for all $i$. Then
  \begin{align*}
  \llbracket \Gamma \vdash e_1 \circ e_2\rrbracket\eta\{x \mapsto \bigvee_i a_i\}
  &=  \llbracket \Gamma \vdash e_1 \rrbracket\eta\{x \mapsto \bigvee_i a_i\} \circ
  \llbracket \Gamma \vdash e_2 \rrbracket\eta\{x \mapsto \bigvee_i a_i\}\\
  &= \perp \circ \llbracket e_2\rrbracket\{x\mapsto\bigvee_i a_i\} = \perp 
  \end{align*}
  Also,
  \begin{align*}
  \bigvee_i \llbracket \Gamma \vdash e_1 \circ e_2\rrbracket\eta\{x \mapsto a_i\}
  &= \bigvee_i \{ \llbracket \Gamma \vdash e_1 \rrbracket\eta\{x \mapsto a_i\} \circ
  \llbracket \Gamma \vdash e_2 \rrbracket\eta\{x \mapsto a_i\}\} \\
  &=\bigvee_i \{ \perp \circ
  \llbracket \Gamma \vdash e_2 \rrbracket\eta\{x \mapsto a_i\}\} \\
  &= \bigvee_i\{ \perp \} = \perp
  \end{align*}

  The same reasoning holds when $\llbracket \Gamma \vdash e_2:\tau\rrbracket\{x\mapsto \bigvee_i a_i\} = \perp$.
  
 In the case that both $\llbracket \Gamma \vdash e_1:\tau\rrbracket\{x\mapsto \bigvee_i a_i\} \neq \perp$ and 
 $\llbracket \Gamma \vdash e_2:\tau\rrbracket\{x\mapsto \bigvee_i a_i\} \neq \perp$, by inductive hypothesis 
 $\bigvee_i \llbracket \Gamma \vdash e_1:\tau\rrbracket\{x\mapsto a_i\} \neq \perp$ and
 $\bigvee_i \llbracket \Gamma \vdash e_2:\tau\rrbracket\{x\mapsto a_i\} \neq \perp$.
 Note $\llbracket \texttt{real} \rrbracket = \R_{\perp}$, so
 \begin{align*}
 \exists i_1 \forall i \geq i_1, \llbracket \Gamma \vdash e_1\rrbracket\eta\{x\mapsto a_i\} = v_1 \in \R \\
 \exists i_2 \forall i \geq i_2, \llbracket \Gamma \vdash e_2\rrbracket\eta\{x\mapsto a_i\} = v_2 \in \R \\ 
 \end{align*}
 Let $i' = max\{i_1, i_2\}$, and note
 \begin{align*}
  \llbracket \Gamma \vdash e_1 \circ e_2\rrbracket\eta\{x \mapsto \bigvee_i a_i\}
  &=  \llbracket \Gamma \vdash e_1 \rrbracket\eta\{x \mapsto \bigvee_i a_i\} \circ
  \llbracket \Gamma \vdash e_2 \rrbracket\eta\{x \mapsto \bigvee_i a_i\}\\
\text{(by inductive hypothesis) } &= \bigvee_i \llbracket \Gamma \vdash e_1 \rrbracket\eta\{x \mapsto  a_i\} \circ
  \bigvee_i \llbracket \Gamma \vdash e_2 \rrbracket\eta\{x \mapsto a_i\}\\
  &= \bigvee_{i \geq i'} \llbracket \Gamma \vdash e_1 \rrbracket\eta\{x \mapsto  a_i\} \circ
  \bigvee_{i \geq i'} \llbracket \Gamma \vdash e_2 \rrbracket\eta\{x \mapsto a_i\}\\
   &= v_1 \circ v_2
  \end{align*}
  Also,
  \begin{align*}
  \bigvee_i \llbracket \Gamma \vdash e_1 \circ e_2\rrbracket\eta\{x \mapsto a_i\}
  &= \bigvee_i \{ \llbracket \Gamma \vdash e_1 \rrbracket\eta\{x \mapsto a_i\} \circ
  \llbracket \Gamma \vdash e_2 \rrbracket\eta\{x \mapsto a_i\}\} \\
  &= \bigvee_{i \geq i'} \{ \llbracket \Gamma \vdash e_1 \rrbracket\eta\{x \mapsto a_i\} \circ
  \llbracket \Gamma \vdash e_2 \rrbracket\eta\{x \mapsto a_i\}\} \\
  &= \bigvee_{i \geq i'}\{ v_1 \circ v_2\} = v_1 \circ v_2
  \end{align*}  
  % conditional
 \item $e = \ \ifelse{e_1}{e_2}{e_3} : \tau$ \\ 
In the case that $\llbracket \Gamma \vdash e_1 : \texttt{bool} \rrbracket\eta\{x\mapsto  \bigvee_i a_i\} = \perp$, note that $\llbracket \Gamma \vdash e : 
\tau \rrbracket\eta\{x\mapsto  \bigvee_i a_i\} =  \perp$. By inductive hypothesis, 
$\llbracket \Gamma \vdash e_1 : \texttt{bool} \rrbracket\eta\{x\mapsto  \bigvee_i a_i\} = 
\bigvee_i\llbracket \Gamma \vdash e_1 : \tau \rrbracket\eta\{x\mapsto a_i\} = \perp$. 
Then it must be that
$\forall i, \llbracket \Gamma \vdash e_1 : \tau \rrbracket\eta\{x\mapsto a_i\} = \perp$. 
Thus, $\bigvee_i\llbracket \Gamma \vdash e : \tau \rrbracket\eta\{x\mapsto  a_i\} = \perp$, so 
$\llbracket \Gamma \vdash e : \tau \rrbracket\eta\{x\mapsto  \bigvee_i a_i\} = 
\bigvee_i\llbracket \Gamma \vdash e : \tau \rrbracket\eta\{x\mapsto  \bigvee_i a_i\}$. \\ \\
%
In the case that $\llbracket \Gamma \vdash e_1 : \texttt{bool} \rrbracket\eta\{x\mapsto  \bigvee_i a_i\} = true$, 
note that $\llbracket \Gamma \vdash e : 
\tau \rrbracket\eta\{x\mapsto  \bigvee_i a_i\} =  \llbracket \Gamma \vdash e_2 \rrbracket\eta\{x\mapsto  \bigvee_i a_i\}$. 
By inductive hypothesis, 
$\llbracket \Gamma \vdash e_1 : \texttt{bool} \rrbracket\eta\{x\mapsto  \bigvee_i a_i\} = 
\bigvee_i\llbracket \Gamma \vdash e_1 : \tau \rrbracket\eta\{x\mapsto a_i\} = true$. 
Since $\{a_i\}^{\infty}_{i=1}$ is a chain 
and $\llbracket\texttt{bool}\rrbracket$ is a flat CPO, there must exist $n \in \mathbb{N}$ such that 
$\forall i > n, \llbracket \Gamma \vdash e_1 : \texttt{bool}
\rrbracket\eta\{x\mapsto a_i\} = true$ and 
$\forall i \leq n, \llbracket \Gamma \vdash e_1 : \texttt{bool}\rrbracket\eta\{x\mapsto  a_i\} = \perp$. Thus, it must 
be that $\bigvee_i \llbracket \Gamma \vdash e : \tau \rrbracket\eta\{x\mapsto a_i\} = 
\llbracket \Gamma \vdash e_2 \rrbracket\eta\{x\mapsto  \bigvee_i a_i\} = 
\llbracket \Gamma \vdash e : \tau \rrbracket\eta\{x\mapsto  \bigvee_i a_i\}$. \\ 
%
In the case that $\llbracket \Gamma \vdash e_1 : \texttt{bool} \rrbracket\eta\{x\mapsto  \bigvee_i a_i\} = false$, by reasoning parallel to the previous case $\bigvee_i \llbracket \Gamma \vdash e : \tau \rrbracket\eta\{x\mapsto a_i\}  = 
\llbracket \Gamma \vdash e : \tau \rrbracket\eta\{x\mapsto  \bigvee_i a_i\}$.
 %pair
 \item $e = (e_1,e_2)$ 
 \begin{align*}
 \llbracket \Gamma \vdash e : \tau_1 \times \tau_2\rrbracket\eta\{x \mapsto a\}
 &= (\llbracket \Gamma \vdash e_1 : \tau_1 \rrbracket\eta\{x \mapsto a\}, \llbracket \Gamma \vdash e_2 : \tau_2 \rrbracket\eta\{x \mapsto a\}) \\
 \text{(by inductive hypothesis)} &= (\bigvee_i \llbracket \Gamma \vdash e_1 : \tau_1 \rrbracket\eta\{x \mapsto a_i\},
 \bigvee_i \llbracket \Gamma \vdash e_2 : \tau_2 \rrbracket\eta\{x \mapsto a_i\}) \\
 &= \bigvee_i ( \llbracket \Gamma \vdash e_1 : \tau_1 \rrbracket\eta\{x \mapsto a_i\},
  \llbracket \Gamma \vdash e_2 : \tau_2 \rrbracket\eta\{x \mapsto a_i\}) \\
  &= \bigvee_i \llbracket \Gamma \vdash (e_1, e_2): \tau_1 \times \tau_2 \rrbracket\eta\{x\mapsto a_i\}
  \end{align*}
% application
 \item $ e = e_1 \ e_2$ \\ 
Let $\beta = \llbracket \Gamma \vdash e_2 : \sigma \rrbracket$, and see that
\begin{align*}
\llbracket \Gamma \vdash e : \tau \rrbracket\eta\{x \mapsto \bigvee_i a_i \} &=
\llbracket \Gamma \vdash e_1 : \sigma \rightarrow \tau \rrbracket\eta\{x \mapsto \bigvee_i a_i\}
(\beta \ \eta\{x \mapsto \bigvee_i a_i\}) \\
&= \llbracket \Gamma \vdash \lambda (y:\sigma).e' : \tau \rrbracket\eta\{x \mapsto a\}
(\beta \ \eta\{x \mapsto a\}) \\
\text{(by inductive hypothesis)} &= 
\llbracket \Gamma \vdash \lambda (y:\sigma).e' : \tau \rrbracket\eta\{x \mapsto a\}
(\bigvee_i\beta\ \eta\{x \mapsto a_i\}) \\
&= \llbracket \Gamma. y : \sigma \vdash e' : \tau' \rrbracket\eta\{x \mapsto a\}\{y \mapsto (\bigvee_i\beta\ \eta\{x \mapsto a_i\})\} \\
\text{(by inductive hypothesis)}&= \bigvee_i \llbracket \Gamma. y : \sigma \vdash e' : \tau' \rrbracket\eta\{x \mapsto 
a_i\}\{y \mapsto (\beta \ \eta\{x \mapsto a_i\})\} \\
&=\bigvee_i \llbracket \Gamma \vdash e_1: \sigma \rightarrow \tau\rrbracket\eta\{x\mapsto a_i\}
(\beta \ \eta\{x\mapsto a_i\}) \\
&= \bigvee_i \llbracket \Gamma \vdash e:\tau\rrbracket\eta\{ x \mapsto a_i\} 
\end{align*}
 % abstraction
 \item $e = \lambda (y : \tau) . (e' : \tau')$
 \begin{enumerate}
 \item Then $\llbracket \Gamma \vdash e : \tau \rightarrow \tau' \rrbracket\eta\{x\mapsto  \bigvee_i a_i\}$ 
 is a function $f: \llbracket \tau \rrbracket \rightarrow \llbracket \tau'  \rrbracket$ such that 
 $\forall d \in \llbracket \tau \rrbracket, \ f(d) = \llbracket \Gamma \ y : \tau \vdash e' : \tau 
 \rrbracket\eta\{x\mapsto  \bigvee_i a_i\}\{y \mapsto d\}$. We want to show that
 
 \begin{align*}
 \tmden{\typing\Gamma e\tau \rightarrow \tau'}{\extend\eta x {\bigvee_i a_i}} &=
\bigvee_i\tmden{\typing\Gamma e\tau \rightarrow \tau'}{\extend\eta x {a_i}}
\end{align*}
 
Let $d \in \llbracket \tau \rrbracket$, and consider that
\begin{align*}
\llbracket \Gamma \vdash e : \tau \rightarrow \tau' \rrbracket\eta\{x\mapsto  \bigvee_i a_i\}(d) 
&= \llbracket \Gamma.y : \tau \vdash e' : \tau'\rrbracket\eta\{x\mapsto  \bigvee_i a_i\}
\{y \mapsto d\} \\
&= \llbracket \Gamma.y : \tau \vdash e' : \tau'\rrbracket(\eta\{y \mapsto d\})\{x\mapsto  \bigvee_i a_i\} \\
\text{(by inductive hypothesis)}&=\bigvee_i\llbracket \Gamma.y:\tau\vdash e':\tau'\rrbracket(\eta\{y\mapsto d\})
\{x\mapsto a_i\} \\
&=\bigvee_i\llbracket \Gamma.y:\tau\vdash e':\tau'\rrbracket(\eta\{x\mapsto a_i\})\{y\mapsto d\} \\
&= \bigvee_i \llbracket \Gamma \vdash \lambda y : \tau.e' : \tau' \rrbracket \eta\{x\mapsto a_i\}(d) \\
&= \bigvee_i \llbracket \Gamma \vdash e : \tau \rightarrow \tau' \rrbracket \eta\{x\mapsto a_i\}(d) \\ 
\end{align*}
%
 \item We want to show that $\llbracket \Gamma \vdash e : \tau \rightarrow \tau' \rrbracket\eta \in \llbracket \tau \rightarrow 
 \tau'\rrbracket$---that is, it is a continuous function from $\llbracket \tau \rrbracket$ to $\llbracket \tau' \rrbracket$. Let 
 $\{a_i\}^{\infty}_{i=1}$ be a chain of elements in $\llbracket \tau \rrbracket$, and see that
 \begin{align*}
 \llbracket \Gamma \vdash e : \tau \rightarrow \tau' \rrbracket\eta(\bigvee_i a_i) &= \llbracket \Gamma.x : \tau 
 \vdash e' : \tau'\rrbracket\eta \{x \mapsto \bigvee_i a_i\} \\
 \text{(by inductive hypothesis)} &= \bigvee_i \llbracket \Gamma.x : \tau \vdash e' : \tau'\rrbracket\eta\{x\mapsto a_i\}\\
 &= \bigvee_i \llbracket \Gamma \vdash e : \tau \rightarrow \tau' \rrbracket\eta(a_i) 
 \end{align*}
 \end{enumerate} 
 % fix operation
 We use several lemmas for this case, which are stated after the proof.
 \item $e = \texttt{fix} (\lambda f.\lambda x.e)$ \\
 \begin{enumerate}
\item Let $F_i = \llbracket \Gamma \vdash \lambda f. \lambda x.e\rrbracket\eta\{x\mapsto a_i\}$, and consider that 
 \begin{align*}
 \llbracket \texttt{fix} (\lambda f.\lambda x.e)\rrbracket\eta\{x\mapsto \bigvee_i a_i\} &= 
 fix(\llbracket \lambda f. \lambda x.e \rrbracket\eta\{x\mapsto \bigvee_i a_i\})\\
 \text{(by inductive hypothesis)} &= fix(\bigvee_i(F_i)) \\
 &= \bigvee_j(\bigvee_i((F_i)^j )\perp) \\
 &= \bigvee_j(\bigvee_i((F_i)^j \perp)) \\ 
\text{(by Lemma \ref{lem1})} &= \bigvee_i(\bigvee_j((F_i)^j \perp)) \\ 
&= \bigvee_i(fix(F_i))\\
&= \bigvee_i \llbracket \Gamma \vdash \texttt{fix}(\lambda f \lambda x.e)\rrbracket\eta\{x\mapsto a_i\}
 \end{align*}
 %
 \item Let $F = \llbracket \Gamma \vdash \lambda f. \lambda x.e\rrbracket\eta$. We want to show that 
 $\llbracket\Gamma\vdash\texttt{fix}(\lambda f \lambda x.e) : \tau 
 \rightarrow\tau\rrbracket\eta \in \llbracket \tau \rightarrow \tau \rrbracket$---that is, it is a continuous function from 
 $\llbracket \tau \rrbracket$ to $\llbracket \tau \rrbracket$. Let $\{a_i\}^{\infty}_{i=1}$ be a chain of elements in $\llbracket \tau \rrbracket$,
  and see that 
 \begin{align*}
 (\llbracket\Gamma\vdash\texttt{fix}(\lambda f \lambda x.e) : \tau \rightarrow\tau\rrbracket\eta)( \bigvee_i a_i) &= 
 fix(F)(\bigvee_i a_i)\\
 &=(\bigvee_j F^j \perp)(\bigvee_i a_i)\\
 &= \bigvee_j (F^j \perp(\bigvee_i a_i))\\ 
 \text{(by continuity of $F^j \perp$) }&= \bigvee_j(\bigvee_i(F^j \perp(a_i)))\\ 
 \text{(by Lemma \ref{lem2})} &=\bigvee(\bigvee(F^j \perp(a_i))) \\
 &= \bigvee_i(fix(F)(a_i))\\
 &= \bigvee_i(\llbracket\Gamma\vdash\texttt{fix}(\lambda f \lambda x.e) : \tau \rightarrow\tau\rrbracket\eta(a_i))\\
 \end{align*}
 \end{enumerate}
 \end{itemize}
 Thus, by definition, $\llbracket \Gamma \vdash \texttt{fix}(\lambda f \lambda x.e) \rrbracket\eta$ is a continuous function
 from $\llbracket \tau \rrbracket$ to $\llbracket \tau \rrbracket$.
 \end{proof}
 % Lemmas
 \begin{lemma} $\bigvee_j(\bigvee_i((F_i)^j \perp))= \bigvee_i(\bigvee_j((F_i)^j \perp))$\\
 \label{lem1}
 \end{lemma}
 \begin{proof}
 $\Rightarrow$ We want to show that\\ \\ \centerline{$\bigvee_j(\bigvee_i((F_i)^j \perp))
  \geq  \bigvee_i(\bigvee_j((F_i)^j \perp))$} \\ \\ First, notice that $\forall i, j \bigvee_j(\bigvee_i
 ((F_i)^j \perp)) \geq (F_i)^j$. \\
 Suppose, now, that 
 \begin{align*}
 &\bigvee_j(\bigvee_i((F_i)^j \perp)) < \bigvee_i(\bigvee_j((F_i)^j \perp))\\ 
 \Rightarrow \exists i \text{ s.t. \ \  \ } &\bigvee_j(\bigvee_i((F_i)^j \perp)) < \bigvee_j (F_i)^j \\
 \Rightarrow \exists i, j \text{ s.t. } &\bigvee_j(\bigvee_i((F_i)^j \perp)) < (F_i)^j \perp  \ \
 \Rightarrow\Leftarrow\\
 \end{align*}
 Thus, it must be that $\bigvee_j(\bigvee_i((F_i)^j \perp)) \geq  \bigvee_i(\bigvee_j((F_i)^j \perp))$ \\ \\
 $\Leftarrow$ By parallel reasoning, $\bigvee_j(\bigvee_i((F_i)^j \perp)) \leq  \bigvee_i(\bigvee_j((F_i)^j  \perp))$ \\ \\
 Therefore, $\bigvee_j(\bigvee_i((F_i)^j \perp)) = \bigvee_i(\bigvee_j((F_i)^j \perp))$ 
 \end{proof}
 %
 \begin{lemma}
 \label{lem2}
 $\bigvee_j(\bigvee_i(F^j \perp(x_i)))= \bigvee_i(\bigvee_j(F^j \perp(x_i)))$\\ 
 \end{lemma}
 \begin{proof} 
 By reasoning similar to Lemma \ref{lem1}. 
 \end{proof}
 %

 We will prove soundness of our denotational semantics for $\texttt{fix}$ expressions, in order to ensure that
 our interpretation of recurrences reflects their equational meaning.
 \begin{thm}
 \begin{align*}
 \llbracket \texttt{fix}(\lambda f. \lambda x.e)\rrbracket \eta = \llbracket \lambda x.e 
 	\{ f \mapsto \texttt{fix}(\lambda f\lambda x.e)\} \rrbracket\eta
 \end{align*}
 \end{thm}
 
\begin{proof}
We may show by a simple inductive argument that 
$\llbracket [x \mapsto e']e\rrbracket\eta = \llbracket e \rrbracket\eta\{x \mapsto \llbracket e' \rrbracket\eta\}$.
Let $\texttt{F} = \lambda f \lambda x.e$, and see that
\begin{align*}
\llbracket [f \mapsto \texttt{fix}(\texttt{F})]\lambda x.e\rrbracket\eta 
&= \llbracket \lambda x.e\rrbracket\eta\{f \mapsto \llbracket \texttt{fix}(\texttt{F})\rrbracket\} \\
&=\llbracket \lambda x.e\rrbracket\eta\{f \mapsto fix\}(\llbracket \texttt{F} \rrbracket\eta)\} \\
&= \llbracket \lambda f.\lambda x.e \rrbracket\eta(fix(\llbracket \texttt{F}\rrbracket\eta)\} \\
&= \llbracket F \rrbracket\eta(fix(\llbracket \texttt{F}\rrbracket\eta)) 
\end{align*}
Since $fix$ assigns the least fixed point to continuous functions, 
$\llbracket \texttt{F} \rrbracket\eta(fix(\llbracket \texttt{F}\rrbracket\eta)) = fix(\llbracket \texttt{F}\rrbracket\eta) =  \llbracket \texttt{fix}(\lambda f. \lambda x.e)\rrbracket \eta$.
\end{proof}
 
We will look at a recurrence Karp offers in this section as an example, to check whether recurrences in our 
language have the same solution. First, consider the following recurrence:
\begin{align*}
T(1) &= 0 \\
T(n) &= 1 + T(pn), \ 0 < p < 1
\end{align*} 
Consider that this recurrence only makes sense if its domain is the set $\{x \in \R \mid \exists k \in \N \text{ s.t. } p^kx = 1\}$;
we will assume that this set is the domain of $T$. 
 See, then, that $T(n) = 1 +  \log_{p}1/n $

We may express this recurrence in our language as the follows:
\begin{align*}
\texttt{T} = \texttt{fix}(\lambda f. \lambda x.\ifelse{x=1}{0}{1 + f(p*x)})
 \end{align*}
 We want to show that the denotation of this expression has the same behavior as the recurrence described above. 
 First, let $\texttt{F} =\lambda f. \lambda x.\ifelse{x=1}{0}{1 + f(p*x)}$, and note the following: 
 \begin{align*}
 \llbracket \texttt{T} \rrbracket &= fix(\llbracket \texttt{F} \rrbracket) \\
 \llbracket \texttt{F} \rrbracket &= F : \llbracket \texttt{real} \rightarrow \texttt{real}\rrbracket \rightarrow \llbracket \texttt{real} \rightarrow \texttt{real}\rrbracket \\ 
 &\text{ s.t. } \forall \alpha \in \llbracket \texttt{real}\rrbracket \rightarrow \llbracket \texttt{real} \rrbracket, \\   
 &F(\alpha) = \llbracket\lambda x.\ifelse{x=1}{0}{1 + f(p*x)}\rrbracket\{f \mapsto \alpha\} \\
 \llbracket \texttt{real}\rrbracket &= \R_{\perp}  
 \end{align*}
 By the CPO fixed-point theorem,  
\begin{align*}
 fix(F) &= \bigvee_n(F^n (\perp)) \\
&= \lim_{n \to \infty}\{F^n (\perp)\}
 \end{align*}
 \begin{lemma}
 $\forall x \in \R_{\perp}, \ F^{n+1}(\perp)(x) =  
 \begin{cases}
 0 \text{ if } x = 1 \\
 1 \text{ if } px = 1 \\
 \dots \\
 n - 1 \text{ if } \ p^{n}x = 1  \\
 \perp \text{ otherwise}
 \end{cases}$
 \end{lemma}
\begin{proof}

\emph{Base Cases: } $n = 0$
\begin{align*}
F^{n+1}(\perp)(x) &= F(\perp) \\
&= 
\begin{cases}
0 \text{ if } x=1 \\
1 + \perp(px) \text{ otherwise}
\end{cases} \\
&=
\begin{cases}
0 \text{ if } x=1 \\
\perp \text{ otherwise}
\end{cases} 
\end{align*}
Note $p^{n}x = p^0x = x$, so trivially the claim holds. \\
\emph{Inductive Case: } Suppose the claim holds for all $k < n$.
\begin{align*}
F^{n+1}(\perp)(x) &=  
\begin{cases}
0 \text{ if } x=1 \\
1 + F^{n}(\perp)(px) \text{ otherwise}
\end{cases} 
\end{align*}
By inductive hypothesis,
\begin{align*}
F^{n}(\perp)(px) &= F^{(n-1) + 1}(\perp)(px) \\
&=  
 \begin{cases}
 0 \text{ if } px = 1 \\
 1 \text{ if } p^2x = 1  \\
 \dots \\
 n - 1 \text{ if }  p^{n}x =1  \\
 \perp \text{ otherwise}
 \end{cases}
 \end{align*}
\end{proof}
See, then, that
\begin{align*}
F^{n+1}(\perp)(x) &=  
\begin{cases}
0 \text{ if } x=1 \\
1 \text{ if } px = 1 \\
2 \text{ if } p^2x = 1  \\
\dots \\
n-1 \text{ if } p^{n}x = 1  \\
\perp \text{ otherwise } 
\end{cases}
\end{align*}
Therefore, the claim holds for all $n$.

Then $\llbracket \texttt{T} \rrbracket = lim_{n \to \infty}\{F^n (\perp)\}$ is defined for all $x \in \{x \in \R \mid \exists k \in \N \text{ s.t. } p^kx = 1\}$ as
\begin{align*}
\lim_{n \to \infty}\{F^n (\perp)\}(x) = n - 1
\end{align*}
Where $n \in \N$ is the minimum number such that $p^nx<1$. Then $\llbracket \texttt{T} \rrbracket(x) = 1 + \log_{p}1/n= T(x)$. Thus, $\llbracket \texttt{T} \rrbracket = T$.
 





\end{document}
