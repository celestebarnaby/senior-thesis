\documentclass[12pt]{westhesis}
\title{A Syntax and Denotational Semantics for Probabilistic Recurrence Relations}

\usepackage{amsthm}
	% Some theorem-like environments, all of which share the same numbering.
	\theoremstyle{plain}
	\newtheorem{thm}{Theorem}
	\newtheorem{lemma}[thm]{Lemma}
	\newtheorem{prop}[thm]{Proposition}

	\theoremstyle{definition}
	\newtheorem{defn}{Definition}

\usepackage{setspace}
\doublespacing
\usepackage{amsmath}
\usepackage{float}
\usepackage{amssymb}
\usepackage{stmaryrd}
\usepackage{enumerate}
\usepackage{wasysym}
\usepackage{slashed}

\usepackage{natbib}

\let\wasysymLightning\lightning

\newcommand{\R}{\mathbb{R}}
\newcommand{\extR}{\R_{[0,+\infty]}}
\newcommand{\N}{\mathbb{N}}
\newcommand{\Z}{\mathbb{Z}}

\newcommand{\Ctxt}       {\mathcal{E}}
\newcommand{\InCtxt} [1] {\Ctxt[#1]}
\newcommand{\ssosredex}        {\rightarrow}
\newcommand{\ctxtreduce}       {\mapsto}
\newcommand{\sstep}     [3] [] {#2 &\ssosredex&  #3 &\textsc{#1}}
\newcommand{\ctxtstep}  [3] [] {#2 &\ctxtreduce& #3 &\textsc{#1}}
\newcommand{\tmden}[2]{\llbracket{#1}\rrbracket{#2}}
\newcommand{\tyden}[1]{\llbracket{#1}\rrbracket}
\newcommand{\typing}[3]{{#1}\vdash{#2}:{#3}}
\newcommand{\extend}[3]{#1\{{#2}\mapsto{#3}\}}
\newcommand{\sfix}{\mathtt{fix}}
\newcommand{\smfix}{\underline{\mathit{fix}}}
\newcommand{\set}[1]{\{#1\}}
\let\bottom\bot
\newcommand{\env}[1]{\{#1\}}
\newcommand{\bind}[2]{{#1}\mapsto{#2}}
\newcommand{\ifelse}[3]{\texttt{if } #1 \texttt{ then } #2 \texttt{ else } #3}
\newcommand{\real}{\texttt{real}}
\newcommand{\lst}{\texttt{list}}

\raggedbottom
\begin{document}

\begin{abstract}
abstract goes here:)
\end{abstract}

%Chapters
\chapter{Introduction}
Analyzing the complexity and performance of an algorithm is an important problem in computer science. We want our programs to run as quickly and as efficiently as possible; thus, it is crucial that we understand how much time and space it takes to execute a program, with respect to its input size. Prior work has been done on the development of tools
for automating the analysis of program complexity, which would allow us to make these judgements more quickly and accurately. 

However, such work has so far only offered a method for obtaining an upper bound on the worst-case runtime of an algorithm---which in certain cases does not reflect the actual runtime of an algorithm. In particular, this method does not
provide useful bounds on algorithms with stochastic processes, where the runtime may be quite slow in the worst case but
is, on average, much more efficient. Developing methods for extracting tighter upper-bounds on the costs of probabilistic algorithms would allow us to analyze a wider array of programs.

\section{Automated Complexity Analysis}

The cost of an algorithm is a function from the size of the input of the algorithm to the number of steps necessary to evaluate 
that input. As an example, consider the following $\texttt{add}$ function, which adds an integer $x$ to every element 
in a list $ys$ of integers: \\
\begin{verbatim}
let rec add x ys =
	match ys with
	| [] -> []
	| (y::ys') -> (y+x):: add x ys'
\end{verbatim}
In a traditional cost analysis of this function, we assume that integer addition requires some constant number of steps $c$. The cost of this function is dependent upon the length of the list---that is, the more integers in the list, the more adding we need to
do, and the longer the function will take to run. We may write a recurrence $T$ describing its cost, based on the length
$n$ of the input list:
\begin{align*}
T(0) &= 0 \\
T(n) &= c + T(n-1)
\end{align*}
To determine the cost of $\texttt{add}$, we need only solve this recurrence. We find that $T(n) = cn$, meaning this function
is linear in the length of the list. 

This approach has several shortcomings. To start, it is an ad hoc way to analyze cost. We cannot generalize
this result for use in other functions; thus, all traditional cost analyses involve writing recurrences by hand. Further, there 
is no formal connection between the algorithm and the recurrence, meaning we must manually verify that the recurrence
we write down accurately describes the function we are analyzing. The combination of these issues creates a lot of 
potential for human error.

Another problem is that we assume that the work done to each element of the list requires a constant number of steps. 
This assumption is fine in the case of $\texttt{add}$, where all we are doing is adding integers; however, if we consider
the more general $\texttt{map}$ function, we run into some complications. 
\begin{verbatim}
let rec map f ys =
	match ys with
	| [] -> []
	| (y::ys') -> f(y):: map f ys'
\end{verbatim}
Suppose $ys$ is a list of lists of integers, and $f$ is a function that performs selection sort on a single list. Then $f(y)$ does
not take a constant amount of work; rather, its cost is quadratic in the length of $y$. In order to write a correct recurrence
for $\texttt{map}$, we need some information about the cost of $f(y)$, where the size of $y$ is variable. This issue
becomes even thornier if the function $f$ is a composition of functions $g \circ h$. In this case, we need information about the cost of $g(h(y))$, meaning we in turn need information about the \emph{size} of $h(y)$--- the list output by applying a list $y$ to $h$. 

\cite{N.-Danner:2015aa} aim to resolve these issues by developing a formal method for extracting
recurrences that bound the complexity of higher-order programs. This work offers an extraction function that maps 
programs written in a language with structural list recursion--- referred to as a source language--- to a complexity. A 
complexity is a pair consisting of a cost--- the steps required to evaluate the program---and a potential--- the size of the 
output of this program. Further, this work provides a bounding theorem which guarantees that this extracted complexity is 
a bound on the actual cost of evaluating the source language program. 

The recurrences produced by the extraction function are all syntactic: they are cost-annotated source language
programs, rather than what we would traditionally think of as a recurrence for an algorithm. Danner et al. define a 
denotational semantics for this complexity language, as a formal means of constructing a mathematical object to
describe the meaning of programs in this language. The denotations of these cost-annotated programs resemble
more familiar recurrences.

\section{Probabilistic Recurrence Relations}

Some algorithms have probabilistic elements which affect the way we analyze them. A prime example of this is randomized 
quicksort---the standard quicksort algorithm wherein the pivot is chosen to be a random element in the list. In the worst case,
this algorithm is $O(n^2)$, making it no more efficient than other sorting algorithms such as bubble sort or insertion sort. 
However, in the average case it is $O(n\log n)$, making it one of the most efficient sorting algorithms. Hence, analyses of probabilistic algorithms must consider analyses of runtime other than worst case. One approach is to bound the upper
tails of the probability distribution of the algorithm. This allows us to ensure, for instance, that the probability of 
quicksort taking much longer than $O(n \log n)$ is very small. 

While there are proofs that derive the bounds of the cost of randomized quicksort, they commonly use
imprecise, ad hoc arguments: that is, they cannot be applied to other algorithms with similar randomized elements. 
\cite{Karp} offers a method for analyzing stochastic divide-and-conquer processes by describing their costs as recurrence relations of the form
\begin{align*}
T(x) = a(x) + T(h_1(x)) + \dots + T(h_n(x))
\end{align*}
where $x$ is a non-negative real number describing the size of the input, $h_1(x)$ through $h_n(x)$ are random 
variables describing the sizes of the subproblems (e.g. in the case of quicksort, $h_1(x)$ and $h_2(x)$ describe the sizes 
of the two sublists), $a$ is a function describing the work necessary to generate these subproblems, and $T(x)$ is a 
random variable describing the total running time of the algorithm on an input of size $x$. Karp then offers several methods 
for obtaining tight bounds on the upper tails of the probability distribution of $T(x)$. 

Note, however, that the recurrence relation $T(x) = a(x) + T(h(x))$ has some clear flaws that 
emerge when we attempt to assign types to its expressions. Since $T(x)$ is supposed to be random
variable over inputs of size $x$, $T(x)$ is a function of type 
$\Omega_x \rightarrow \R$, where $\Omega_x$ is the sample space of inputs of size $x$. Then $T$ is essentially a 
function of type $\R \rightarrow (\Omega_x \rightarrow \R)$. However, we see that $T$ recursively takes 
$h(x)$ as an argument, even though $h(x)$ is described as a random variable with type $\Omega_x \rightarrow \R$, 
rather than $\R$. This type-checking error is likely a result of the informality of the recurrence description, 
and we are meant to understand that $T$ takes not the 
random variable $h(x)$ as an argument, but the result of plugging an input $l \in \Omega_x$ into $h(x)$. Using this 
interpretation, we then have $T(x)(l) = a(x) + T(h(x)(l))$. But this is also problematic, since $T(x)(l)$, the result 
of plugging an input $l$ into $T(x)$, is a real number, while $T(h(x)(l))$ is a random variable. 

Taking this a step further, we could interpret this to mean $T(x)(l) = a(x) + T(h(x)(l))(l^\prime)$, where $l^\prime \in 
\Omega_{h(x)(l)}$ is the derived subproblem. But acquiring such an $l^\prime$ requires us to define an 
additional function $\hat{h}(x)$ that takes an input $l$ and returns the derived subproblem $l'$. 
Even worse, this function is of type $\Omega_x \rightarrow \Omega_{h(x)(l)}$ --- that is, its type is dependent 
upon the random variable $h(x)$. Thus, we find that Karp's seemingly straightforward recurrence relation quickly becomes 
increasingly complicated when we attempt to type-check it. Other researchers (\cite{Tassarotti:2017aa}) have corroborated 
these observations. 


\section{Contributions of This Thesis}

Karp offers the equation $T'(x) = a(x) + T'(m(x))$ as a deterministic counterpart of a recurrence relation $T(x) = a(x) + 
T(h(x))$, where, for all $x$, $m(x)$ is equal to the upper bound of $h(x)$. This thesis defines languages that allow
us to write and type-check both probabilistic recurrences of the form $T'$, and deterministic recurrences
of the form $T$. These languages comprise a simple expression language including real number constants, identifiers, basic arithmetic operations, booleans, and application. These languages must also provide
a way to express recursive functions, since this work centers around expressing recurrence relations. 
 
In Chapter 2 we will define a language and denotational semantics that formalizes deterministic recurrences.  A natural 
approach would be to use a general fixed-point construction and CPO semantics; however, this turns out to result in some 
unexpected problems. Instead, we define a language that has a more restrictive recursion construct $\texttt{rec}$, which 
we interpret using infinite sums.  

While Karp proves theorems about recurrences of the form $T(x) = a(x) + T(m(x))$, the recurrences that come up in his 
examples of applications take a different form. Thus in Chapter 3, we formulate a language for those recurrences, where 
we are in fact able to use a general fixed-point construction and CPO semantics.  We then investigate the difficulty of 
reconciling these two seemingly different forms of recursive definitions in Chapter 4. 

 Of course, the main point of Karp's paper is probabilistic recurrences, so we define a language and semantics for 
 them in Chapter 5.  This turns out to be quite complex: as discussed above, the typing of these recurrences itself is 
 already problematic. Our solution is to support dependent types in our language. Since it proves to be quite difficult to 
 combine dependent types with CPOs, we focus on a semantics that corresponds to the infinite sum interpretation for
 deterministic recurrences. We conclude in Chapter 6 with a discussion of how your work fits into Danner et al.'s 
 recurrence extraction framework and further work.
 


\chapter{Deterministic Recurrences as Infinite Sums}
In section 1, Karp states that the expected cost of stochastic processes that appear in divide-and-conquer algorithms may be 
described by
recurrence relations of the form $T(x) = a(x) + T(h(x))$, where $x$ is a non-negative real number, $h(x)$ and $T(x)$ are 
random variables, and $a$ is a non-negative, real-valued function. He then offers a deterministic counterpart to this
probabilistic recurrence, $T'(x) = a(x) + T'(m(x))$, where $m$ is a real-valued function such that for 
$\forall x \in \R, \ E(h(x)) \leq m(x) , 0 \leq m(x)$, and $m(x), m(x)/x$ are nondecreasing. He asserts that this recurrence 
has a least nonnegative solution $u$ given by the Tarski fixed-point theorem, $u(x) = \sum_{i=1}^{\infty}a(m^i(x))$, where 
$m^i$ is the $i^{th}$ iterate of $m$. The solution $u$ and the function $m$ are used in theorems 1.1 and 1.2 to obtain a 
bound on the upper tail of $T(x)$.

In this chapter, we will define an expression language that allows us to write and type-check
deterministic recurrences such as $T'$. Further, we will define a denotational semantics for this language, 
such that the interpretation of our recurrences exhibit the same behavior as Karp's recurrences. 

Consider that, in order to describe recurrence relations, our language must have some way of defining recursive functions.
A first attempt at this involves defining a PCF-like language with a $\texttt{real}$ type and a $\texttt{fix}$ operator
that assigns the least fixed-point to continuous functions using the CPO fixed-point theorem. Our denotational semantics 
then interprets all base types as flat-ordered CPOs---for instance, $\texttt{real}$  interprets to $R_{\perp}$, the reals with a 
bottom element $\perp$, with each real number comparable only to $\perp$. Following standard PCF, arrow types
$\tau \rightarrow \sigma$, interpret to a CPO whose bottom element is a function $\perp: \tau \rightarrow \sigma$ such
that $\perp(x) = \perp_{\sigma}, \forall x \in \tau$.

However, this approach proves problematic, due to the fact that the standard interpretation of the base operations is strict---e.g. the 
sum of $\perp$ and any other element should be $\perp$. However, if we consider a function $F: (\R_{\perp} \rightarrow 
\R_{\perp}) \rightarrow (\R_{\perp} \rightarrow \R_{\perp})$ such that $F(g) = a(x) + g(px)$, then by the CPO fixed-point
theorem, the least fixed point of F is $\bigvee\{F^n(\perp)\}^{\infty}_{n=1}$. Then 
\begin{align*}
F(\perp)(x) &= a(x) + \perp(px)\\
&= a(x) + \perp \\
&= \perp 
\end{align*}
so by induction we may show that for $\forall n, \ F^n(\perp)(x) = \perp \implies \bigvee\{F^n(\perp)\}^{\infty}_{n=1}  = \perp$. 
This result, is, clearly, not a solution to the recurrence $T(x) + a(x) + T(px)$, making this denotational semantics
unusable to us. 

A second approach involves altering the interpretation of the $\texttt{real}$ type to 
$\extR$---that is, the non-negative, extended reals. See that $\extR$ is a CPO: $0$ is a bottom element,
and any directed subset of $\extR$ must have a least upper bound. Such an interpretation
resolves the issue of all recursive functions interpreting to $\perp$. In the above example, for instance, we would have
\begin{align*}
F(\perp)(x) &= a(x) + \perp(px)\\
&= a(x) + 0 \\
&= a(x) \\
F(F(\perp))(x) &= a(x) + F(\perp)(px) \\
&= a(x) + a(px)
\end{align*}
so, by induction, 
$\forall n, \ F^n(\perp)(x) = \sum_{i=1}^{n}a(p^{i-1}x) \ \implies \bigvee\{F^n(\perp)\}^{\infty}_{n=1}(x) = \sum_{i=1}^{\infty}
a(p^{i-1}x)$. This $\emph{is}$ the solution to the recurrence $T(x) = a(x) + T(px)$, suggesting that this is the 
appropriate denotational semantics to use. However, this causes unexpected problems. For one, note that
 the standard interpretation of the equality operator is as follows:
\begin{align*}
 \llbracket e_1 = e_2 \rrbracket\eta &= 
 \begin{cases} 
      true \text{ if } (\llbracket e_1 \rrbracket\eta = \llbracket e_2 \rrbracket\eta \neq \perp) \\
      false \text{  if } (\llbracket e_1 \rrbracket\eta \neq \llbracket e_2\rrbracket\eta, \llbracket e_1 \rrbracket\eta \neq \perp, \llbracket e_2 \rrbracket\eta \neq \perp)\\
      \perp \text{ otherwise}
   \end{cases} 
\end{align*}
with similar rules for $<, >, \leq, \geq$. But if our bottom element of $\llbracket \texttt{real} \rrbracket$ is $0$, this 
prohibits us from writing a program where we compare a number to $0$. This is too restrictive, as there are many 
divide-and-conquer algorithms where we continue a process on the condition that an integer is non-negative, or a 
list is non-empty. Then we must rewrite our interpretation of such expressions as
\begin{align*}
 \llbracket e_1 = e_2 \rrbracket\eta &= 
 \begin{cases} 
      true \text{ if } \llbracket e_1 \rrbracket\eta = \llbracket e_2 \rrbracket\eta \\
      false \text{  if } \llbracket e_1 \rrbracket\eta \neq \llbracket e_2\rrbracket\eta\\
   \end{cases} 
\end{align*}

To see why this is a problem, consider the expression $x = \texttt{0}$ and the chain $\{n\}_{n=0}^{\infty} \in \llbracket 
\texttt{real} \rrbracket$. See, then, that
\begin{align*}
\llbracket x = \texttt{0} \rrbracket\{x \mapsto 0\} &= true \\
\llbracket x = \texttt{0} \rrbracket\{x \mapsto 1\} &= false \\
\llbracket x = \texttt{0} \rrbracket\{x \mapsto 2\} &= false \\
\ldots
\end{align*}
so, since $\llbracket \texttt{bool} \rrbracket$ is flat-ordered, the set 
$\{\llbracket x = \texttt{0}\rrbracket\{x \mapsto n\}\}_{n=0}^{\infty}$ has no supremum. However, as we will see in chapter 3,
in order to use the CPO fixed-point theorem it will be necessary to show that for all expressions $e$, environments $\Gamma$,
and chains $a_0 \leq a_1 \leq \ldots,$
\begin{align*}
  \llbracket \Gamma \vdash e : \tau \rrbracket\eta\{x\mapsto \bigvee_i a_i\}
  = \bigvee_i \llbracket \Gamma \vdash e : \tau \rrbracket\eta\{x \mapsto a_i\}
\end{align*}
%Also, (interpretation of non-terminating functions).

Thus, this approach will not work. In light of this, we define the following expression language. 
\[
\begin{array}{rcl}
\tau &::=& \texttt{real} \mid \texttt{bool} \mid \tau \times \tau \mid \tau \rightarrow \tau \\
e &::=& x  \mid \texttt{0} \mid \texttt{1} \mid \texttt{2} \mid \dotsc \mid \lambda x.e \mid e \ e \mid e + e \mid e - e \mid  e  *  e \mid e / e \mid \texttt{true} \mid \texttt{false} \mid \\
  && e  =  e \mid e < e \mid e > e \ | e \leq e \mid e \geq e \mid 
     \ifelse{e}{e}{e} \mid \\
    && \texttt{rec}(\lambda x.e, \lambda x.e) 
\end{array}
\]

Note that there is no $\texttt{fix}$ operator---instead, we have a $\texttt{rec}$ operator, which allows us to write 
recurrences of the form $T(x) = a(x) + T(m(x))$, where $T$, $a$, and $m$ are real-valued functions. While this is a very 
restrictive subset of recursive functions, it allows us to express the recurrences described by Karp in section 1, without
running into the problems of the previous approaches. 

\begin{figure}
\[
\begin{array}{lr}
\dfrac{}{\Gamma \vdash \texttt{0}: \texttt{real}}, \ \ \dfrac{}{\Gamma \vdash \texttt{1}: \texttt{real}}, \ldots \\ \\
\dfrac{}{\Gamma \vdash \texttt{true} : \texttt{bool}}, \ \ \dfrac{}{\Gamma \vdash \texttt{false} : \texttt{bool}} \\  \\
\dfrac{x : \tau \in \Gamma}{\Gamma \vdash x : \tau } \\ \\ 
\dfrac{\Gamma, x : \sigma \vdash e : \tau}{\Gamma \vdash \lambda (x : \sigma).e : \sigma \rightarrow \tau } \\ \\
\dfrac{\Gamma \vdash e_1: \sigma \rightarrow \tau \ \ \ \Gamma \vdash e_2 : \sigma}{\Gamma \vdash e_1 \ e_2 : \tau} \\ \\
\dfrac{\Gamma \vdash e_1 : \texttt{real} \ \ \ \Gamma \vdash e_2 : \texttt{real}}{\Gamma \vdash e_1 \circ e_2 : \texttt{real}}
, \circ \in \{+,-,*,/\} \\ \\ 
\dfrac{\Gamma \vdash e_1 : \texttt{real} \ \ \ \Gamma \vdash e_2 : \texttt{real}}{\Gamma \vdash e_1 \circ e_2 : \texttt{bool}}
, \circ \in \{=, <, >, \geq, \leq\} \\ \\ 
\dfrac{\Gamma \vdash e_1 : \texttt{bool} \ \ \ \Gamma \vdash e_2 : \tau \ \ \ \Gamma \vdash e_3 : \tau}
{\Gamma \vdash \ifelse{e_1}{e_2}{e_3} : \tau} \\ \\ 
\dfrac{\Gamma \vdash \lambda x.e_1 : \texttt{real} \rightarrow \texttt{real} \ \ \ \Gamma \vdash \lambda x.e_2 : 
\texttt{real} \rightarrow \texttt{real}}
{\Gamma \vdash \texttt{rec}(\lambda x.e_1, \lambda x.e_2) : \texttt{real} \rightarrow \texttt{real}} 
\end{array}
\]
\caption{Typing for the language.}
\label{fig:typing}
\end{figure}

\begin{figure}
\[
\begin{array}{lr}
e = e \\ \\
\texttt{0} + \texttt{0} = \texttt{0}, \ \texttt{0} + \texttt{1} = 1, \ldots, \ \texttt{3} + \texttt{5} = \texttt{8}, \ldots  \\
\text{equivalent rules for } -, \ * \text{, and } /\text{ operations.}
\\ \\
(n =n) = \texttt{true} \\ (n=m) = \texttt{false}\text{, provided } n, \ m \text{ distinct numerals.}\\ \\ 
\ifelse{\texttt{ true }}{e_1}{e_2} = e_1 \\
\ifelse{\texttt{ false }}{e_1}{e_2} = e_2 \\ \\ 
\lambda x.e = \lambda y.[x \mapsto y]e, \text{provided } y \text{ not free in } e. \\ \\ 
\lambda x.e_1 \ e_2 = [x \mapsto e_2]e_1 \\
\texttt{rec}(\lambda x.e_1, \lambda x.e_2) = 
			\lambda x.(e_1 + \texttt{rec}(\lambda x.e_1, \lambda x.e_2)(e_2)) 
\end{array}
\]
\caption{Equational semantics for the language.}
\label{fig:eqsem}
\end{figure}

\begin{figure}
 \begin{align*}
\llbracket \texttt{real} \rrbracket &= \R \\
 \llbracket \texttt{bool} \rrbracket &= {\{true, false\}} \\
 \llbracket \tau \times \sigma \rrbracket &= \llbracket \tau \rrbracket \times \llbracket \sigma \rrbracket  \\
 \llbracket \tau \rightarrow \sigma \rrbracket &= \{f: \llbracket \tau \rrbracket \rightarrow \llbracket \sigma \rrbracket \mid
 \ f \text{ is continuous}\} 
 \end{align*}
 \caption{Denotational semantics for types.}
 \label{fig:densemtypes}
 \end{figure}
 \begin{figure}
\begin{align*}
 \llbracket \texttt{0} \rrbracket\eta &= 0, \  \llbracket \texttt{1} \rrbracket\eta = 1, \ \ldots \\
  \llbracket x : \tau \rrbracket\eta &= \eta(x) \\
  \llbracket \lambda (x : \tau) . (e : \sigma) \rrbracket\eta &= f : \llbracket \tau \rrbracket \rightarrow \llbracket \sigma \rrbracket \\
\text{ s.t. } \forall d \in \llbracket \tau \rrbracket, f(d) &= \llbracket e \rrbracket\eta\{ x \mapsto d \} \\
 \llbracket e_1 \ e_2 \rrbracket \eta &= \llbracket e_1 \rrbracket\eta ( \llbracket e_2 \rrbracket\eta ) \\
 \llbracket e_1 + e_2 \rrbracket\eta &= \llbracket e_1 \rrbracket\eta + \llbracket e_2 \rrbracket\eta \\
 \llbracket e_1 - e_2 \rrbracket\eta &= \llbracket e_1 \rrbracket\eta - \llbracket e_2 \rrbracket\eta \\
 \llbracket e_1 * e_2 \rrbracket\eta &= \llbracket e_1 \rrbracket\eta * \llbracket e_2 \rrbracket\eta \\
  \llbracket e_1 / e_2 \rrbracket\eta &= \llbracket e_1 \rrbracket\eta / \llbracket e_2 \rrbracket\eta \\
  \llbracket \texttt{true} \rrbracket\eta &= true, \ \llbracket \texttt{false} \rrbracket\eta = false \\
 \llbracket e_1 = e_2 \rrbracket\eta &= 
 \begin{cases} 
      true \text{ if } \llbracket e_1 \rrbracket\eta = \llbracket e_2 \rrbracket\eta \\
      false \text{  if } \llbracket e_1 \rrbracket\eta \neq \llbracket e_2\rrbracket\eta \\
   \end{cases}
 \\
\text{Similar rules for $<, \ \leq , >$\, and $\geq$} \\
  \llbracket \ifelse{e_1}{e_2}{e_3} \rrbracket \eta &= 
 \begin{cases} 
      \llbracket e_2 \rrbracket\eta \text{ if } \llbracket e_1 \rrbracket\eta = true \\
      \llbracket e_3 \rrbracket\eta \text{ if } \llbracket e_1 \rrbracket\eta = false \\
   \end{cases}
  \\
   \llbracket  \texttt{rec} (e_1 : \texttt{real} \rightarrow \texttt{real}, e_2 : \texttt{real} \rightarrow \texttt{real}) \rrbracket\eta &= rec : \R \rightarrow \R \text{ s.t. } \forall d \in \R, \\
   rec(d) &= \sum_{i=0}^{\infty}\llbracket \lambda x.e_1 \rrbracket\eta ((\llbracket \lambda x.e_2 \rrbracket\eta)^i(d))
 \end{align*}
 \caption{Denotational semantics for expressions.}
 \label{fig:densemexps}
 \end{figure}
 
  We will prove soundness of our denotation for $\texttt{rec}$ expressions, in order to ensure that
 our interpretation of recurrences reflects their equational meaning.
 \begin{thm}
 \begin{align*}
 \llbracket \texttt{rec}(\lambda x.e_1, \lambda x.e_2) \rrbracket \eta 
 = \llbracket \lambda x.(e_1 + \texttt{rec}(\lambda x.e_1, \lambda x.e_2)(e2)) \rrbracket \eta
 \end{align*} 
 \end{thm}
 
 \begin{proof}
 Note that
 \begin{align*}
 \llbracket \lambda x.(e_1 + \texttt{rec}(\lambda x.e_1, \lambda x.e_2)(e_2)) \rrbracket \eta
 &= f : \R \rightarrow \R \text{ s.t. } \forall d \in \R, \\
 f(d) =  \llbracket e_1 \rrbracket\{x\mapsto d\} &+ \llbracket \texttt{rec}(\lambda x.e_1, \lambda x.e_2)(e_2)\rrbracket\eta
 \{x\mapsto d\} \\
 = \llbracket \lambda x.e_1\rrbracket(d) &+ \llbracket\texttt{rec}(\lambda x.e_1, \lambda x.e_2)\rrbracket\eta
 (\llbracket e_2 \rrbracket\eta\{x \mapsto d\}) \\
 =  \llbracket \lambda x.e_1\rrbracket(d) &+ \llbracket\texttt{rec}(\lambda x.e_1, \lambda x.e_2)\rrbracket\eta
 (\llbracket \lambda x.e_2 \rrbracket(d)) \\
 = \llbracket \lambda x.e_1\rrbracket(d) &+ \sum_{i=0}^{\infty} \llbracket \lambda x.e_1\rrbracket\eta
 ((\llbracket \lambda x.e_2 \rrbracket\eta)^i (\llbracket \lambda x.e_2\rrbracket\eta(d))) \\
 = \llbracket \lambda x.e_1\rrbracket(d) &+ \sum_{i=1}^{\infty} \llbracket \lambda x.e_1\rrbracket\eta
 ((\llbracket \lambda x.e_2 \rrbracket\eta)^i (d)) \\
&= \sum_{i=0}^\infty \llbracket \lambda x.e_1 \rrbracket\eta ((\llbracket \lambda x.e_2\rrbracket\eta)^i(d)) \\
&=  \llbracket \texttt{rec}(\lambda x.e_1, \lambda x.e_2) \rrbracket \eta (d)
\end{align*}
\end{proof}

\section{Example}
Consider the example Karp offers, wherein $m(x) = px$, with $p$ a
positive constant less than 1, and $a(x) = 0, \ x < 1, \ a(x) = 1, \ x \geq 1$.
Then $T(x) = a(x) + T(px)$, so by a simple inductive argument we can show that 

\begin{align*}
T(x) &= 
 \begin{cases}
 0 \text{ if } x < 1 \\
 k \text{ if }  \dfrac{1}{p^{k-1}} \leq x < \dfrac{1}{p^k} \\
 \end{cases}
 \end{align*}
 We express this function in our expression language as follows: 
 \begin{align*}
 \texttt{T} = \texttt{rec}(\lambda x.(\ifelse{x<\texttt{1}}{\texttt{0}}{\texttt{1}}), \lambda x. \texttt{p}*x)
 \end{align*}
 We want to show that the interpretation of this expression has the same behavior as the function described by Karp.
 Let $\texttt{a} = \lambda x.\ifelse{x<1}{0}{1}, \ \texttt{m} = \lambda x.\texttt{p}*x$, and see that
 \begin{align*}
  \llbracket \texttt{a} \rrbracket &= \underline{a} : \R \rightarrow \R \text{ s.t. } \underline{a}(d) = 0, 
  \ d<1, \underline{a}(d) = 1, \ d \geq 1  \\
 \llbracket \texttt{m} \rrbracket &= \underline{m} : \R \rightarrow \R \text{ s.t } \forall d \in \R, \
 \underline{m}(d) = pd \\
 \llbracket \texttt{T} \rrbracket&= \llbracket \texttt{rec}(\texttt{a}, \texttt{m}) \rrbracket \\
 &= rec: \R \rightarrow \R \text{ s.t. } \forall d \in \R, \\
 &rec(d) = \sum_{i=0}^{\infty}\llbracket \texttt{a} \rrbracket (\llbracket \texttt{m} \rrbracket^i(d)) \\
 &= \sum_{i=0}^{\infty}\underline{a} (\underline{m}^i(d)) \\
 &= \sum_{i=0}^{\infty}\underline{a} (p^i(d))
 \end{align*}
Note that, for $\forall d<1 \in \R, \ \llbracket \texttt{T} \rrbracket(d) = 0$
and for $\forall d \geq 1 \in R$, since $p < 1, \ \exists k \in \N \text{ s.t } \dfrac{1}{p^{k-1}} \leq d < \dfrac{1}{p^k}$. Then
\begin{align*}
\llbracket \texttt{T} \rrbracket(d)&= \sum_{i=0}^{\infty}\underline{a} (p^i(d)) \\
&= \sum_{i=0}^{k-1}\underline{a} (p^i(d)) + \sum_{i=k}^{\infty}\underline{a} (p^i(d)) \\
&= k + 0 \\
&= k
\end{align*}
Therefore, $\llbracket \texttt{T} \rrbracket = T$
 


We define an expression language which will allow us to write and type-check recurrence relations such as 
those described by Karp. The syntax of this language is mostly analogous to PCF, with one key exception: instead of a 
$\texttt{nat}$ type, we have a $\texttt{real}$ type, as this is the domain of the functions Karp discusses. \\ 
\[
\begin{array}{rcl}
\tau &::=& \texttt{real} \mid \texttt{bool} \mid \tau \times \tau \mid \tau \rightarrow \tau \\
e &::=& x  \mid \texttt{0} \mid \texttt{1} \mid \texttt{2} \mid \dotsc \mid \lambda x.e \mid e \ e \mid e + e \mid e - e \mid  e  *  e \mid e / e \mid \texttt{true} \mid \texttt{false} \mid \\
  && e  =  e \mid e < e \mid e > e \ | e \leq e \mid e \geq e \mid 
     if \ e \ then \ e \ else \ e \mid \texttt{fix} (\lambda f.\lambda x.e) \\
v &::=& x  \mid \texttt{0} \mid \texttt{1} \mid \texttt{2} \mid \dotsc \mid \lambda x.e \mid \texttt{true} \mid \texttt{false} \mid \texttt{fix} (\lambda f.\lambda x.e)
\end{array}
\]

\section{Type System}

\[
\begin{array}{lr}
\dfrac{}{\Gamma \vdash \texttt{0}: \texttt{real}}, \ \ \dfrac{}{\Gamma \vdash \texttt{1}: \texttt{real}}, \ldots \\
\dfrac{}{\Gamma \vdash \texttt{true} : \texttt{bool}}, \ \ \dfrac{}{\Gamma \vdash \texttt{false} : \texttt{bool}} \\ \\ \\
\dfrac{x : \tau \in \Gamma}{\Gamma \vdash x : \tau } \\ \\ \\
\dfrac{\Gamma, x : \sigma \vdash e : \tau}{\Gamma \vdash \lambda (x : \sigma).e : \sigma \rightarrow \tau } \\ \\ \\ 
\dfrac{\Gamma \vdash e_1: \sigma \rightarrow \tau \ \ \ \Gamma \vdash e_2 : \sigma}{\Gamma \vdash e_1 \ e_2 : \tau} \\ \\ \\
\dfrac{\Gamma \vdash e_1 : \texttt{real} \ \ \ \Gamma \vdash e_2 : \texttt{real}}{\Gamma \vdash e_1 \circ e_2 : \texttt{real}}
, \circ \in \{+,-,*,/\} \\ \\ \\
\dfrac{\Gamma \vdash e_1 : \texttt{real} \ \ \ \Gamma \vdash e_2 : \texttt{real}}{\Gamma \vdash e_1 \circ e_2 : \texttt{bool}}
, \circ \in \{=, <, >, \geq, \leq\} \\ \\ \\
\dfrac{\Gamma \vdash e_1 : \texttt{bool} \ \ \ \Gamma \vdash e_2 : \tau \ \ \ \Gamma \vdash e_3 : \tau}
{\Gamma \vdash \ifelse{e_1}{e_2}{e_3} : \tau} \\ \\ \\
\dfrac{\Gamma \vdash \lambda f. \lambda x.e : (\tau \rightarrow \tau) \rightarrow (\tau \rightarrow \tau)}
{\Gamma \vdash \texttt{fix}(\lambda f. \lambda x.e) : \tau \rightarrow \tau} \\ \\ \\ 
\end{array}
\]

\section{Equational Semantics}
\[
\begin{array}{lr}
e = e \\ \\
\texttt{0} + \texttt{0} = \texttt{0}, \ \texttt{0} + \texttt{1} = 1, \ldots, \ \texttt{3} + \texttt{5} = \texttt{8}, \ldots  \\
\text{equivalent rules for } -, \ * \text{, and } /\text{ operations.}
\\ \\
(n =n) = \texttt{true} \\ (n=m) = \texttt{false}\text{, provided } n, \ m \text{ distinct numerals.}\\ \\ 
\ifelse{\texttt{ true }}{e_1}{e_2} = e_1 \\
\ifelse{\texttt{ false }}{e_1}{e_2} = e_2 \\ \\ 
\lambda x.e = \lambda y.e\{x \mapsto y \}, \text{provided } y \text{ not free in } e. \\ \\ \\
\lambda x.e_1 \ e_2 = e_1\{x \mapsto e_2\} \\
\texttt{fix}(\lambda f. \lambda x.e) = (\lambda x.e)\{f \mapsto \texttt{fix}(\lambda f.\lambda x.e)\}
\end{array}
\]

\section{Denotational Semantics}
The denotational semantics of our expression language is mostly standard, with types interpreting to flat-ordered sets. Again, 
the exception is the $\texttt{real}$ type, which interprets to the non-negative reals, with an additional element $+\infty$ such 
that for $\forall x \in R$, $x<+\infty$. (Need more explanation of this here).
 \begin{align*}
\llbracket \texttt{real} \rrbracket &= \R_{[0, +\infty]} = \{x \in \R \ | \ x \geq 0\} \cup \{+\infty\} \\
 \llbracket \texttt{bool} \rrbracket &= {\{true, false\}}_{\perp} \\
 \llbracket \tau \times \sigma \rrbracket &= \llbracket \tau \rrbracket \times \llbracket \sigma \rrbracket  \\
 \llbracket \tau \rightarrow \sigma \rrbracket &= \{f: \llbracket \tau \rrbracket \rightarrow \llbracket \sigma \rrbracket \ : 
 \ f \text{ is continuous}\}
 \end{align*}
\begin{align*}
 \llbracket \texttt{0} \rrbracket\eta &= 0, \  \llbracket \texttt{1} \rrbracket\eta = 1, \ \ldots \\
  \llbracket x : \tau \rrbracket\eta &= \eta(x) \\
  \llbracket \lambda (x : \tau) . (e : \sigma) \rrbracket\eta &= f : \llbracket \tau \rrbracket \rightarrow \llbracket \sigma \rrbracket
\text{ s.t. } \forall d \in \llbracket \tau \rrbracket, f(d) = \llbracket e \rrbracket\eta\{ x \mapsto d \} \\
 \llbracket e_1 \ e_2 \rrbracket \eta &= \llbracket e_1 \rrbracket\eta ( \llbracket e_2 \rrbracket\eta ) \\
 \llbracket e_1 + e_2 \rrbracket\eta &= \llbracket e_1 \rrbracket\eta + \llbracket e_2 \rrbracket\eta \\
 \llbracket e_1 - e_2 \rrbracket\eta &= \llbracket e_1 \rrbracket\eta - \llbracket e_2 \rrbracket\eta \\
 \llbracket e_1 * e_2 \rrbracket\eta &= \llbracket e_1 \rrbracket\eta * \llbracket e_2 \rrbracket\eta \\
  \llbracket e_1 / e_2 \rrbracket\eta &= \llbracket e_1 \rrbracket\eta / \llbracket e_2 \rrbracket\eta \\
  \llbracket \texttt{true} \rrbracket\eta &= true, \ \llbracket \texttt{false} \rrbracket\eta = false \\
 \llbracket e_1 = e_2 \rrbracket\eta &= 
 \begin{cases} 
      true \text{ if } (\llbracket e_1 \rrbracket\eta = \llbracket e_2 \rrbracket\eta \neq \perp) \\
      false \text{  if } (\llbracket e_1 \rrbracket\eta \neq \llbracket e_2\rrbracket\eta, \llbracket e_1 \rrbracket\eta \neq \perp, \llbracket e_2 \rrbracket\eta \neq \perp)\\
      \perp \text{ otherwise}
   \end{cases}
 \\
  \llbracket e_1 < e_2 \rrbracket\eta &= 
 \begin{cases} 
      true \text{ if } (\llbracket e_1 \rrbracket\eta < \llbracket e_2 \rrbracket\eta, \llbracket e_1 \rrbracket\eta \neq \perp, \llbracket e_2 \rrbracket\eta \neq \perp) \\
      false \text{  if } (\llbracket e_1 \rrbracket\eta \geq \llbracket e_2\rrbracket\eta, \llbracket e_1 \rrbracket\eta \neq \perp, \llbracket e_2 \rrbracket\eta \neq \perp)\\
      \perp \text{ otherwise}
   \end{cases}
 \\
  \llbracket e_1 > e_2 \rrbracket\eta &= 
 \begin{cases} 
      true \text{ if } (\llbracket e_1 \rrbracket\eta > \llbracket e_2 \rrbracket\eta, \llbracket e_1 \rrbracket\eta \neq \perp, \llbracket e_2 \rrbracket\eta \neq \perp) \\
      false \text{  if } (\llbracket e_1 \rrbracket\eta \leq \llbracket e_2\rrbracket\eta, \llbracket e_1 \rrbracket\eta \neq \perp, \llbracket e_2 \rrbracket\eta \neq \perp)\\
      \perp \text{ otherwise}
   \end{cases}
 \\
  \llbracket e_1 \leq e_2 \rrbracket\eta &= 
 \begin{cases} 
      true \text{ if } (\llbracket e_1 \rrbracket\eta \leq \llbracket e_2 \rrbracket\eta, \llbracket e_1 \rrbracket\eta \neq \perp, \llbracket e_2 \rrbracket\eta \neq \perp) \\
      false \text{  if } (\llbracket e_1 \rrbracket\eta > \llbracket e_2\rrbracket\eta, \llbracket e_1 \rrbracket\eta \neq \perp, \llbracket e_2 \rrbracket\eta \neq \perp)\\
      \perp \text{ otherwise}
   \end{cases}
 \\
  \llbracket e_1 \geq e_2 \rrbracket\eta &= 
 \begin{cases} 
      true \text{ if } (\llbracket e_1 \rrbracket\eta \geq \llbracket e_2 \rrbracket\eta, \llbracket e_1 \rrbracket\eta \neq \perp, \llbracket e_2 \rrbracket\eta \neq \perp) \\
      false \text{  if } (\llbracket e_1 \rrbracket\eta < \llbracket e_2\rrbracket\eta, \llbracket e_1 \rrbracket\eta \neq \perp, \llbracket e_2 \rrbracket\eta \neq \perp)\\
      \perp \text{ otherwise}
   \end{cases}
 \\
  \llbracket \ifelse{e_1}{e_2}{e_3} \rrbracket \eta &= 
 \begin{cases} 
      \llbracket e_2 \rrbracket\eta \text{ if } \llbracket e_1 \rrbracket\eta = true \\
      \llbracket e_3 \rrbracket\eta \text{ if } \llbracket e_1 \rrbracket\eta = false \\
      \perp \text{      otherwise} \\
   \end{cases}
  \\
   \llbracket  \texttt{fix} (\lambda f.\lambda x.e) \rrbracket\eta &= fix\llbracket \lambda f.\lambda x.e \rrbracket\eta \
 \text{, where } fix \text{ assigns the least fixed point} \\ 
 &\text{ \ \ \ \ \ \ \ \ \ \ \ \ \ \ \ \ \ \ \ \ \ \ \ to continuous functions} \\
 \end{align*}
\begin{thm}
For all types $\tau, \ \llbracket \tau \rrbracket$ is a CPO --- that is, $\llbracket \tau \rrbracket$ is an ordered 
set with a bottom element, such that for $\forall C \subset \llbracket \tau \rrbracket$, if  $C$ is a chain, then $C$ has a least 
upper bound in $\llbracket \tau \rrbracket$. \\
\end{thm}
\begin{proof}
by induction on the structure of $\tau$. \\
\emph{Base Cases: }
\begin{itemize}
\item $\tau = \texttt{real}$ \\ \\
In this case, $\llbracket \tau \rrbracket = \extR$. Then $0$ is a bottom element of $\llbracket \tau \rrbracket$, and for 
any chain $C \subset \R, \ C$ is bounded by $+\infty$, so by the least upper bound property $C$ has a least upper bound in 
$\llbracket \tau \rrbracket$.
\item $\tau = \texttt{bool}$ \\ \\
In this case, $\llbracket \tau \rrbracket$ is set with flat ordering, and is thus a CPO. \\ 
\end{itemize}
\emph{Inductive Cases: }
\begin{itemize}
\item $\tau = \tau_1 \times \tau_2$ \\ \\
Note $\llbracket \tau \rrbracket = \llbracket \tau_1 \rrbracket \times \llbracket \tau_2 \rrbracket$. By inductive hypothesis,
$\llbracket \tau_1 \rrbracket$ and $\llbracket \tau_2 \rrbracket$ are CPO's, and thus have bottom elements $\perp_1$ and $
\perp_2$, respectively. Thus $(\perp_1, \perp_2)$ is a bottom element of $\tau$. 

Let $C \subset \llbracket \tau \rrbracket$ be a
chain, and define
\begin{center}
$C_1 = \{x \in \llbracket \tau_1 \rrbracket : (x,y) \in C\},$ \\
$C_2 = \{ y \in \llbracket \tau_2 \rrbracket: (x,y) \in C\}$. \\ 
\end{center}
Then $C_1 \subset \llbracket \tau_1 \rrbracket$ and $C_2 \subset \llbracket \tau_2 \rrbracket$ are chains, so $\bigvee C_1$
and $\bigvee C_2$ exist. We claim that $(\bigvee C_1, \bigvee C_2)$ is the least upper bound of $C$. 

Let $(c_1, c_2) \in C$. Then $c_1 \in C_1$ and $c_2 \in C_2$, so $c_1 \leq \bigvee C_1$ and $c_2 \leq \bigvee C_2$. Thus, 
$(c_1, c_2) \leq (\bigvee C_1, \bigvee C_2)$, so $(\bigvee C_1, \bigvee C_2)$ is an upper bound of $C$.

Now let $(x,y) \in \llbracket \tau \rrbracket$ be an upper bound of C. Then $\forall c_1 \in C_1, \ c_1 \leq x$, so $x$ is an upper 
bound of $C_1$. Similarly, $y$ is an upper bound of $C_2$. Thus, it must be that $\bigvee C_1 \leq x$ and 
$\bigvee C_2 \leq y$, so $(\bigvee C_1, \bigvee C_2) \leq (x,y)$. Therefore, $(\bigvee C_1, \bigvee C_2)$ is the least upper
bound of $C$.
 
See, then, that by definition $\llbracket \tau \rrbracket$ is a CPO.

\item $\tau = \tau_1 \rightarrow \tau_2$ \\ \\
Note $\llbracket \tau \rrbracket = \{f : \llbracket \tau_1 \rrbracket \rightarrow \llbracket \tau_2 \rrbracket : f \text{ is continuous}\}
$. By inductive hypothesis, $\llbracket \tau_1 \rrbracket$ and $\llbracket \tau_2 \rrbracket$ are CPO's, and thus have bottom
elements $\perp_1$ and $\perp_2$, respectively. Let $\perp: \llbracket \tau_1 \rrbracket \rightarrow \llbracket \tau_2 
\rrbracket$ be a function such that, for $\forall x \in \llbracket \tau_1 \rrbracket, \ \perp(x) = \perp_2$. See, then, that 
$\perp$ is a continuous function that is less than or equal to every function in $\llbracket \tau \rrbracket$ (using pointwise
ordering), so $\perp$ is the bottom element of $\llbracket \tau \rrbracket$.  

Let $C \subset \llbracket \tau \rrbracket$ be a 
chain, and consider that for $\forall x \in \tau_1, \{f(x) : f \in C\}$ is a chain in $\llbracket \tau_2 \rrbracket$. Call this set $F_x$,
and note that $\bigvee F_x$ exists. We define a function $g: \llbracket \tau_1 \rrbracket \rightarrow \llbracket \tau_2 \rrbracket$ 
such that $\forall x \in \llbracket \tau_1 \rrbracket, \ g(x) = \bigvee F_x$. We claim that $g = \bigvee C$. 

Let $f \in C$, and note that for $\forall x \in \llbracket \tau_1 \rrbracket, \ f(x) \leq \bigvee F_x =g(x)$. Then $g$ is an upper bound of $C$.

Now let $\rho \in \llbracket \tau \rrbracket$ be an upper bound of $C$. Then for $\forall x \in \llbracket \tau_1 \rrbracket, \forall f \in C, \ f(x) \leq \rho(x)$, so $\rho(x)$ is an upper bound of $\{f(x) : f \in C\} = F_x$. Thus $\rho(x) \geq F_x = g(x), \ \forall x \in 
\llbracket \tau_1 \rrbracket$, so $g \leq \rho$. Therefore, g is the least upper bound of $C$.

See, then, that by definition $\llbracket \tau \rrbracket$ is a CPO. \\
\end{itemize} 
\end{proof}
%
\begin{thm} 
(Need to update these proofs)
For all expressions $e$ and environments $\Gamma$,
\begin{enumerate}
\item For all $\Gamma\vdash e : \tau$ and all $\Gamma$-environments~$\eta$,
$\tmden{\typing\Gamma e\tau}\eta\in\tyden{\tau}$.
\item For all chains $a_0\leq a_1\leq\dotsb$,
$\tmden{\typing\Gamma e\tau}{\extend\eta x {\bigvee_i a_i}} =
\bigvee_i\tmden{\typing\Gamma e\tau}{\extend\eta x {a_i}}$.
\end{enumerate}
\end{thm}
\begin{proof}
by induction on the structure of $e$. For all cases, let $\{ \eta_i \}^{\infty}_{i=1}$ be a chain, and suppose $ a = \bigvee_i a_i$. 
Note that 1 is trivial in all but the abstraction and $\texttt{fix}$ cases.\\
 \emph{Base Cases: } 
 \begin{itemize}
 % real number constants
 \item $e \in \{ \texttt{0}, \ \texttt{1}, \ \ldots \}$
 % boolean constants
 \item $e \in \{ \texttt{true}, \ \texttt{false} \}$\\ \\
 %
  In both of these cases, $\llbracket \Gamma \vdash e : \tau \rrbracket$ is a constant function from a $\Gamma$-environment
 $\eta$ to an element of $\llbracket \tau \rrbracket$, so clearly $\tmden{\typing\Gamma e\tau}{\extend\eta x {\bigvee_i a_i}} =
\bigvee_i\tmden{\typing\Gamma e\tau}{\extend\eta x {a_i}}$.

% variable
 \item $e = x : \tau$ \\ \\
 We want to show that $\tmden{\typing\Gamma x\tau}{\extend\eta y {\bigvee_i a_i}} =
\bigvee_i\tmden{\typing\Gamma x\tau}{\extend\eta y {a_i}}$.

In the case that $ y \neq x$, note that
\begin{align*}
\tmden{\typing\Gamma x\tau}{\extend\eta y {\bigvee_i a_i}} &= \eta\{y\mapsto \bigvee_i a_i\}(x) \\
&= \eta (x) \\
&= \bigvee_i\tmden{\typing\Gamma x\tau}{\extend\eta y {a_i}}
\end{align*}

 In the case that $y = x$, note that
  \begin{align*}
  \llbracket \Gamma \vdash x : \tau \rrbracket\eta\{y\mapsto\bigvee_i a_i\} &= \eta\{y\mapsto \bigvee_i a_i\}(x) \\
  &= \bigvee_i a_i \\
  &=\bigvee_i \eta\{y \mapsto a_i\}(x)  \\
  &= \bigvee_i\tmden{\typing\Gamma x\tau}{\extend\eta y {a_i}} \\
  \end{align*}
 \end{itemize}
 \emph{Inductive Cases: }
 \begin{itemize}
 % arithmetic operations
 \item $e = e_1 \circ e_2, \ \circ \in \{+, -, *, / \}$ \\ \\
 Then 
 \begin{align*}
 \llbracket \Gamma \vdash e : \tau \rrbracket\eta\{x \mapsto \bigvee_i a_i\} &= \llbracket \Gamma\vdash e_1 : \tau 
 \rrbracket\eta\ \{x \mapsto \bigvee_i a_i\} \circ \llbracket \Gamma \vdash e_2 : \tau \rrbracket\eta\{x \mapsto \bigvee_i a_i\}\\
  \text{(by inductive hypothesis)} &=  \bigvee_i\llbracket \Gamma \vdash e_1: \tau \rrbracket\eta\{x\mapsto a_i\} \circ
  \bigvee_i\llbracket \Gamma \vdash e_2: \tau \rrbracket\eta\{x\mapsto a_i\}  \\
  &= \bigvee_i(\llbracket \Gamma \vdash e_1: \tau \rrbracket\eta\{x\mapsto a_i\} \circ 
  \llbracket \Gamma \vdash e_2: \tau \rrbracket\eta\{x\mapsto a_i\}) \\
  &= \bigvee_i \llbracket \Gamma \vdash e: \tau \rrbracket\eta\{x\mapsto a_i\}
 \end{align*}

  % equality operation
  \item $e = e_1 \circ e_2, \ \circ \in \{=,<,>,\leq,\geq\}$\\ \\
  In the case that 
 \begin{align*}
  \llbracket \Gamma \vdash e_1 : \tau \rrbracket\eta\{x\mapsto  \bigvee_i a_i\} &= \perp \\ 
  \text{or }  \llbracket \Gamma \vdash e_2 : \tau \rrbracket\eta\{x\mapsto  \bigvee_i a_i\} &= \perp,
  \end{align*} 
  note that $\llbracket \Gamma \vdash e : \texttt{bool}\rrbracket\eta\{x\mapsto  \bigvee_i a_i\} = \perp$. 
  By inductive hypothesis, 
  \begin{align*}
  \llbracket \Gamma \vdash e_1 : \tau \rrbracket\{x\mapsto  \bigvee_i a_i\} &= \bigvee_i\llbracket \Gamma \vdash e_1 : \tau \rrbracket\{x\mapsto a_i\} \\
 \text{and } \llbracket \Gamma \vdash e_2 : \tau \rrbracket\{x\mapsto  \bigvee_i a_i\} &= \bigvee_i\llbracket \Gamma \vdash e_2 : \tau \rrbracket\{x\mapsto a_i\}, 
  \end{align*}
  so assuming WLOG that $\llbracket \Gamma \vdash e_1 : \tau \rrbracket\eta\{x\mapsto  \bigvee_i a_i\} = \perp$, 
  it must be that for \\
  $\forall i, \llbracket \Gamma \vdash e_1 : \tau \rrbracket\eta\{x\mapsto a_i\} = \perp$. 
  Thus, $\bigvee_i\llbracket \Gamma \vdash e : \tau \rrbracket\eta\{x\mapsto a_i\} = \perp$, so \\
  $\bigvee_i\llbracket \Gamma \vdash e : \tau \rrbracket\eta\{x\mapsto a_i\} = 
  \llbracket \Gamma \vdash e_1 : \tau \rrbracket\eta\{x\mapsto  \bigvee_i a_i\}$.\\ \\
  %
  In the case that $\llbracket \Gamma \vdash e_1 : \tau \rrbracket\eta\{x\mapsto  \bigvee_i a_i\} 
  \circ \llbracket \Gamma \vdash e_2 : \tau \rrbracket\eta\{x\mapsto  \bigvee_i a_i\}, \ 
  \llbracket \Gamma \vdash e_1:\tau \rrbracket\eta\{x\mapsto  \bigvee_i a_i\} \neq \perp, 
  \llbracket \Gamma \vdash e_2 : \tau \rrbracket\eta\{x\mapsto  \bigvee_i a_i\} \neq \perp$, 
  note that $\llbracket \Gamma \vdash e : \texttt{bool}\rrbracket\eta\{x\mapsto  \bigvee_i a_i\} = true$. 
 By inductive hypothesis, 
  \begin{align*}
  \llbracket \Gamma \vdash e_1 : \tau \rrbracket\{x\mapsto  \bigvee_i a_i\} &= \bigvee_i\llbracket \Gamma \vdash e_1 : \tau \rrbracket\{x\mapsto a_i\} \\
 \text{and } \llbracket \Gamma \vdash e_2 : \tau \rrbracket\{x\mapsto  \bigvee_i a_i\} &= \bigvee_i\llbracket \Gamma \vdash e_2 : \tau \rrbracket\{x\mapsto a_i\}, 
  \end{align*}
  so $\bigvee_i\llbracket \Gamma \vdash e_1 : \tau \rrbracket\{x\mapsto a_i\} 
  \circ \bigvee_i\llbracket \Gamma \vdash e_2 : \tau \rrbracket\{x\mapsto a_i\}$, with neither supremum equal to bottom. Thus there must $\exists n \in \mathbb{N} \text{ s.t. } \forall i \geq n, \llbracket \Gamma \vdash e_1 : \tau \rrbracket\eta\{x\mapsto   a_i\} \circ
  \llbracket \Gamma \vdash e_2 : \tau \rrbracket\eta\{x\mapsto   a_i\}$. Therefore, $\bigvee_i\llbracket \Gamma \vdash e : \tau \rrbracket\eta\{x\mapsto  a_i\}\} = true$, so $\llbracket \Gamma \vdash e : \texttt{bool}\rrbracket\eta\{x\mapsto \bigvee_i a_i\} 
  = \bigvee_i\llbracket \Gamma \vdash e : \tau \rrbracket\eta\{x\mapsto  a_i\}$. \\ \\
  %
 In the case that $\llbracket \Gamma \vdash e_1 : \tau \rrbracket\eta\{x\mapsto  \bigvee_i a_i\} \ \slashed{\circ} \ \llbracket \Gamma \vdash e_2 : \tau\rrbracket\eta\{x\mapsto  \bigvee_i a_i\}, \ \\
 \llbracket \Gamma \vdash e_1 : \tau \rrbracket\eta\{x\mapsto  \bigvee_i a_i\} \neq \perp, 
 \llbracket \Gamma \vdash e_2 : \tau \rrbracket\eta\{x\mapsto  \bigvee_i a_i\} \neq \perp$, by reasoning similar to the previous case see that $\llbracket \Gamma \vdash e : \tau \rrbracket\eta\{x\mapsto  \bigvee_i a_i\} = 
 \bigvee_i\llbracket \Gamma \vdash e : \tau \rrbracket\eta\{x\mapsto  a_i\}$.
  % conditional
 \item $e = \ \ifelse{e_1}{e_2}{e_3} : \tau$ \\ \\
In the case that $\llbracket \Gamma \vdash e_1 : \texttt{bool} \rrbracket\eta\{x\mapsto  \bigvee_i a_i\} = \perp$, note that $\llbracket \Gamma \vdash e : 
\tau \rrbracket\eta\{x\mapsto  \bigvee_i a_i\} =  \perp$. By inductive hypothesis, 
$\llbracket \Gamma \vdash e_1 : \texttt{bool} \rrbracket\eta\{x\mapsto  \bigvee_i a_i\} = 
\bigvee_i\llbracket \Gamma \vdash e_1 : \tau \rrbracket\eta\{x\mapsto a_i\} = \perp$. 
Then it must be that for 
$\forall i, \llbracket \Gamma \vdash e_1 : \tau \rrbracket\eta\{x\mapsto a_i\} = \perp$. 
Thus, $\bigvee_i\llbracket \Gamma \vdash e : \tau \rrbracket\eta\{x\mapsto  a_i\} = \perp$, so 
$\llbracket \Gamma \vdash e : \tau \rrbracket\eta\{x\mapsto  \bigvee_i a_i\} = 
\bigvee_i\llbracket \Gamma \vdash e : \tau \rrbracket\eta\{x\mapsto  \bigvee_i a_i\}$. \\ \\
%
In the case that $\llbracket \Gamma \vdash e_1 : \texttt{bool} \rrbracket\eta\{x\mapsto  \bigvee_i a_i\} = true$, 
note that $\llbracket \Gamma \vdash e : 
\tau \rrbracket\eta\{x\mapsto  \bigvee_i a_i\} =  \llbracket \Gamma \vdash e_2 \rrbracket\eta\{x\mapsto  \bigvee_i a_i\}$. 
By inductive hypothesis, 
$\llbracket \Gamma \vdash e_1 : \texttt{bool} \rrbracket\eta\{x\mapsto  \bigvee_i a_i\} = 
\bigvee_i\llbracket \Gamma \vdash e_1 : \tau \rrbracket\eta\{x\mapsto a_i\} = true$. 
Since $\{a_i\}^{\infty}_{i=1}$ is a chain 
and $\llbracket\texttt{bool}\rrbracket$ is a flat CPO, there must exist $n \in \mathbb{N}$ such that 
$\forall i > n, \llbracket \Gamma \vdash e_1 : \texttt{bool}
\rrbracket\eta\{x\mapsto a_i\} = true$ and 
$\forall i \leq n, \llbracket \Gamma \vdash e_1 : \texttt{bool}\rrbracket\eta\{x\mapsto  a_i\} = \perp$. Thus, it must 
be that $\bigvee_i \llbracket \Gamma \vdash e : \tau \rrbracket\eta\{x\mapsto a_i\} = 
\llbracket \Gamma \vdash e_2 \rrbracket\eta\{x\mapsto  \bigvee_i a_i\} = 
\llbracket \Gamma \vdash e : \tau \rrbracket\eta\{x\mapsto  \bigvee_i a_i\}$. \\ \\
%
In the case that $\llbracket \Gamma \vdash e_1 : \texttt{bool} \rrbracket\eta\{x\mapsto  \bigvee_i a_i\} = false$, by reasoning parallel to the previous case $\bigvee_i \llbracket \Gamma \vdash e : \tau \rrbracket\eta\{x\mapsto a_i\}  = 
\llbracket \Gamma \vdash e : \tau \rrbracket\eta\{x\mapsto  \bigvee_i a_i\}$.
% application
 \item $ e = e_1 \ e_2$ \\ \\
Then
\begin{align*}
\llbracket \Gamma \vdash e : \tau \rrbracket\eta\{x \mapsto \bigvee_i a_i\} &=
\llbracket \Gamma \vdash e_1 : \sigma \rightarrow \tau \rrbracket\eta\{x \mapsto \bigvee_i a_i\}
(\llbracket \Gamma \vdash e_2 : \sigma \rrbracket\eta\{x \mapsto \bigvee_i a_i\}) \\
&= \llbracket \Gamma \vdash \lambda (y:\sigma).e' : \tau \rrbracket\eta\{x \mapsto \bigvee_i a_i\}
(\llbracket \Gamma \vdash e_2 : \sigma \rrbracket\eta\{x \mapsto \bigvee_i a_i\}) \\
\text{(by inductive hypothesis)} &= 
\llbracket \Gamma \vdash \lambda (y:\sigma).e' : \tau \rrbracket\eta\{x \mapsto \bigvee_i a_i\}
(\bigvee_i \llbracket \Gamma \vdash e_2 : \sigma \rrbracket\eta\{x \mapsto a_i\}) \\
&= \llbracket \Gamma. y : \sigma \vdash e' : \tau' \rrbracket\eta\{x \mapsto \bigvee_i a_i\}\{y \mapsto (\bigvee_i \llbracket 
\Gamma \vdash e_2 : \sigma \rrbracket\eta\{x \mapsto a_i\})\} \\
\text{(by inductive hypothesis)}&= \bigvee_i \llbracket \Gamma. y : \sigma \vdash e' : \tau' \rrbracket\eta\{x \mapsto 
a_i\}\{y \mapsto (\llbracket \Gamma \vdash e_2 : \sigma \rrbracket\eta\{x \mapsto a_i\})\} \\
&=\bigvee_i \llbracket \Gamma \vdash e_1: \sigma \rightarrow \tau\rrbracket\eta\{x\mapsto a_i\}
(\llbracket\Gamma\vdash e_2 : \sigma\rrbracket\eta\{x\mapsto a_i\}) \\
&= \bigvee_i \llbracket \Gamma \vdash e:\tau\rrbracket\eta\{ x \mapsto a_i\} \\
\end{align*}
 % abstraction
 \item $e = \lambda (y : \tau) . (e' : \tau')$
 \begin{enumerate}
 \item Then $\llbracket \Gamma \vdash e : \tau \rightarrow \tau' \rrbracket\eta\{x\mapsto  \bigvee_i a_i\}$ i
 s a function $f: \llbracket \tau \rrbracket \rightarrow \llbracket \tau'  \rrbracket$ such that for 
 $\forall d \in \llbracket \tau \rrbracket, \ f(d) = \llbracket \Gamma y : \tau \vdash e' : \tau 
 \rrbracket\eta\{x\mapsto  \bigvee_i a_i\}\{y \mapsto d\}$. We want to show that
 $\tmden{\typing\Gamma e\tau}{\extend\eta x {\bigvee_i a_i}} =
\bigvee_i\tmden{\typing\Gamma e\tau}{\extend\eta x {a_i}}$
 
Let $d \in \llbracket \tau \rrbracket$, and consider that
\begin{align*}
\llbracket \Gamma \vdash e : \tau \rightarrow \tau' \rrbracket\eta\{x\mapsto  \bigvee_i a_i\}(d) 
&= \llbracket \Gamma.y : \tau \vdash e' : \tau'\rrbracket\eta\{x\mapsto  \bigvee_i a_i\}
\{y \mapsto d\} \\
&= \llbracket \Gamma.y : \tau \vdash e' : \tau'\rrbracket(\eta\{y \mapsto d\})\{x\mapsto  \bigvee_i a_i\} \\
\text{(by inductive hypothesis)}&=\bigvee_i\llbracket \Gamma.y:\tau\vdash e':\tau'\rrbracket(\eta\{y\mapsto d\})
\{x\mapsto a_i\} \\
&=\bigvee_i\llbracket \Gamma.y:\tau\vdash e':\tau'\rrbracket(\eta\{x\mapsto a_i\})\{y\mapsto d\} \\
&= \bigvee_i \llbracket \Gamma \vdash \lambda y : \tau.e' : \tau' \rrbracket \eta\{x\mapsto a_i\}(d) \\
&= \bigvee_i \llbracket \Gamma \vdash e : \tau \rightarrow \tau' \rrbracket \eta\{x\mapsto a_i\}(d) \\ 
\end{align*}
%
 \item We want to show that $\llbracket \Gamma \vdash e : \tau \rightarrow \tau' \rrbracket\eta \in \llbracket \tau \rightarrow 
 \tau'\rrbracket$---that is, it is a continuous function from $\llbracket \tau \rrbracket$ to $\llbracket \tau' \rrbracket$. Let 
 $\{a_i\}^{\infty}_{i=1}$ be a chain of elements in $\llbracket \tau \rrbracket$, and see that
 \begin{align*}
 \llbracket \Gamma \vdash e : \tau \rightarrow \tau' \rrbracket\eta(\bigvee_i a_i) &= \llbracket \Gamma.x : \tau 
 \vdash e' : \tau'\rrbracket\eta \{x \mapsto \bigvee_i a_i\} \\
 \text{(by 1)} &= \bigvee_i \llbracket \Gamma.x : \tau \vdash e' : \tau'\rrbracket\eta\{x\mapsto a_i\}\\
 &= \bigvee_i \llbracket \Gamma \vdash e : \tau \rightarrow \tau' \rrbracket\eta(a_i) 
 \end{align*}
 \end{enumerate} 
 % fix operation
 \item $e = \texttt{fix} (\lambda f.\lambda x.e)$ \\
 \begin{enumerate}
\item Let $F_i = \llbracket \Gamma \vdash \lambda f. \lambda x.e\rrbracket\eta\{x\mapsto a_i\}$, and consider that 
 \begin{align*}
 \llbracket \texttt{fix} (\lambda f.\lambda x.e)\rrbracket\eta\{x\mapsto \bigvee_i a_i\} &= 
 fix(\llbracket \lambda f. \lambda x.e \rrbracket\eta\{x\mapsto \bigvee_i a_i\})\\
 \text{(by inductive hypothesis)} &= fix(\bigvee\{F_i\}^{\infty}_{i=1}) \\
 &= \bigvee\{\bigvee\{(F_i)^j \}^{\infty}_{i=1}\perp\}^{\infty}_{j=1} \\
 &= \bigvee\{\bigvee\{(F_i)^j \perp\}^{\infty}_{i=1}\}^{\infty}_{j=1} \\ 
\text{(by \ref{lem1})} &= \bigvee\{\bigvee\{(F_i)^j \perp\}^{\infty}_{j=1}\}^{\infty}_{i=1} \\ 
&= \bigvee\{fix(F_i)\}^{\infty}_{i=1}\\
&= \bigvee_i \llbracket \Gamma \vdash \texttt{fix}(\lambda f \lambda x.e)\rrbracket\eta\{x\mapsto a_i\}
 \end{align*}
 %
 \item Let $F = \llbracket \Gamma \vdash \lambda f. \lambda x.e\rrbracket\eta$. We want to show that 
 $\llbracket\Gamma\vdash\texttt{fix}(\lambda f \lambda x.e) : \tau 
 \rightarrow\tau\rrbracket\eta \in \llbracket \tau \rightarrow \tau \rrbracket$---that is, it is a continuous function from 
 $\llbracket \tau \rrbracket$ to $\llbracket \tau \rrbracket$. Let $\{a_i\}^{\infty}_{i=1}$ be a chain of elements in $\llbracket \tau \rrbracket$,
  and see that 
 \begin{align*}
 \llbracket\Gamma\vdash\texttt{fix}(\lambda f \lambda x.e) : \tau \rightarrow\tau\rrbracket\eta \bigvee_i a_i &= 
 fix(F)\bigvee_i a_i\\
 &=\bigvee\{F^j \perp\}^{\infty}_{j=1}\bigvee_i a_i\\
 &= \bigvee\{F^j \perp(\bigvee_i a_i)\}^{\infty}_{j=1}\\ 
 \text{(by continuity of $F^j \perp$) }&= \bigvee\{\bigvee\{F^j \perp(a_i)\}^{\infty}_{i=1}\}^{\infty}_{j=1}\\ 
 \text{(by \ref{lem2})} &=\bigvee\{\bigvee\{F^j \perp(a_i)\}^{\infty}_{j=1}\}^{\infty}_{i=1}\\
 &= \bigvee\{fix(F)(a_i)\}^{\infty}_{i=1}\\
 &= \bigvee\{\llbracket\Gamma\vdash\texttt{fix}(\lambda f \lambda x.e) : \tau \rightarrow\tau\rrbracket\eta(a_i)\}^{\infty}_{i=1}\\
 \end{align*}
 \end{enumerate}
 \end{itemize}
 \end{proof}
 % Lemmas
 \begin{lemma} $\bigvee\{\bigvee\{(F_i)^j \perp\}^{\infty}_{i=1}\}^{\infty}_{j=1}= \bigvee\{\bigvee\{(F_i)^j \perp\}^{\infty}
 \label{lem1}
 _{j=1}\}^{\infty}_{i=1}$\\
 \end{lemma}
 \begin{proof}
 $\Rightarrow$ We want to show that\\ \\ \centerline{$\bigvee\{\bigvee\{(F_i)^j \perp\}^{\infty}_{i=1}\}^{\infty}
 _{j=1} \geq  \bigvee\{\bigvee\{(F_i)^j \perp\}^{\infty}_{j=1}\}^{\infty}_{i=1}$} \\ \\ First, notice that $\forall i, j \bigvee\{\bigvee
 \{(F_i)^j \perp\}^{\infty}_{i=1}\}^{\infty}_{j=1} \geq (F_i)^j$. \\
 Suppose, now, that 
 \begin{align*}
 &\bigvee\{\bigvee\{(F_i)^j \perp\}^{\infty}_{i=1}\}^{\infty}_{j=1} < \bigvee\{\bigvee\{(F_i)^j \perp\}^{\infty}_{j=1}\}^{\infty}_{i=1}\\ 
 \Rightarrow \exists i \text{ s.t. \ \  \ } &\bigvee\{\bigvee\{(F_i)^j \perp\}^{\infty}_{i=1}\}^{\infty}_{j=1} < \bigvee\{(F_i)^j \perp\}
 ^{\infty}_{j=1}\\
 \Rightarrow \exists i, j \text{ s.t. } &\bigvee\{\bigvee\{(F_i)^j \perp\}^{\infty}_{i=1}\}^{\infty}_{j=1} < (F_i)^j \perp 
 \Rightarrow\Leftarrow\\
 \end{align*}
 Thus, it must be that $\bigvee\{\bigvee\{(F_i)^j \perp\}^{\infty}_{i=1}\}^{\infty}_{j=1} \geq  \bigvee\{\bigvee\{(F_i)^j \perp\}^{\infty}
 _{j=1}\}^{\infty}_{i=1}$ \\ \\
 $\Leftarrow$ By parallel reasoning, $\bigvee\{\bigvee\{(F_i)^j \perp\}^{\infty}_{i=1}\}^{\infty}_{j=1} \leq  \bigvee\{\bigvee\{(F_i)^j 
 \perp\}^{\infty}_{j=1}\}^{\infty}_{i=1}$ \\ \\
 Therefore, $\bigvee\{\bigvee\{(F_i)^j \perp\}^{\infty}_{i=1}\}^{\infty}_{j=1} = \bigvee\{\bigvee\{(F_i)^j \perp\}^{\infty}_{j=1}\}
 ^{\infty}_{i=1}$ \\ \\
 \end{proof}
 %
 \begin{lemma}
 \label{lem2}
 $\bigvee\{\bigvee\{F^j \perp(x_i)\}^{\infty}_{i=1}\}^{\infty}_{j=1} = \bigvee\{\bigvee\{F^j \perp(x_i)\}^{\infty}
 _{j=1}\}^{\infty}_{i=1}$\\ 
 \end{lemma}
 \begin{proof} 
 By reasoning similar to \ref{lem1}. \\ \\
 \end{proof}
 
\section{Examples}
Consider that we may express the general equation $T(x) = a(x) + T(m(x))$ in our expression language as follows:
 \begin{align*}
 \texttt{T} = \texttt{fix}(\lambda f. \lambda x.\texttt{a}(x) + f (\texttt{m}(x)))
 \end{align*}
 We want to show that the denotation of this expression has the same behavior as this deterministic function. First, let 
 $\texttt{F} = \lambda f. \lambda x.\texttt{a}(x) + f (\texttt{m}(x)$, and note the following:
 \begin{align*}
 \llbracket \texttt{T} \rrbracket &= fix(\llbracket \texttt{F} \rrbracket) \\
 \llbracket \texttt{F} \rrbracket &= F : \llbracket \texttt{real} \rightarrow \texttt{real}\rrbracket \rightarrow \llbracket \texttt{real} \rightarrow 
 \texttt{real}\rrbracket \\ &\text{ \ \ \ s.t. } \forall \alpha \in \llbracket \texttt{real}\rrbracket \rightarrow \llbracket \texttt{real}
 \rrbracket, \ 
 F(\alpha) = \llbracket \lambda x.a + f (p * x)\rrbracket\{f \mapsto \alpha\} \\
 \llbracket \texttt{real}\rrbracket &= \R_{[0, +\infty]}  
 \\\text{By the CPO fixpoint theorem, } fix(\llbracket \texttt{F} \rrbracket) &= fix(F) \\
&= \bigvee\{F^n (\perp)\}^{\infty}_{n=1} \\
&= \lim_{n \to \infty}\{F^n (\perp)\}
 \end{align*}
 Recall that the bottom element of $\llbracket \texttt{real} \rrbracket$ is $0$, so the bottom element of $\llbracket \texttt{real} 
 \rightarrow \texttt{real}\rrbracket$ is the function $\perp$ that maps every element in $\llbracket \texttt{real} \rrbracket$ to 
 $0$.  \\
 %
 \begin{lemma}
 \label{lem2}
 \begin{align*}
 \forall d \in \R_{[0, +\infty]}, \ F^n(\perp)(d) = \sum_{i = 0}^{n-1}a(m^{i}(d))
 \end{align*}
 \end{lemma}
 %
 \begin{proof} by induction on n. \\
 \emph{Base case:} n = 1
 \begin{align*}
 \forall d \in \extR, \ F(\perp)(d) &= \llbracket \lambda x.\texttt{a}(x) + f (\texttt{m}(x))\rrbracket\{f \mapsto \perp, x \mapsto d\} \\
 &= a(d) + \perp(m(d)) \\
 &= a(d)  \\
 \end{align*}
 \emph{Inductive case:} Suppose the claim hold for $\forall k < n$.
 \begin{align*}
 \forall d \in \extR, \ F^n(\perp)(d) &= \llbracket \lambda x.\texttt{a}(x) + f (\texttt{m}(x))\rrbracket
 \{f \mapsto F^{n-1}(\perp),x\mapsto d\} \\
 &= a(d) + F^{n-1}(\perp)(m(d)) \\
\text{(by inductive hypothesis)} &= a(d) 
					    + \sum_{i = 0}^{n-2}a(m^{i}(m(d))) \\
&= a(d) + \sum_{i = 0}^{n-2} a( m^{i+1}d) \\
&= \sum_{i=0}^{n-1} a(m^{i}d) \\
\end{align*}
\end{proof}
Therefore, 
\begin{thm}
\begin{align*}
\forall d \in \extR, \ \llbracket \texttt{T} \rrbracket\{x \mapsto d\} &= \lim_{n \to \infty}\{F^n (\perp)\}(d) \\
&= \sum_{i=0}^{\infty}a(m^{i}(d)) \\
 \end{align*}
 \end{thm}
 
\subsection{Example 1}
Consider the simple example Karp offers, wherein $m(x) = px$, with $p$ a positive constant less than 1, and $a(x) = 0, \ x < 1, 
\ a(x) = 1, \ x \geq 1$. Then $T(x) = a(x) + T(px)$, so by a simple inductive argument we can show that 
\begin{align*}
T(x) &= 
 \begin{cases}
 0 \text{ if } x < 1 \\
 k \text{ if }  \dfrac{1}{p^{k-1}} \leq x < \dfrac{1}{p^k} \\
 \end{cases}
 \end{align*}
We may express this function in our expression language using the expression $\texttt{T}$ defined above, with 
$\texttt{a}(x) = \ifelse{x<1}{0}{1}$ and $\texttt{m}(x) = p*x$:
 \begin{align*}
 \texttt{fix}(\lambda f. \lambda x. (\ifelse{x<1}{0}{1}) + f (p * x))
 \end{align*}
 Then, by theorem 6,
 \begin{align*}
 \forall d \in \extR, \llbracket \texttt{T} \rrbracket\{x \mapsto d\} = \sum_{i=0}^{\infty}a(p^{i}d)
 \end{align*}
We must verify that $\llbracket \texttt{T} \rrbracket$ has the same behavior as the function $T$ described by Karp.
Note that for $\forall d \in \extR, \ \exists k \in \N$  s.t. $\dfrac{1}{p^{k-1}} \leq d < \dfrac{1}{p^k}$, so $T(d) = k$. Also, 
for $\forall i < k, \ p^id \geq 1$, and for $\forall i \geq k, p^id < 1$. Note, then, that
\begin{align*}
\llbracket \texttt{T} \rrbracket\{x\mapsto d\} &= \sum_{i=0}^{k} a(p^{i}d) + \sum_{i=k+1}^{\infty} a(p^{i}d) \\
&= \sum_{i=0}^{k} 1 + \sum_{i=k+1}^{\infty} 0\\
&= k \\
&= T(d) \\
\end{align*}
Thus, $\llbracket \texttt{T} \rrbracket = T$.

\subsection{Example 2}
Let $m(x) = px$, where $p$ is a positive constant less than 1, and $a(x) = x$. Then $T(x) = x + T(px)$, so, for instance,
\begin{align*}
T(1) = 1 + p + p^2 + p^3 + \ldots
\end{align*}
See, then, that this function describes the following geometric series:
\begin{align*}
T(x) &= \sum_{i=0}^{\infty} p^ix \\ 
&= \frac{x}{1 - p} \ (\text{since } 0 \leq p \leq 1)
\end{align*}
We may express this function in our expression language using the expression $\texttt{T}$ defined above, with 
$\texttt{a}(x) = x$ and $\texttt{m}(x) = p*x$:
\begin{align*}
 \texttt{T} = \texttt{fix}(\lambda f. \lambda x.x + f (p * x))
 \end{align*}
Then, by theorem 6,
 \begin{align*}
 \forall d \in \extR, \llbracket \texttt{T} \rrbracket\{x \mapsto d\} = \sum_{i=0}^{\infty}p^{i}d
 \end{align*}
See, then, that for $\forall d \in \extR, \ T(d) = \sum_{i=0}^{\infty}p^{i}d = \llbracket \texttt{T} \rrbracket \{x \mapsto d\}$.
Thus, $\llbracket \texttt{T} \rrbracket$ has the same behavior as the function $T$ described by Karp.
 
 \subsection{Example 3} Let $m(x) = x$ and $a(x) = x$. Then $m(x), a(x)$, and $m(x)/x$ are all non-decreasing functions and 
 thus meet the assumptions specified by Karp. Note that $T(x) = x + T(x)$, so, for instance,
\begin{align*}
T(1) = 1 + 1 + 1 + \ldots
\end{align*}
See, then, that this function describes the following series:
\begin{align*}
T(x) &= \sum_{i=0}^{\infty} x 
\end{align*}
Clearly this series diverges for $\forall x \in \R$. We may express this function in our expression language using the 
expression $\texttt{T}$ defined above, with $\texttt{a}(x) = x$ and $\texttt{m}(x) = x$:
\begin{align*}
 \texttt{T} = \texttt{fix}(\lambda f. \lambda x.x + f (x))
 \end{align*}
Then, by theorem 6,
 \begin{align*}
 \forall d \in \extR, \llbracket \texttt{T} \rrbracket\{x \mapsto d\} = \sum_{i=0}^{\infty}d
 \end{align*}
See, then, that for $\forall d \in \extR, \ T(d) = \sum_{i=0}^{\infty}d = \llbracket \texttt{T} \rrbracket\{x \mapsto d\}$.
Thus, $\llbracket \texttt{T} \rrbracket$ has the same behavior as the function $T$ described by Karp.



\chapter{Reconciling the Two Semantics}
Chapters 2 and 3 have presented languages for defining the two distinct kinds of deterministic recurrences present in 
\cite{Karp}. In this chapter, we look at several attempts to connect these two recurrences. In particular, we want to find a 
way to write a recurrence with an initial condition as a recurrence of the form $T(x) = a(x) + T(m(x))$, as this is the kind
of recurrence for which Karp proves tail bounds. We will offer
several methods of translating from one expression language to the other, and investigate the validity of these 
approaches. Our goal here is to show that, for every expression of the form $\texttt{fix}(\lambda f.\lambda x.e)$ in our
 chapter 3 language, there are $a, m$ in our chapter 2 language such that 
 $\llbracket \texttt{rec}(a,m)\rrbracket_{sum} = \llbracket \texttt{fix}(\lambda f.\lambda x.e)\rrbracket_{fix}$. We use
 $\llbracket \cdot\rrbracket_{sum}$ and $\llbracket \cdot\rrbracket_{rec}$ to denote the semantics of chapters 2 and
 3, respectively. We will see that this will only be possible for $e$ of a certain form, but that this form covers all the 
 examples that Karp provides.

As a first attempt, we consider a simple recurrence with an initial condition, as described in chapter 3.
\begin{align*}
T(1) &= 0 \\
T(n) &= 1 + T(n/2) 
\end{align*}
As in the example in chapter 3, we assume that the actual meaning of the general recurrence
is $T(n) = 1 + T(\lfloor n/2 \rfloor)$.
We can write this recurrence as an expression \texttt{T} in our fixed-point language, where
\begin{align*}
\llbracket \texttt{T}\rrbracket_{fix}(n) = \lfloor \log_2(n) \rfloor.
\end{align*} 

 In order to make this fit the definition of a recurrence described in 
chapter 2 --- that is, a recurrence of the form $T(n) = a(n) + T(m(n))$--- we may interpret $a$ as a 
piecewise function with two sub-functions: one for the general recurrence and one for the initial condition.
We write this as a single recurrence
\begin{align*}
T'(n) &= a(n) + T'(m(n)) \\
m(n) &= \lfloor n/2 \rfloor, \ \forall n \\ 
a(n) &=
\begin{cases} 
0, \ n \leq 1 \\
1, \ n > 1
\end{cases}
\end{align*} 
This recurrence can then be expressed in our infinite sum language as 
\begin{align*}
\texttt{T} &= \texttt{rec}(a,m) \\
m &= \lambda x.\texttt{floor}(x/2) \\
a &= \lambda x.\ifelse{x\leq 1}{0}{1} \\
\llbracket \texttt{T} \rrbracket(n) &= \llbracket \texttt{rec}(a,m) \rrbracket(n) \\
&= \sum_{i=0}^{\infty}\llbracket a \rrbracket(\llbracket m\rrbracket^i(n)) \\
&= \sum_{i=0}^{\lfloor \log_2(n) \rfloor} 1 + \sum_{i= \lfloor \log(n)\rfloor+1}^{\infty} 0 \\
&=\lfloor \log_2(n) \rfloor.
\end{align*}

However, this approach will not work in every case. Let us try to use the same process on the following recurrence
\begin{align*}
T(1) &= 1 \\
T(n) &= 2 + T(\lfloor n/2\rfloor) 
\end{align*}
We first write this as a single recurrence. See that, unlike the previous example, $a$ never maps to $0$.
\begin{align*}
T'(n) &= a(n) + T'(m(n)) \\
m(n) &= \lfloor n/2 \rfloor, \ \forall n \\ 
a(n) &= 
\begin{cases}
1, \ n \leq 1 \\
2, \ n > 1
\end{cases}
\end{align*}
We then express this recurrence using our infinite sum language as
\begin{align*}
\texttt{T} &= \texttt{rec}(a,m) \\
m &= \lambda x.\texttt{floor}(x/2) \\
a &=\lambda x.\ifelse{x\leq1}{1}{2} \\
\llbracket \texttt{T} \rrbracket(d) &= \llbracket \texttt{rec}(a,m) \rrbracket(d) \\
&= \sum_{i=0}^{\infty}\llbracket a \rrbracket(\llbracket m\rrbracket^i(d)) \\
&= \sum_{i=0}^{\lfloor \log(n) \rfloor} 2 + \sum_{i=\lfloor \log(n) \rfloor+1 }^{\infty} 1 \\
&= \infty
\end{align*}
Which, clearly, is not the same as the fixed-point semantics for $T$.

A second attempt takes a similar but slightly altered approach. Suppose that we interpret $a$ as a piecewise function with three sub-functions:
one for the general recurrence, one for the initial condition, and one for values smaller than the initial condition, which all
map to $0$. Considering the example that failed in the previous attempt, we would have
\begin{align*}
T'(n) &= a(n) + T'(m(n)) \\
m &= \lfloor n/2 \rfloor, \forall n \\
a(n) &=
\begin{cases}
0, \ n < 1 \\
1, \ n = 1 \\
2, \ n > 1
\end{cases}
\end{align*}
We can express this recurrence using our infinite sum language as
\begin{align*}
\texttt{T} &= \texttt{rec}(a,m) \\
m &= \lambda x.\texttt{floor}(x/2) \\
a &= \lambda x.\ifelse{x <1}{0}{(\ifelse{x = 1}{1}{2})}.
\end{align*}
See, then, that
\begin{align*}
\llbracket a \rrbracket(n) &= 
 \begin{cases}
0, \ n < 1 \\
1, \ n = 1 \\
2, \ n > 1 
\end{cases} 
\end{align*}
\begin{align*}
\llbracket \texttt{T} \rrbracket(n) &= \llbracket \texttt{rec}(a,m)\rrbracket(n) \\
&= \sum_{i=0}^{\infty}\llbracket a \rrbracket(\llbracket m\rrbracket^i(n)) \\
&= \sum_{i=0}^{\lfloor \log(n) \rfloor} 2 + \sum_{i=\lfloor \log(n) \rfloor+1 }^{\lfloor \log(n) \rfloor+1} 1 + \sum_{i =\lfloor \log(n) \rfloor + 2}^{\infty} 0 \\
&= \lfloor 2\log(n) \rfloor + 1
\end{align*}

This approach matches our intuition for how an initial condition should work---that once you reach the initial condition,
you stop recursing. 

With this in mind, we can identify a general strategy for extracting a recurrence in our chapter 2 syntax (with
the $\texttt{rec}$ operator), given a recurrence in our chapter 3 syntax (with the $\texttt{fix}$ operator). 
What we want to show is that for any $\texttt{fix}(\lambda f. \lambda x.e)$ expression, there are expressions $a$ and $m$
in our infinite sum language such that $\llbracket \texttt{fix}(\lambda f.\lambda x.e)\rrbracket_{fix} = 
\llbracket \texttt{rec}(a,m)\rrbracket_{sum}$. But we cannot prove this exact theorem, as not every
expression $e$ is of a form that corresponds to a recurrence with an initial condition.

What we can do instead is define a sublanguage of our fixed-point language such that, for any expression $E$ in this 
sublanguage, we may translate $E$ to an expression $E^*$ in our infinite sum language. We may then prove
that for any expression $E$, $\llbracket E \rrbracket_{fix}$ is equivalent to $\llbracket E^* \rrbracket$. However,
there are some underlying complexities in our semantics that make this proof of equivalence non-trivial; formalizing
this translation and proof is left for future work.

Still, we can look at an example of what it might look like to make a translation from a $\texttt{fix}$ expression corresponding to a chapter 3 recurrence to a $\texttt{rec}$ expression corresponding to a chapter 2 recurrence.
Suppose we have a recurrence with an initial condition, given as follows:
\begin{align*}
T(b) &= c, \text{ where } b, c  \in \R, b > 0 \\
T(x) &= a(x) + T( m(x) ).
\end{align*}
We write this recurrence in our chapter 3 syntax as 
\begin{align*}
\texttt{fix}(\lambda f.\lambda x. \ifelse{x = b}{c}{a(x) + f((m(x)))}),
\end{align*}
an expression whose denotation will be a function $F$ such that
\begin{align*}
F(x) &=
\begin{cases}
c \text{ if } x  = b \\
a(x) + c \text{ if } m(x)  = b \\
a(x) + a(m(x)) + c \text{ if } m(m(x))  = b \\
\ldots \\
\sum_{i=1}^{n} a(m^{i-1}(x)) + c \text{ if } m^n(x) = b \\
\ldots
\end{cases}
\end{align*}

We may then write this recurrence in our chapter 2 syntax as $\texttt{rec}(a',m)$, where $m$ is the same function
as in the previous recurrence, and
\begin{align*}
a' = \lambda x.\lambda f. \ifelse{x < b}{0}{(\ifelse{x = b}{c}{a(x)})}.
\end{align*}
Then the denotation of $\texttt{rec}(a',m)$ is a function $F'$ such that for all $x$ in the domain of $F$, 
\begin{align*}
F'(x) &= \sum_{i=0}^{\infty} \llbracket a' \rrbracket(\llbracket m \rrbracket^i(x)) \\
&= \sum_{i=0}^{n-1} \llbracket a' \rrbracket( \llbracket m \rrbracket^i(x)) + \sum_{i=n}^n c + \sum_{i = n+1}^{\infty} 0 \\
&\text{ where } \llbracket m \rrbracket^n(x) = b \\
&= \sum_{i=1}^{n}\llbracket a' \rrbracket ( \llbracket m \rrbracket^{i-1}(x)) + c
\end{align*}
See, then, that $F' = F$. Therefore, this confirms that, given an expression $e$ of the specific form
$e = \ifelse{x = m}{c}{a(x) + f(m(x))}$, we can extract a function $a'$ such that 
\begin{align*}
\llbracket \texttt{fix}(\lambda f. \lambda x.e) \rrbracket_{fix} = \llbracket \texttt{rec}(a',m)\rrbracket_{sum}
\end{align*}





 
\chapter{Well-Typed Probabilistic Recurrence Relations}
\cite{Karp} is premised on the notion that the cost of a stochastic algorithm may be described as a
recurrence relation of the form
\begin{align*}
T(x) = a(x) + T(h_1(x)) + \dots + T(h_n(x)).
\end{align*}
But what does this equation actually mean? To unpack this, we will attempt to assign types to this recurrence.

 Following Karp's definitions, we can immediately assign types to some of these terms:
\begin{align*}
T(x) &: \Omega_x \rightarrow \R \text{, where } \Omega_x \text{ is the sample space of all problem instances of size } x \\
T &: \prod_{x \in \R} (\Omega_x \rightarrow \R) \\ 
a &: \R \rightarrow \R
\end{align*}

However, for other terms it is less clear what the type should should be. Karp alternates between saying that $h$ is a random 
variable and $h(x)$ is a random variable, so it is not immediately obvious 
what the meaning of this function is. However, since $h$ takes $x \in \R$ as an input, it would not make sense for $h$ to be a 
random variable of type $\R \rightarrow \R$. This would make $h$ a function that takes an input size and returns the 
size of the derived subproblem---thus sidestepping the stochastic part of the algorithm. Then it must be that $h(x)$ is a 
random variable on \emph{specific input} of size $x$---i.e., the sample space $\Omega_x$ and returns the size of the derived 
subproblem.
\begin{align*}
h_i(x) &: \Omega_x \rightarrow \R \\
h_i &: \prod_{x \in \R}(\Omega_x \rightarrow \R) 
\end{align*} 

There are some inconsistencies in these type assignments. To start, $T$ is a function that takes a real number and 
returns a random variable. But on the right-hand side of the relation, $T$ takes $h_i(x)$ as an argument; $h_i(x)$ is
not a real number, but a random variable. 

Thinking in terms of what we \emph{want} this recurrence to express, the arguments to the $T$'s on the righthand side of
the equation should be the sizes of the derived subproblems---that is, the result of applying $h_i(x)$ to the original input.
Then, given an input $l \in \Omega_x$, we should have
\begin{align*}
T(x)(l) &= a(x) + T(h_1(x)(l)) \dots + T(h_n(x)(l)) \\
&\text{where } h(x)(l) : \R 
\end{align*}
 Thus, the term $T(h_i(x)(l))$ typechecks. However, we now run into another problem: $T(x)(l)$ and $a(x)$ are real numbers,
but $T(h_i(x)(l))$ is a random variable of type $\Omega_{h_i(x)(l)} \rightarrow \R$. Then $T(h_i(x)(l))$ needs an argument.
Consider that $T(h_i(x)(l))$ is a random variable describing the running time of the algorithm with an input of size 
$h_i(x)(l)$. For instance, given a recurrence for randomized quicksort, $T(h_1(x)(l))$ and $T(h_2(x)(l))$ describe how long it 
takes to quicksort the left and right sublists of an input list $l$, which themselves have length $h_1(x)(l)$ and
$h_2(x)(l)$. The arguments to these random variables, then, should be the left and right sublists themselves. 

 In order for us to obtain these sublists, it will be necessary for us to define a function 
 \begin{align*}
 \hat{h_i}(x) &: \prod_{l \in \Omega_x} \Omega_{h_i(x)(l)} \\
 \hat{h_i} &: \prod_{x \in R} \ ( \prod_{l \in \Omega_x} \Omega_{h_i(x)(l)})
 \end{align*}
 such that $\hat{h_i}(x)$ takes an input of size $x$ and returns the derived subproblem of size $h_i(x)(l)$. See, then, that
 $T(h_i(x)(l))(\hat{h_i}(x)(l))$ has type $\R$. So, this leaves us with the equation
 \begin{align*}
 T(x)(l) &= a(x) + T(h_1(x)(l))(\hat{h_1}(x)(l)) \dots + T(h_n(x)(l))(\hat{h_n}(x)(l))
 \end{align*}
 which, finally, typechecks.
 
 Thus, we find that attempting to formally assigns types to Karp's probabilistic recurrences reveals a lot of hidden 
 complications and ambiguities, which Karp never addresses directly. Most notably, we find that the types of the 
 functions $T, \ h,$ and $\hat{h}$ are dependent upon a real number $x$, describing the size of problems in the sample space.
 
We define a language to write and type-check these recurrences. This language is inspired $\lambda$LF---a simple system of 
dependent types (\cite{Pierce:2005aa}). However, instead of general dependent types, our language only needs to
describe sample space types $\Omega_n$ which are dependent upon a natural number $n$. Thus, we introduce
a constant $\lst$ to our syntax, which works as a function from expressions to types. That is, for each 
expression $e$, $\lst(e)$ is a type. There are many stochastic divide-and-conquer algorithms involving sample spaces
on datatypes other than lists; this language could easily be expanded to include other constants such as
 $\texttt{tree}, \ \texttt{graph}$, etc. Note that lists can only be indexed over natural numbers (it would not make 
 sense for a list to be any other size), so we include the type $\texttt{nat}$ in our language as well.
 \[
\begin{array}{rcl}
\tau &::=& \texttt{real} \mid \texttt{nat} \mid \texttt{bool} \mid \tau \times \tau \mid \Pi x:\tau.\tau
\mid \texttt{list}(e) \\
e &::=& x  \mid \texttt{0} \mid \texttt{1} \mid \texttt{2} \mid \dotsc \mid \lambda x.e \mid e \ e \mid e + e \mid e - e \mid  e  *  e \mid e / e \mid \texttt{true} \mid \texttt{false} \mid \\
  && e  =  e \mid e < e \mid e > e \mid e \leq e \mid e \geq e \mid 
     \ifelse{e}{e}{e} \mid (e,e) \mid \\
     && \texttt{rec}(a, (h_1, \ldots, h_n), (\hat{h_1}, \ldots \hat{h_n})) 
\end{array}
\]

In this syntax, we replace the arrow type $\sigma \rightarrow \tau$ with the dependent product type
$\Pi x:\sigma.\tau$, and introduce type families. 
These are collections of types that depend an input: for instance,
the type $\Pi n:\texttt{nat}.\lst(n)$, where the type $\lst(n)$ is dependent upon an input $n$ of type \texttt{nat}.
As with our languages for deterministic recurrences, we need a way of defining recursive functions. We first attempt
to define a PCF-like language analogous to the one in chapter 3, with a $\texttt{fix}$ operator that assigns the least
fixed-point to continuous functions. However, the inclusion of dependent product types creates some unique problems
in our denotational semantics.

Consider that in standard PCF, the denotation of the arrow type, $\llbracket \tau \rightarrow \sigma \rrbracket$, is the
set of continuous functions of type
$\llbracket \tau \rrbracket \rightarrow \llbracket \sigma \rrbracket$. But for dependent product types $\Pi(x:\tau).\sigma$,
this denotation does not make sense: 
since $\sigma$ may be dependent upon expressions of type $\tau$, there is no guarantee
that $\llbracket \sigma \rrbracket$ is a well-formed type (for instance, given the type $\Pi(n:\texttt{nat}).\lst(n)$, see that
$\llbracket \lst(n) \rrbracket$ is not well-formed because we do not know what $n$ is).

Thus, the denotation of a dependent product type $\Pi(x:\tau).\sigma$ must instead be the set of all
continuous functions $f$ 
of type $\llbracket \tau \rrbracket \rightarrow \bigcup_{a \in \llbracket \tau \rrbracket} \llbracket \sigma \rrbracket\{x \mapsto a\}$
where $\forall a \in \llbracket \tau \rrbracket, \ f(a) \in \llbracket \sigma \rrbracket\{x \mapsto a\}$. If we again look at the 
type $\Pi(n:\texttt{nat}).\lst(n)$, we see that its denotation is the set of all functions that map a natural number $n$ 
to a list of length $n$.

But there is a problem here, stemming from the fact that our denotation of dependent product types must be a 
set of \emph{continuous} functions. This is the case because the denotation of our fix operator,
$\llbracket \texttt{fix}(\lambda f. \lambda x.e) \rrbracket$, is $fix(\llbracket \lambda f. \lambda x.e \rrbracket)$, where $fix$ 
is a function that assigns the least fixed point to continuous functions. Thus, we must be able to guarantee that the denotation of any $\lambda f. \lambda x.e$ expression--that is, any expression with a dependent product type--- must be continuous. The formal definition of continuity in this setting is as 
follows: we say $f: P \rightarrow Q$ (where $P$ and $Q$ are CPOs) is continuous if, for every chain $C$ in $P, \ 
f(\bigvee C) = \bigvee f(C)$. That is, continuity of a function is premised on the fact that the domain and range of
a function are CPOs. However, it is not necessarily the case that
 $\bigcup_{a \in \llbracket \tau \rrbracket} \llbracket \sigma \rrbracket\{ x \mapsto a\}$ is a CPO, even though 
 $\llbracket \sigma \rrbracket\{x\mapsto a\}$ is a CPO for $\forall a \in \llbracket \tau \rrbracket$.

As a result, continuity does not have any meaning for functions of type $\llbracket \Pi (x: \tau).\sigma \rrbracket$. 
So we cannot use the $\texttt{fix}$ operator on arbitrary $\lambda f. \lambda x.e$ expressions, and the whole
framework of our denotational semantics falls apart. 
 
 \begin{figure}
\[
\begin{array}{c}
\Gamma \vdash \texttt{real} :: * \\ \\
\Gamma \vdash \texttt{bool} :: *\\ \\
\Gamma \vdash \texttt{nat} :: * \\ \\ 
\dfrac{\Gamma \vdash \tau_1 :: * \ \ \ \Gamma, x:\tau_1 \vdash \tau_2 :: *}{\Gamma \vdash \Pi x:\tau_1.\tau_2 :: *} \\ \\
\dfrac{\Gamma \vdash e : \texttt{nat}}{\Gamma \vdash \lst(e) :: *} 
\end{array}
\]
\caption{Well-formed types for dependent type language.}
\end{figure}

As an alternative approach, we define a language which includes a $\texttt{rec}$ operator similar to the one in chapter 2,
 which allows to write recurrences of the form $T(x)(l) = a(x) + T(h(x)(l))\hat{h}(x)(l)$, where $T, \ a, \ h$, and $\hat{h}$ 
 are functions with types as described above. As in chapter 2, this approach restricts us to expressing only a small subset
 of recursive functions; however, it does allow us to express the probabilistic recurrences described by Karp,
 without having to worry about the problem of requiring continuity on functions with dependent product types. 

We offer standard type judgements in Figure 2, as well as judgements for when a type is well-formed, denoted by
 $\tau :: *$, in Figure 1. We 
 also offer an equational semantics in Figure 3 and denotational semantics for types and expressions in Figures 4 and 5. 
 We may verify soundness of our denotational semantics with an argument analogous to the one given in chapter 2. 
 
  \begin{thm}
  If $\Gamma \vdash e_0 : \tau$ and $\Gamma \vdash e_1 : \tau$ and $e_0 = e_1$, then for all $\Gamma$-environments
  $\eta$,   
 \begin{align*}
 \llbracket \Gamma \vdash e_0 \rrbracket \eta 
 = \llbracket \Gamma \vdash e_1 \rrbracket \eta 
 \end{align*} 
 \end{thm}

\begin{figure}
\[
\begin{array}{lr}
 \\ \\
\dfrac{}{\Gamma \vdash \texttt{0}: \texttt{real}}, \ \ \dfrac{}{\Gamma \vdash \texttt{1}: \texttt{real}}, \ldots \\ \\
\dfrac{}{\Gamma \vdash \texttt{true} : \texttt{bool}}, \ \ \dfrac{}{\Gamma \vdash \texttt{false} : \texttt{bool}} \\  \\
\dfrac{x : \tau \in \Gamma \ \ \ \Gamma \vdash \tau :: *}{\Gamma \vdash x : \tau } \\ \\ 
\dfrac{\Gamma \vdash \sigma :: * \ \ \ \Gamma, x : \sigma \vdash e : \tau}
	{\Gamma \vdash \lambda (x : \sigma).e : \Pi x:\sigma.\tau } \\ \\
\dfrac{\Gamma \vdash e_1: \Pi x:\sigma.\tau \ \ \ \Gamma \vdash e_2 : \sigma}
	{\Gamma \vdash e_1 \ e_2 : [x \mapsto e_2]\tau} \\ \\
\dfrac{\Gamma \vdash e_1 : \texttt{real} \ \ \ \Gamma \vdash e_2 : \texttt{real}}
	{\Gamma \vdash e_1 \circ e_2 : \texttt{real}}
, \circ \in \{+,-,*,/\} \\ \\ 
\dfrac{\Gamma \vdash e_1 : \texttt{real} \ \ \ \Gamma \vdash e_2 : \texttt{real}}
	{\Gamma \vdash e_1 \circ e_2 : \texttt{bool}}
	, \circ \in \{=, <, >, \geq, \leq\} \\ \\ 
\dfrac{\Gamma \vdash e_1 : \texttt{bool} \ \ \ \Gamma \vdash e_2 : \tau \ \ \ \Gamma \vdash e_3 : \tau}
	{\Gamma \vdash \ifelse{e_1}{e_2}{e_3} : \tau} \\ \\ 
\dfrac{\Gamma \vdash e_1 : \tau_1 \ \ \ \Gamma \vdash e_2 : \tau_2}{\Gamma \vdash (e_1,e_2) : \tau_1 \times \tau_2}\\ \\
\dfrac{\Gamma \vdash a: \Pi (n:\texttt{nat}).\texttt{nat} \ \ \ \Gamma \vdash h_i: \Pi(n:\texttt{nat}).(\Pi (l:\lst(n)).\texttt{nat}) \ \ \ \Gamma \vdash \hat{h_i}: \Pi(n: \texttt{nat}).(\Pi(l : \lst(n)).\lst(h n l))}
	{\Gamma \vdash \texttt{rec}(a, (h_1, \ldots, h_n), (\hat{h_1} \ldots \hat{h_n})) : \Pi(n:\texttt{nat}).(\Pi (l:\lst(n)).\texttt{nat})}
\end{array}
\]
\caption{Typing for dependent type language.}
\end{figure}

\begin{figure}
\[
\begin{array}{lr}
e = e \\ \\
\texttt{0} + \texttt{0} = \texttt{0}, \ \texttt{0} + \texttt{1} = 1, \ldots, \ \texttt{3} + \texttt{5} = \texttt{8}, \ldots  \\
\text{equivalent rules for } -, \ * \text{, and } /\text{ operations.}
\\ \\
(n =n) = \texttt{true} \\ (n=m) = \texttt{false}\text{, provided } n, \ m \text{ distinct numerals.}\\ \\ 
\ifelse{\texttt{ true }}{e_1}{e_2} = e_1 \\
\ifelse{\texttt{ false }}{e_1}{e_2} = e_2 \\ \\ 
\lambda x.e = \lambda y.[x \mapsto y]e, \text{provided } y \text{ not free in } e. \\ \\ \\
\lambda x.e_1 \ e_2 = [x \mapsto e_2]e_1 \\
\texttt{rec}(\lambda x.a, ( \lambda x.\lambda l.h_1, \ldots \lambda x.\lambda l.h_n)
( \lambda x.\lambda l.\hat{h_1}, \ldots \lambda x.\lambda l.\hat{h_n}))= \\
\lambda x.(e_1 + \sum_{i=1}^n(\texttt{rec}(\lambda x.a, \sum_{i=1}^n \lambda x.\lambda l.h_i, 
\sum_{i=1}^n \lambda x.\lambda l.\hat{h_i}) h_i  \ \hat{h_i}))
\end{array}
\]
\caption{Equational Semantics for the language.}
\label{fig:typing}
\end{figure}

\begin{figure}
 \begin{align*}
\llbracket \texttt{real} \rrbracket &= \R \\
\llbracket \texttt{nat} \rrbracket &= \N\\
 \llbracket \texttt{bool} \rrbracket &= {\{true, false\}}\\
 \llbracket \Pi x:\tau.\sigma \rrbracket\eta &= \{f: \llbracket \tau \rrbracket\eta \rightarrow 
 \bigcup_{a \in \llbracket \tau \rrbracket\eta} \llbracket \sigma \rrbracket\eta\{x \mapsto a\}
 \mid  \\
 &\forall a \in \llbracket \tau \rrbracket\eta, \ f(a) \in \llbracket \sigma\rrbracket\eta\{x\mapsto a\}\} \\
 \llbracket \lst(e) \rrbracket\eta &= \Omega_{\llbracket e \rrbracket\eta} \\
 \forall n \in \N, \ \Omega_n &\text{ is the sample space of all lists of length } n. 
 \end{align*}
 \caption{Denotational semantics for types.}
 \end{figure}
 \begin{figure}
 \begin{align*}
 \llbracket \texttt{0} \rrbracket\eta &= 0, \  \llbracket \texttt{1} \rrbracket\eta = 1, \ \ldots \\
  \llbracket x : \tau \rrbracket\eta &= \eta(x) \\
  \llbracket \lambda (x : \tau) . e \rrbracket\eta &= f : \llbracket \tau \rrbracket\eta 
  \rightarrow \bigcup_{a \in \llbracket \tau \rrbracket\eta} \llbracket \sigma \rrbracket\eta\{x\mapsto a\} \\
\text{ s.t. } \forall a \in \llbracket \tau \rrbracket\eta, f(a) &= \llbracket e \rrbracket\eta\{ x \mapsto a \} 
  \in \llbracket \sigma \rrbracket\eta\{x\mapsto a\}\\
 \llbracket e_1 \ e_2 \rrbracket \eta &= \llbracket e_1 \rrbracket\eta ( \llbracket e_2 \rrbracket\eta ) \\
 \llbracket e_1 + e_2 \rrbracket\eta &= \llbracket e_1 \rrbracket\eta + \llbracket e_2 \rrbracket\eta \\
\text{Similar rules for $-, \ *, \ /$} \\
  \llbracket \texttt{true} \rrbracket\eta &= true, \ \llbracket \texttt{false} \rrbracket\eta = false \\
 \llbracket e_1 = e_2 \rrbracket\eta &= 
 \begin{cases} 
      true \text{ if } (\llbracket e_1 \rrbracket\eta = \llbracket e_2 \rrbracket\eta) \\
      false \text{  if } (\llbracket e_1 \rrbracket\eta \neq \llbracket e_2\rrbracket\eta)\\
   \end{cases} \\
\text{Similar rules for $<, \ \leq, >, \geq$} \\
  \llbracket \ifelse{e_1}{e_2}{e_3} \rrbracket \eta &= 
 \begin{cases} 
      \llbracket e_2 \rrbracket\eta \text{ if } \llbracket e_1 \rrbracket\eta = true \\
      \llbracket e_3 \rrbracket\eta \text{ if } \llbracket e_1 \rrbracket\eta = false \\ 
   \end{cases}
  \\
  \llbracket (e_1,e_2) \rrbracket\eta &= (\llbracket e_1 \rrbracket\eta, \llbracket e_2 \rrbracket\eta) \\
    \llbracket  \texttt{rec} (a,\sum_{i=1}^n h_i,\sum_{i=1}^n \hat{h_i}) \rrbracket\eta &= rec: \N \rightarrow 
    (\bigcup_{n\in \N} \Omega_n \rightarrow \N)   \text{ s.t. }  \\
  \forall n \in \N, \ 
    \forall l \in \Omega_n, 
    rec(n)(l) &= \sum_{j=0}^{\infty}(\sum_{(n',l') \in S^j(n,l)} \llbracket a \rrbracket\eta(n')) \\
    &S^j \text{ is defined inductively as follows:} \\
    S^0(n,l) &= \{(n, l)\} \\
    S^{j+1}(n,l) &= \bigcup_{0<i \leq n}\{(\llbracket h_i \rrbracket\eta(n')(l') , \llbracket \hat{h_i} \rrbracket\eta(n')(l')) \mid 
    \forall (n', l') \in S^j \}
 \end{align*}
 \caption{Denotational semantics for expressions.}
 \end{figure}

We want to gain some intuition that our denotation of a $\texttt{rec}$ expression is equivalent to the solution of
the corresponding semantic recurrence. To that end, we will show that, for a program $T$ describing a recurrence for 
quicksort, the denotation of $T$ for valid inputs is equal to the actual cost of quicksorting. 
 Following Karp's definition, this random variable has the form 
\begin{align*}
T(x)(l) = a(x) + T(h_1(x)(l))(\hat{h_1}(x)(l)) + T(h_2(x)(l))(\hat{h_2}(x)(l)), 
\end{align*}
where $a(x) = x-1$, the cost of one step of quicksort (i.e. comparing all other elements
in the list to the head of the list), $h_1(x)$ and $h_2(x)$ give the sizes of the left and right sublists of an input, and
$\hat{h_1}$ and $\hat{h_2}$ give the sublists themselves. 

We may express this function in our expression language
as follows:
\begin{align*}
\texttt{T} = \texttt{rec}(a, (h_1, h_2), (\hat{h_1}, \hat{h_2}))
\end{align*}
Then, we want to prove the following claim:
\begin{thm}
For all $n \in \N, \ l \in \Omega_n$, suppose $\llbracket h_1 \rrbracket\eta (n)(l)$ and $\llbracket h_2 \rrbracket\eta (n)(l)$
are equal to the number of elements in the $l$ less than and greater than or equal to the head of the $l$, respectively (that is,
they accurately reflect the sizes of the subproblems of quicksort). Then $\llbracket T \rrbracket\eta (n)(l)$ is equal to the cost
of performing quicksort on $l$. 
\end{thm} 

\begin{proof}
By induction on $n$.

\emph{Base case}: $n = 1$.  For all $l \in \Omega_n, \ l$ is a list of length $1$. Then
\begin{align*}
\llbracket \texttt{T} \rrbracket\eta (n)(l) &=  \sum_{j=0}^{\infty}(\sum_{(n',l') \in S^j(n,l)} \llbracket a \rrbracket\eta(n')) \\
&= \llbracket a \rrbracket\eta(n) + \sum_{j=1}^{\infty}(\sum_{(n',l') \in S^j(n,l)} \llbracket a \rrbracket\eta(n')) \\
&= \llbracket a \rrbracket\eta(n) + 0 = n - 1 = 0,
\end{align*}
since the subproblems of quicksorting a list of length 1 both have size $0$. See that this is the actual cost of performing
quicksort on a list of length $1$, so the claim holds.

\emph{Inductive case}: Suppose the claim holds for all $k < n$, and let $l \in \Omega_n$. Then
\begin{align*}
 \llbracket \texttt{T} \rrbracket\eta (n)(l) &=  \sum_{j=0}^{\infty}(\sum_{(n',l') \in S^j(n,l)} \llbracket a \rrbracket\eta(n')) \\
 &= \llbracket a \rrbracket\eta(n) + \sum_{j=1}^{\infty}(\sum_{(n',l') \in S^j(n,l)} \llbracket a \rrbracket\eta(n')) \\
&= (n-1) + \sum_{j=0}^{\infty}(\sum_{(n',l') \in S^j(\llbracket h_1 \rrbracket\eta(n),
\llbracket \hat{h_1} \rrbracket \eta(l))} \llbracket a \rrbracket\eta(n')) \\
 & \text{ \ \ \ \ \ \ \ \ \ \ \ \ } + \sum_{j=0}^{\infty}(\sum_{(n',l') \in S^j(\llbracket h_2 \rrbracket \eta(n),
 \llbracket \hat{h_2} \rrbracket \eta(l))} \llbracket a \rrbracket\eta(n')) \\
 &= (n-1) + \llbracket \texttt{T} \rrbracket\eta(\llbracket h_1 \rrbracket\eta(n)(l))(\llbracket \hat{h_1} \rrbracket\eta(n)(l)) \\
 &\text{ \ \ \ \ \ \ \ \ \ \ \ \ }+  \llbracket \texttt{T} \rrbracket\eta(\llbracket h_2 \rrbracket\eta(n)(l))(\llbracket \hat{h_2} \rrbracket\eta(n)(l))  
\end{align*}
By inductive hypothesis, 
\begin{align*}
&\llbracket \texttt{T} \rrbracket\eta(\llbracket h_1 \rrbracket\eta(n)(l))(\llbracket \hat{h_1} \rrbracket\eta(n)(l)) \text{ and } \\
&\llbracket \texttt{T} \rrbracket\eta(\llbracket h_2 \rrbracket\eta(n)(l))(\llbracket \hat{h_2} \rrbracket\eta(n)(l))
\end{align*}
 are equal
to the cost of performing quicksort on the left and right sublists of $l$. Then $\llbracket \texttt{T} \rrbracket\eta (n)(l)$ is equal 
to the sum of these costs, plus $n-1$; this is the cost of performing quicksort on $l$.

Therefore, the claim holds for all $n$.
\end{proof}
Thus, we have a clearer indication that the syntax and denotational semantics defined in this chapter correctly 
represent Karp's probabilistic recurrence relations.


\chapter{Conclusion}

We have defined three syntaxes which allow us to write and type-check the several types of recurrences
described by Karp. The first allows us to describe deterministic recurrences of the form $T(x) = a(x) + T(m(x))$, where $a$ and $m$ are real-valued functions. The second allows us to describe recurrences of this form, but with an added initial 
condition $T(m) = c$. And the third allows us to describe probabilistic recurrences of the form 
$T(x) = a(x) + T(h(x)(l))\hat{h}(x)(l)$, where $a$ is a real-valued function, $h(x)$ is a random variable of type
$\Omega_x \rightarrow \R$, and $\hat{h}(x)$ is a random variable of type
 $\prod_{l \in \Omega_x} \Omega_{h(x)(l)}$. Moreover, we have discussed a means of translating between
 the first and second syntaxes. 
 
 \section{Future Work}
 In future work, we hope to tie in the work done in this thesis to the current cost extraction framework developed
 by \cite{N.-Danner:2015aa}. Its current structure is as follows: a source language program $e$ is mapped by an 
 extraction function to a syntactic cost recurrence $E$. This syntactic recurrence is then mapped by a denotational
 semantics to a semantic recurrence $\llbracket E \rrbracket$ which describes an upper bound on the cost of $e$. 
 \begin{align*}
 e\xrightarrow{\text{extr. function}}E\xrightarrow{\text{den. semantics}}\llbracket E \rrbracket
 \end{align*}
 However, as discussed in the introduction, this denotational semantics does not always offer useful bounds on 
 probabilistic recurrences. We aim to expand the cost recurrence syntax to support the syntax for 
 probabilistic recurrences offered in this thesis, as well as to include the denotational semantics for that syntax.
 This will mean that, given a syntactic recurrence $E$, we can interpret it using either denotational semantics;
which semantics we use will be determined by what kind of analysis we wish to make. This will allow us to obtain 
better bounds on the costs of a wider range of algorithms.

%\begin{thebibliography}{9}
%\bibitem{Karp}
%	R. M. Karp. Probabilistic recurrence relations. J.
%	ACM, 41(6):1136?1150, 1994.
%	
%\bibitem{ICFP}
%	N. Danner, D. Licata, and R. Ramyaa.
%        Denotational cost semantics for functional languages with inductive types.
%	\emph{International Conference on Functional Programming}, 2015
%
%\bibitem{Coq}
%	J. Tassarotti and R. Harper. 2017. Verified Tail Bounds for Randomized Programs. 
%	Proc. ACM Program. Lang. 1, 1, Article 1 (January 2017).
%
%\end{thebibliography}

\bibliographystyle{plainnat}
\bibliography{bibliography}

\end{document}