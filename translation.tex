\chapter{Reconciling the Two Semantics}
Chapter 2 and 3 have presented languages for defining the two distinct kinds of recurrences present in \cite{Karp}. 
In this chapter, we look at several attempts to connect these two recurrences. In particular, we want to find a way 
to write a recurrence with an initial condition as a recurrence of the form $T(x) = a(x) + T(m(x))$, as this is the kind
of recurrence that Karp proves theorems about. We will offer
several methods of translating from one expression language to the other, and investigate the validity of these 
approaches.

As a first attempt, we consider a simple recurrence with an initial condition.
\begin{align*}
T(1) &= 0 \\
T(n) &= 1 + T(n/2 ) 
\end{align*}
As in the example in chapter 3, we assume that the domain of $T$ is positive powers of $2$.
This is the kind of recurrence described in chapter 3. In order to make this fit the definition of a recurrence described in 
chapter 2 --- that is, a recurrence of the form $T(n) = a(n) + T(m(n))$--- we may interpret $a$ as a 
piecewise function with two sub-functions: one for the general recurrence and one for the initial condition.
We can instead write this as a single recurrence
\begin{align*}
T(n) &= a(n) + T(m(n)) \\
m(n) &= n/2, \ \forall n \\ 
a(n) &=
\begin{cases} 
0, \ n \leq 1 \\
1, \ n > 1
\end{cases}
\end{align*} 
This recurrence can then by expressed in our chapter 2 syntax as 
\begin{align*}
\texttt{T} &= \texttt{rec}(a,m) \\
m &= \lambda x.x/\texttt{2} \\
a &= \lambda x.\ifelse{x\leq\texttt{1}}{\texttt{0}}{\texttt{1}} \\
\llbracket \texttt{T} \rrbracket(d) &= \llbracket \texttt{rec}(a,m) \rrbracket(d) \\
&= \sum_{i=0}^{\infty}\llbracket a \rrbracket(\llbracket m\rrbracket^i(d)) \\
&= \sum_{i=0}^{\log(n)-1} 1 + \sum_{i=log(n)}^{\infty} 0 \\
&= \log(n) - 1
\end{align*}

However, this approach will not work in every case. Let us try to use the same process on the following recurrence
\begin{align*}
T(1) &= 1 \\
T(n) &= 2 + T(n/2) 
\end{align*}
We first write this as a single recurrence. See that, unlike the previous example, $a$ never maps to $0$.
\begin{align*}
T(n) &= a(n) + T(m(n)) \\
m(n) &= n/2, \ \forall n \\ 
a(n) &= 
\begin{cases}
1, \ n \leq 1 \\
2, \ n > 1
\end{cases}
\end{align*}
We then express this recurrence using our chapter 2 syntax as
\begin{align*}
\texttt{T} &= \texttt{rec}(a,m) \\
m &= \lambda x.x/2 \\
a &=\lambda x.\ifelse{x\leq1}{1}{2} \\
\llbracket \texttt{T} \rrbracket(d) &= \llbracket \texttt{rec}(a,m) \rrbracket(d) \\
&= \sum_{i=0}^{\infty}\llbracket a \rrbracket(\llbracket m\rrbracket^i(d)) \\
&= \sum_{i=0}^{\log(n)-1} 2 + \sum_{i=\log(n) }^{\infty} 1 \\
&= \infty
\end{align*}
Which, clearly, is not a correct solution.

A second attempt takes a similar approach to writing a recurrence with an initial condition as a recurrence of 
the form $T(n) = a(n) + T(m(n))$. Suppose that we interpret $a$ as a piecewise function with three sub-functions:
one for the general recurrence, one for the initial condition, and one for values smaller than the initial condition, which all
map to $0$. Considering the example that failed in the previous attempt, we would have
\begin{align*}
T(n) &= a(n) + T(m(n)) \\
m &= n/2, \forall n \\
a(n) &=
\begin{cases}
0, \ n < 1 \\
1, \ n = 1 \\
2, \ n > 1
\end{cases}
\end{align*}
We can express this recurrence using our fixed-point syntax as
\begin{align*}
\texttt{T} &= \texttt{rec}(a,m) \\
m &= \lambda x.x/\texttt{2} \\
a &= \lambda x.\ifelse{x <1}{0}{(\ifelse{x = 1}{1}{2})} \\
\llbracket a \rrbracket(d) &= 
 \begin{cases}
0, \ d < 1 \\
1, \ d = 1 \\
2, \ d > 1 
\end{cases} \\
\llbracket \texttt{T} \rrbracket(d) &= \llbracket \texttt{rec}(a,m)\rrbracket(d) \\
&= \sum_{i=0}^{\infty}\llbracket a \rrbracket(\llbracket m\rrbracket^i(d)) \\
&= \sum_{i=0}^{\log(n)-1} 2 + \sum_{i=\log(n) }^{\log(n)} 1 + \sum_{i = \log(n) + 1}^{\infty} 0 \\
&= 2\log(n) - 1
\end{align*}

This approach matches our intuition for how an initial condition should work---that once you reach the initial condition,
you stop recursing. 

With this in mind, we can identify a general strategy for extracting a recurrence in our chapter 2 syntax (with
the $\texttt{rec}$ operator),given a recurrence in our chapter 3 syntax (with the $\texttt{fix}$ operator). 
While we cannot extract a recurrence from every $\texttt{fix}(\lambda f \lambda x.e)$ expression, we can achieve this
if $e$ is of a form that corresponds to a recurrence with an initial condition. 

Suppose we have a recurrence with an initial condition, given as follows:
\begin{align*}
T(b) &= c, \text{ where } b, c,  \in \R \\
T(x) &= a(x) + T(m(x)).
\end{align*}
We write this recurrence in our chapter 3 syntax as 
\begin{align*}
\texttt{fix}(\lambda f.\lambda x. \ifelse{x = m}{c}{a(x) + f(m(x))}),
\end{align*}
an expression whose denotation will be a function $F$ such that
\begin{align*}
F(x) &=
\begin{cases}
c \text{ if } x = b \\
a(x) + c \text{ if } m(x) = b \\
a(x) + a(m(x)) + c \text{ if } m(m(x)) = b \\
\ldots \\
\sum_{i=1}^{n} a(m^{i-1}(x)) + c \text{ if } m^n(x) = b \\
\ldots
\end{cases}
\end{align*}

We may then write this recurrence in our chapter 2 syntax as $\texttt{rec}(a',m)$, where m is the same function
as in the previous recurrence, and
\begin{align*}
a' = \lambda x.\lambda f. \ifelse{x < b}{0}{(\ifelse{x = b}{c}{a(x)})}.
\end{align*}
Then the denotation of $\texttt{rec}(a',m)$ is a function F' such that 
\begin{align*}
F'(x) &= \sum_{i=0}^{\infty} \llbracket a' \rrbracket(\llbracket m \rrbracket^i(x)) \\
&= \sum_{i=0}^{n-1} \llbracket a' \rrbracket( \llbracket m \rrbracket^i(x)) + \sum_{i=n}^n c + \sum_{i = n+1}^{\infty} 0 \\
&\text{ where } \llbracket m \rrbracket^n(x) = b \\
&= \sum_{i=1}^{n}\llbracket a' \rrbracket ( \llbracket m \rrbracket^{i-1}(x)) + c
\end{align*}
See, then, that $F' = F$. Therefore, this confirms that, given an expression $e$ of the specific form
$e = \ifelse{x = m}{c}{a(x) + f(m(x))}$, we can extract a function a' such that 
\begin{align*}
\llbracket \texttt{fix}(\lambda f. \lambda x.e) \rrbracket = \llbracket \texttt{rec}(a',m)\rrbracket
\end{align*}





 