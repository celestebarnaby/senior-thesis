\chapter{Well-Typed Probabilistic Recurrence Relations}
Karp's paper is premised on the notion that the cost of a stochastic algorithm may be described as a
recurrence relation of the form
\begin{align*}
T(x) = a(x) + T(h_1(x)) + \dots + T(h_n(x)).
\end{align*}
But what does this equation actually mean? To unpack this, we will attempt to assign types to this recurrence.

 Following Karp's definitions, we can immediately assign types to some of these terms:
\begin{align*}
T(x) &: \Omega_x \rightarrow \R \text{, where } \Omega_x \text{ is the sample space of all problem instances of size } x \\
T &: \prod_{x \in \R} (\Omega_x \rightarrow \R) \\ 
a &: \R \rightarrow \R
\end{align*}

However, for other terms it is less clear what the type should should be. Karp alternates between saying that $h$ is a random 
variable and $h(x)$ is a random variable, so it is not immediately obvious 
what the meaning of this function is. However, since $h$ takes $x \in \R$ as an input, it would not make sense for $h$ to be a 
random variable of type $\R \rightarrow \R$. This would make $h$ a function that takes an input size and returns the 
size of the derived subproblem---thus sidestepping the stochastic part of the algorithm. Then it must be that $h(x)$ is a 
random variable on \emph{specific input} of size $x$---i.e., the sample space $\Omega_x$ and returns the size of the derived 
subproblem.
\begin{align*}
h_i(x) &: \Omega_x \rightarrow \R \\
h_i &: \prod_{x \in \R}(\Omega_x \rightarrow \R) 
\end{align*} 

There are some inconsistencies in these type assignments. To start, $T$ is a function that takes a real number and 
returns a random variable. But on the right-hand side of the relation, $T$ takes $h_i(x)$ as an argument; $h_i(x)$ is
not a real number, but a random variable. 

Thinking in terms of what we \emph{want} this recurrence to express, the arguments to the $T$'s on the righthand side of
the equation should be the sizes of the derived subproblems---that is, the result of applying $h_i(x)$ to the original input.
Then, given an input $l \in \Omega_x$, we should have
\begin{align*}
T(x)(l) &= a(x) + T(h_1(x)(l)) \dots + T(h_n(x)(l)) \\
&\text{where } h(x)(l) : \R 
\end{align*}
 Thus, the term $T(h_i(x)(l))$ typechecks. However, we now run into another problem: $T(x)(l)$ and $a(x)$ are real numbers,
but $T(h_i(x)(l))$ is a random variable of type $\Omega_{h_i(x)(l)} \rightarrow \R$. Then $T(h_i(x)(l))$ needs an argument.
Consider that $T(h_i(x)(l))$ is a random variable describing the running time of the algorithm with an input of size 
$h_i(x)(l)$. For instance, given a recurrence for randomized quicksort, $T(h_1(x)(l))$ and $T(h_2(x)(l))$ describe how long it 
takes to quicksort the left and right sublists of an input list $l$, which themselves have length $h_1(x)(l)$ and
$h_2(x)(l)$. The arguments to these random variables, then, should be the left and right sublists themselves. 

 In order for us to obtain these sublists, it will be necessary for us to define a function 
 \begin{align*}
 \hat{h_i}(x) &: \prod_{l \in \Omega_x} \Omega_{h_i(x)(l)} \\
 \hat{h_i} &: \prod_{x \in R} \ ( \prod_{l \in \Omega_x} \Omega_{h_i(x)(l)})
 \end{align*}
 such that $\hat{h_i}(x)$ takes an input of size $x$ and returns the derived subproblem of size $h_i(x)(l)$. See, then, that
 $T(h_i(x)(l))(\hat{h_i}(x)(l))$ has type $\R$. So, this leaves us with the equation
 \begin{align*}
 T(x)(l) &= a(x) + T(h_1(x)(l))(\hat{h_1}(x)(l)) \dots + T(h_n(x)(l))(\hat{h_n}(x)(l))
 \end{align*}
 which, finally, typechecks.
 
 Thus, we find that attempting to formally assigns types to Karp's probabilistic recurrences reveals a lot of hidden 
 complications and ambiguities, which Karp never addresses directly. Most notably, we find that the types of the 
 functions $T, \ h,$ and $\hat{h}$ are dependent upon a real number $x$, describing the size of problems in the sample space.
 
We define a syntax to write and type-check these recurrences. This syntax is inspired $\lambda$LF---a simple system of 
dependent types (\cite{Pierce:2005aa}). However, instead of general dependent types, our syntax only needs to
describe sample space types $\Omega_n$ which are dependent upon a natural number $n$. Thus, we introduce
a constant $\lst$ to our syntax, which works as a function from expressions to types. That is, for each 
expression $e$, $\lst(e)$ is a type. There are many stochastic divide-and-conquer algorithms involving sample spaces
on datatypes other than lists; this language could easily be expanded to include other constants such as
 $\texttt{tree}, \ \texttt{graph}$, etc.

 \[
\begin{array}{rcl}
\tau &::=& \texttt{real} \mid \texttt{nat} \mid \texttt{bool} \mid \tau \times \tau \mid \Pi x:\tau.\tau
\mid \texttt{list}(e) \\
e &::=& x  \mid \texttt{0} \mid \texttt{1} \mid \texttt{2} \mid \dotsc \mid \lambda x.e \mid e \ e \mid e + e \mid e - e \mid  e  *  e \mid e / e \mid \texttt{true} \mid \texttt{false} \mid \\
  && e  =  e \mid e < e \mid e > e \mid e \leq e \mid e \geq e \mid 
     \ifelse{e}{e}{e} \mid \\
     && \texttt{fix}(\lambda f. \lambda x.e) 
\end{array}
\]

In this syntax, we replace the arrow type $\sigma \rightarrow \tau$ with the dependent product type
$\Pi x:\sigma.\tau$, and introduce type families. 
These are collections of types that depend an input: for instance,
the type $\Pi n:\texttt{nat}.\lst(n)$, where the type $\lst(n)$ is dependent upon an input $n$ of type \texttt{nat}.

We offer standard type judgements, as well as judgements for when a type is well-formed, denoted by $\tau :: *$. 

\begin{figure}
\[
\begin{array}{c}
\Gamma \vdash \texttt{real} :: * \\ \\
\Gamma \vdash \texttt{bool} :: *\\ \\
\Gamma \vdash \texttt{nat} :: * \\ \\ 
\dfrac{\Gamma \vdash \tau_1 :: * \ \ \ \Gamma, x:\tau_1 \vdash \tau_2 :: *}{\Gamma \vdash \Pi x:\tau_1.\tau_2 :: *} \\ \\
\dfrac{\Gamma \vdash e : \texttt{nat}}{\Gamma \vdash \lst(\ e) :: *} 
\end{array}
\]
\caption{Well-formed types for dependent type language.}
\end{figure}

\begin{figure}
\[
\begin{array}{lr}
 \\ \\
\dfrac{}{\Gamma \vdash \texttt{0}: \texttt{real}}, \ \ \dfrac{}{\Gamma \vdash \texttt{1}: \texttt{real}}, \ldots \\ \\
\dfrac{}{\Gamma \vdash \texttt{true} : \texttt{bool}}, \ \ \dfrac{}{\Gamma \vdash \texttt{false} : \texttt{bool}} \\  \\
\dfrac{x : \tau \in \Gamma \ \ \ \Gamma \vdash \tau :: *}{\Gamma \vdash x : \tau } \\ \\ 
\dfrac{\Gamma \vdash \sigma :: * \ \ \ \Gamma, x : \sigma \vdash e : \tau}
	{\Gamma \vdash \lambda (x : \sigma).e : \Pi x:\sigma.\tau } \\ \\
\dfrac{\Gamma \vdash e_1: \Pi x:\sigma.\tau \ \ \ \Gamma \vdash e_2 : \sigma}
	{\Gamma \vdash e_1 \ e_2 : [x \mapsto e_2]\tau} \\ \\
\dfrac{\Gamma \vdash e_1 : \texttt{real} \ \ \ \Gamma \vdash e_2 : \texttt{real}}
	{\Gamma \vdash e_1 \circ e_2 : \texttt{real}}
, \circ \in \{+,-,*,/\} \\ \\ 
\dfrac{\Gamma \vdash e_1 : \texttt{real} \ \ \ \Gamma \vdash e_2 : \texttt{real}}
	{\Gamma \vdash e_1 \circ e_2 : \texttt{bool}}
	, \circ \in \{=, <, >, \geq, \leq\} \\ \\ 
\dfrac{\Gamma \vdash e_1 : \texttt{bool} \ \ \ \Gamma \vdash e_2 : \tau \ \ \ \Gamma \vdash e_3 : \tau}
	{\Gamma \vdash \ifelse{e_1}{e_2}{e_3} : \tau} \\ \\ 
\dfrac{\Gamma \vdash \lambda f \lambda x.e : \Pi x:(\Pi x:\tau.\tau).(\Pi x:\tau.\tau)}
	{\Gamma \vdash \texttt{fix}(\lambda f. \lambda x.e) : \Pi x:\tau.\tau}
\end{array}
\]
\caption{Typing for dependent type language.}
\end{figure}


\begin{figure}
\[
\begin{array}{lr}
e = e \\ \\
\texttt{0} + \texttt{0} = \texttt{0}, \ \texttt{0} + \texttt{1} = 1, \ldots, \ \texttt{3} + \texttt{5} = \texttt{8}, \ldots  \\
\text{equivalent rules for } -, \ * \text{, and } /\text{ operations.}
\\ \\
(n =n) = \texttt{true} \\ (n=m) = \texttt{false}\text{, provided } n, \ m \text{ distinct numerals.}\\ \\ 
\ifelse{\texttt{ true }}{e_1}{e_2} = e_1 \\
\ifelse{\texttt{ false }}{e_1}{e_2} = e_2 \\ \\ 
\lambda x.e = \lambda y.[x \mapsto y]e, \text{provided } y \text{ not free in } e. \\ \\ \\
\lambda x.e_1 \ e_2 = [x \mapsto e_2]e_1 \\
\texttt{fix}(\lambda f. \lambda x.e) = [f \mapsto  \texttt{fix}(\lambda f.\lambda x.e)](\lambda x.e)
\end{array}
\]
\caption{Equational Semantics for the language.}
\label{fig:typing}
\end{figure}

\begin{figure}
 \begin{align*}
\llbracket \texttt{real} \rrbracket &= \R_{\perp} \\
\llbracket \texttt{nat} \rrbracket &= \N_{\perp} \\
 \llbracket \texttt{bool} \rrbracket &= {\{true, false\}}_{\perp} \\
 \llbracket \tau \times \sigma \rrbracket\eta &= \llbracket \tau \rrbracket\eta \times \llbracket \sigma \rrbracket\eta  \\
 \text{For } \forall (a, b), \ (c, d) \in \llbracket \tau \times \sigma \rrbracket &\text{, let }  (a,b) \leq (c,d) \text{ if and only if } a<c \text{ and } b<d. \\ 
 \llbracket \Pi x:\tau.\sigma \rrbracket\eta &= \{f: \llbracket \tau \rrbracket\eta \rightarrow 
 \bigcup_{a \in \llbracket \tau \rrbracket\eta} \llbracket \sigma \rrbracket\eta\{x \mapsto a\}
 \mid f \text{ is continuous}\} \\
 \text{ For } \forall f, \ g \in \llbracket \Pi x : \tau.\sigma \rrbracket\eta &\text{, let } f \leq g \text{ if and only if } \forall x \in 
 \llbracket \tau \rrbracket\eta, \ f(x) \leq g(x) \\
 \llbracket \lst(e) \rrbracket\eta &= \Omega_{\llbracket e \rrbracket\eta} \\
 \text{For } \forall n \in \N, \ &\Omega_n \text{ is the flat-ordered set of all lists of length } n. \\
 &\Omega_{\perp_\N} \text{ is the set } \{\perp_\N\}.
 \end{align*}
 \caption{Denotational semantics for types.}
 \end{figure}
 \begin{figure}
 \begin{align*}
 \llbracket \texttt{0} \rrbracket\eta &= 0, \  \llbracket \texttt{1} \rrbracket\eta = 1, \ \ldots \\
  \llbracket x : \tau \rrbracket\eta &= \eta(x) \\
  \llbracket \lambda (x : \tau) . e \rrbracket\eta &= f : \llbracket \tau \rrbracket\eta 
  \rightarrow \bigcup_{a \in \llbracket \tau \rrbracket\eta} \llbracket \sigma \rrbracket\eta\{x\mapsto a\} \\
\text{ s.t. } \forall a \in \llbracket \tau \rrbracket\eta, f(a) &= \llbracket e \rrbracket\eta\{ x \mapsto a \} 
  \in \llbracket \sigma \rrbracket\eta\{x\mapsto a\}\\
 \llbracket e_1 \ e_2 \rrbracket \eta &= \llbracket e_1 \rrbracket\eta ( \llbracket e_2 \rrbracket\eta ) \\
 \llbracket e_1 + e_2 \rrbracket\eta &= \llbracket e_1 \rrbracket\eta + \llbracket e_2 \rrbracket\eta \\
 \llbracket e_1 - e_2 \rrbracket\eta &= \llbracket e_1 \rrbracket\eta - \llbracket e_2 \rrbracket\eta \\
 \llbracket e_1 * e_2 \rrbracket\eta &= \llbracket e_1 \rrbracket\eta * \llbracket e_2 \rrbracket\eta \\
  \llbracket e_1 / e_2 \rrbracket\eta &=
  \begin{cases}
  \perp \text{ if }  \llbracket e_2 \rrbracket\eta = 0 \\
   \llbracket e_1 \rrbracket\eta / \llbracket e_2 \rrbracket\eta \text{ otherwise}
   \end{cases} \\
  \llbracket \texttt{true} \rrbracket\eta &= true, \ \llbracket \texttt{false} \rrbracket\eta = false \\
 \llbracket e_1 = e_2 \rrbracket\eta &= 
 \begin{cases} 
      true \text{ if } (\llbracket e_1 \rrbracket\eta = \llbracket e_2 \rrbracket\eta \neq \perp) \\
      false \text{  if } (\llbracket e_1 \rrbracket\eta \neq \llbracket e_2\rrbracket\eta, \llbracket e_1 \rrbracket\eta \neq \perp, \llbracket e_2 \rrbracket\eta \neq \perp)\\
      \perp \text{ otherwise}
   \end{cases} \\
\text{Similar rules for $<, \ \leq, >, \geq$} \\
  \llbracket \ifelse{e_1}{e_2}{e_3} \rrbracket \eta &= 
 \begin{cases} 
      \llbracket e_2 \rrbracket\eta \text{ if } \llbracket e_1 \rrbracket\eta = true \\
      \llbracket e_3 \rrbracket\eta \text{ if } \llbracket e_1 \rrbracket\eta = false \\
      \perp \text{      otherwise} \\
   \end{cases}
  \\
    \llbracket  \texttt{fix} (\lambda f.\lambda x.e) \rrbracket\eta &= fix\llbracket \lambda f.\lambda x.e \rrbracket\eta \\
 \text{where } fix &\text{ assigns the least fixed point to continuous functions} 
 \end{align*}
 \caption{Denotational semantics for expressions.}
 \end{figure}
 
 In order to use the CPO fixpoint theorem, we need to prove the same theorems as in chapter 3. These proofs are 
 identical except for the types and expressions that are unique to this language, so we will only consider those cases.
 
 \begin{thm}
For all well-formed types $\tau, \ \llbracket \tau \rrbracket$ is a CPO --- that is, $\llbracket \tau \rrbracket$ is an ordered 
set with a bottom element, such that for $\forall C \subset \llbracket \tau \rrbracket$, if  $C$ is a chain, then $C$ has a 
least upper bound in $\llbracket \tau \rrbracket$. \\
\end{thm}
\begin{proof} 
\begin{itemize}
\item $\tau = \texttt{nat}$ \\
Then $\llbracket \tau \rrbracket\eta$ is flat-ordered, so it is a CPO.
\item $\tau = \lst(e)$ \\
Then $e : \texttt{nat}$ and $\llbracket \tau \rrbracket\eta = \Omega_{\llbracket e \rrbracket\eta}$. This set is flat-ordered,
so $\llbracket \tau \rrbracket$ is a CPO.
\item $\tau = \Pi x:\tau_1.\tau_2$ \\
Then $\llbracket \tau \rrbracket\eta =  \{f: \llbracket \tau_1 \rrbracket\eta \rightarrow 
 \bigcup_{a \in \llbracket \tau_1 \rrbracket\eta} \llbracket \tau_2 \rrbracket\eta\{x \mapsto a\}
 \mid f \text{ is continuous}\}$.  \\ \\
 %
 Need Lemma: $\forall f \in \llbracket \tau \rrbracket\eta, \forall a \in \llbracket \tau_1 \rrbracket\eta, \
 f(a) \in \llbracket \tau_2 \rrbracket\eta\{x \mapsto a \}$. (True b/c of type soundness)  \\ \\
 %
 By inductive hypothesis, $\llbracket \tau_2 \rrbracket\eta\{x\mapsto a\}$ is a CPO
 $\forall a \in \llbracket \tau_2 \rrbracket\eta$, so it has a bottom element. 
 Let $\perp : \llbracket \tau_1 \rrbracket\eta \rightarrow 
 \bigcup_{a \in \llbracket \tau_1 \rrbracket\eta} \llbracket \tau_2 \rrbracket\eta\{x \mapsto a\}$ be a function such that, for 
 $\forall a \in \llbracket \tau_1 \rrbracket, \ \perp(a) = \perp_{\llbracket \tau_2 \rrbracket\eta\{x\mapsto a\}}$. See, then,
 that $\perp$ is a continuous(?) function that is less than or equal to every functions in $\llbracket \tau \rrbracket\eta$,
 so $\perp$ is the bottom element of $\llbracket \tau \rrbracket\eta$.
\end{itemize}
\end{proof}

\begin{thm} 
For all expressions $e$ and environments $\Gamma$,
\begin{enumerate}
\item For all chains $a_0\leq a_1\leq\dotsb$,
\begin{align*}
\tmden{\typing\Gamma e\tau}{\extend\eta x {\bigvee_i a_i}} =
\bigvee_i\tmden{\typing\Gamma e\tau}{\extend\eta x {a_i}}.
\end{align*}
\item For all $\Gamma\vdash e : \tau$ and all $\Gamma$-environments~$\eta$,
$\tmden{\typing\Gamma e\tau}\eta\in\tyden{\tau}\eta$.
\end{enumerate}
\end{thm}

\begin{proof}
\begin{itemize}
 \item $ e = e_1 \ e_2$ \\ \\
Let $\beta = \llbracket \Gamma \vdash e_2 : \sigma \rrbracket$, and see that
\begin{align*}
\llbracket \Gamma \vdash e : [y \mapsto e_2]\tau \rrbracket\eta\{x \mapsto a \} &=
\llbracket \Gamma \vdash e_1 : \Pi y: \sigma.\tau \rrbracket\eta\{x \mapsto a\}
(\beta \ \eta\{x \mapsto a\}) \\
&= \llbracket \Gamma \vdash \lambda (y:\sigma).e'  \rrbracket\eta\{x \mapsto a\}
(\beta \ \eta\{x \mapsto a\}) \\
\text{(by inductive hypothesis)} &= 
\llbracket \Gamma \vdash \lambda (y:\sigma).e' \rrbracket\eta\{x \mapsto a\}
(\bigvee_i \beta \ \eta\{x \mapsto a_i\}) \\
&= \llbracket \Gamma. y : \sigma \vdash e' : \tau \rrbracket\eta\{x \mapsto a\}\{y \mapsto (\bigvee_i \beta \ \eta\{x \mapsto a_i\})\} \\
\text{(by inductive hypothesis)}&= \bigvee_i \llbracket \Gamma. y : \sigma \vdash e' : \tau \rrbracket\eta\{x \mapsto 
a_i\}\{y \mapsto (\beta \ \eta\{x \mapsto a_i\})\} \\
&=\bigvee_i \llbracket \Gamma \vdash e_1: \Pi y:\sigma.\tau\rrbracket\eta\{x\mapsto a_i\}
(\beta \ \eta\{x\mapsto a_i\}) \\
&= \bigvee_i \llbracket \Gamma \vdash e:\tau\rrbracket\eta\{ x \mapsto a_i\} \\
\end{align*}
 \item $e = \lambda (y : \tau) . (e' : \tau')$
 \begin{enumerate}
 \item Then $\llbracket \Gamma \vdash e : \Pi y: \tau.\tau' \rrbracket\eta\{x\mapsto  \bigvee_i a_i\}$ 
 is a function $f: \llbracket \tau \rrbracket \rightarrow \bigcup_{d \in \llbracket \tau \rrbracket\eta\{x\mapsto a\}}\llbracket 
 \tau'  \rrbracket\{y\mapsto d\}$ such that for 
 $\forall d \in \llbracket \tau \rrbracket, \ f(d) = \llbracket \Gamma \ y : \tau \vdash e' : \tau' 
 \rrbracket\eta\{x\mapsto  \bigvee_i a_i\}\{y \mapsto d\}$. 
 
 We want to show that
 \begin{align*}
 \tmden{\typing\Gamma e\tau}{\extend\eta x {\bigvee_i a_i}} =
\bigvee_i\tmden{\typing\Gamma e\tau}{\extend\eta x {a_i}}
\end{align*}
Let $d \in \llbracket \tau \rrbracket$, and consider that
\begin{align*}
\llbracket \Gamma \vdash e : \Pi y: \tau.\tau' \rrbracket\eta\{x\mapsto  \bigvee_i a_i\}(d) 
&= \llbracket \Gamma.y : \tau \vdash e' : \tau'\rrbracket\eta\{x\mapsto  \bigvee_i a_i\}
\{y \mapsto d\} \\
&= \llbracket \Gamma.y : \tau \vdash e' : \tau'\rrbracket(\eta\{y \mapsto d\})\{x\mapsto  \bigvee_i a_i\} \\
\text{(by inductive hypothesis)}&=\bigvee_i\llbracket \Gamma.y:\tau\vdash e':\tau'\rrbracket(\eta\{y\mapsto d\})
\{x\mapsto a_i\} \\
&=\bigvee_i\llbracket \Gamma.y:\tau\vdash e':\tau'\rrbracket(\eta\{x\mapsto a_i\})\{y\mapsto d\} \\
&= \bigvee_i \llbracket \Gamma \vdash \lambda y : \tau.e' : \tau' \rrbracket \eta\{x\mapsto a_i\}(d) \\
&= \bigvee_i \llbracket \Gamma \vdash e : \Pi y:\tau.\tau' \rrbracket \eta\{x\mapsto a_i\}(d) 
\end{align*}
%
 \item We want to show that $\llbracket \Gamma \vdash e : \Pi x:\tau.\tau' \rrbracket\eta \in \llbracket \Pi x:\tau.
 \tau'\rrbracket\eta$---that is, it is a continuous function from $\llbracket \tau \rrbracket\eta$ to 
 $\bigcup_{d\in \llbracket \tau \rrbracket\eta}\llbracket \tau' \rrbracket\eta\{x \mapsto d\}$. Let 
 $\{a_i\}^{\infty}_{i=1}$ be a chain of elements in $\llbracket \tau \rrbracket$, and see that
 \begin{align*}
 \llbracket \Gamma \vdash e : \Pi x:\tau.\tau' \rrbracket\eta(\bigvee_i a_i) &= \llbracket \Gamma.x : \tau 
 \vdash e' : \tau'\rrbracket\eta \{x \mapsto \bigvee_i a_i\} \\
 \text{(by 1)} &= \bigvee_i \llbracket \Gamma.x : \tau \vdash e' : \tau'\rrbracket\eta\{x\mapsto a_i\}\\
 &= \bigvee_i \llbracket \Gamma \vdash e : \Pi x:\tau.\tau' \rrbracket\eta(a_i) 
 \end{align*}
 \end{enumerate} 
\end{itemize}
\end{proof}


\begin{lemma}
For all types $\tau, \ \sigma$, environments $\Gamma$ and $\Gamma$-environments $\eta$,
\begin{align*}
\Pi(x : \tau).\sigma :: * \implies \bigcup_{a \in \llbracket \tau \rrbracket\eta}
\llbracket \sigma \rrbracket\eta\{x\mapsto a\} 
\end{align*}
is a CPO.
\end{lemma}

\begin{proof}
By induction on $\sigma$.
\begin{itemize}
\item $\sigma = \Pi(y: \sigma_1).\sigma_2$ \\
We want to show $\bigcup_{a \in \llbracket \tau \rrbracket\eta} 
\llbracket \Pi(y: \sigma_1).\sigma_2\rrbracket\eta\{x \mapsto a\}$ is a CPO. Note
\begin{align*}
\bigcup_{a \in \llbracket \Gamma \vdash \tau \rrbracket\eta} 
\llbracket \Pi(y: \sigma_1).\sigma_2\rrbracket\eta\{x \mapsto a\}
&= \bigcup_{a \in \llbracket \Gamma \vdash \tau \rrbracket\eta} \{ f : \llbracket \sigma_1 \rrbracket\eta\{x\mapsto a\}
\rightarrow \bigcup_{b \in \llbracket \sigma_1 \rrbracket\eta\{x\mapsto a\}} \llbracket \sigma_2\rrbracket\eta
\{x\mapsto a\}\{y\mapsto b\}\}
\end{align*} 
By induction hypothesis, for $\forall a \in \llbracket \tau \rrbracket\eta,$
\begin{align*}
\bigcup_{b \in \llbracket \sigma_1 \rrbracket\eta\{x\mapsto a\}} \llbracket \sigma_2\rrbracket\eta
\{x\mapsto a\}\{y\mapsto b\}
\end{align*} 
is a CPO --- call this $Q_a$.  Also,  $\llbracket\sigma_1\rrbracket\eta\{x\mapsto a\} $ is a CPO --- call this $P_a$.
\begin{align*}
&= \{ f : \bigcup_{a \in \llbracket\tau\rrbracket\eta} P_a \rightarrow 
\bigcup_{a \in \llbracket \tau \rrbracket\eta} Q_a \mid
d \in P_a \implies f(d) \in Q_a\}
\end{align*}
We can show that this set is a CPO by constructing a bottom element $\perp$ such that for 
$\forall a \in \llbracket\tau\rrbracket\eta, \ \forall d \in P_a, \
\perp(d) = \perp_{Q_a}$, the bottom element of $Q_a$. Also, for any chain $C$ in this set, we can construct
a least upper bound $\bigvee C$ in a similar pointwise fashion. This proof is analogous to the proof that
$\llbracket \Pi(x: \tau).\sigma\rrbracket\eta$ is a CPO.
\end{itemize}
\end{proof}

\section{Example}
We will look at a random variable $T$ describing the cost of quicksort, where the pivot is chosen to be the first element 
of the list. Following Karp's definition, this random variable has the form $T(x)(l) = a(x) + T(h_1(x)(l))(\hat{h_1}(x)(l))
+ T(h_2(x)(l))(\hat{h_2}(x)(l))$, where $a(x) = x-1$, the cost of one step of quicksort (i.e. comparing all other elements
in the list to the pivot), $h_1(x)$ and $h_2(x)$ give the sizes of the left and right sublists of an input, and
$\hat{h_1}$ and $\hat{h_2}$ give the sublists themselves. We may express this function in our expression language
as follows:
\begin{align*}
\texttt{T} = \texttt{fix}(\lambda x. \lambda f. \lambda l. \ifelse{x=0}{0}{((x-1) 
		+ f(\texttt{h\_left} \ x \ l)(\texttt{H\_left} \ x \ l) + f(\texttt{h\_right} \ x \ l)(\texttt{H\_right} \ x \ l)))}
\end{align*}
 
 We want to gain some intuition that the interpretation of this expression corresponds to what we would expect
 the cost of quicksort to be. Let $l$ be a sorted list of length $3$, $F = \lambda x. \lambda f. \lambda l. \ifelse{x = 0}{0}
 {((x-1) + f(\texttt{h\_left} \ x \ l)(\texttt{H\_left} \ x \ l) + f(\texttt{h\_right} \ x \ l)(\texttt{H\_right} \ x \ l))}$  and see that
 \begin{align*}
 \llbracket \texttt{T} \rrbracket &= fix(\llbracket \texttt{F} \rrbracket) \\
 \llbracket \texttt{F} \rrbracket &= F : \llbracket \Pi x.\texttt{real}.\texttt{real}\rrbracket \rightarrow \llbracket \Pi x:\texttt{real} . \texttt{real}\rrbracket \\ 
 &\text{ s.t. } \forall \alpha \in \llbracket \texttt{real}\rrbracket \rightarrow \llbracket \texttt{real} \rrbracket, \\   
 &F(\alpha) = \llbracket\lambda f. \lambda l. ((x-1) 
		+ f(\texttt{h\_left} \ x \ l)(\texttt{H\_left} \ x \ l) + f(\texttt{h\_right} \ x \ l)(\texttt{H\_right} \ x \ l))\rrbracket\{f \mapsto \alpha\} 
 \end{align*}
 By the CPO fixpoint theorem,  
\begin{align*}
 fix(F) &= \bigvee_n(F^n (\perp)) \\
&= \lim_{n \to \infty}\{F^n (\perp)\}
 \end{align*}
 
 
 
 